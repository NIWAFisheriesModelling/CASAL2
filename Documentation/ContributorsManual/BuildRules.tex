\section{\CNAME\ build rules\label{sec:build_rules}}
This section will cover, the standards contributors are expected to follow and the builds that must pass on the local machine in order to add a pull request for changes to be accepted into the 'master' repository.

\subsection{\CNAME\ coding practice and style}
\CNAME\ is written in C++ and follows googles C++ style guide which can be found at \url{https://google.github.io/styleguide/cppguide.html}. Because we are scientists and not computer programmers we don't expect you to read all that standard, but the main things the development team would like you to follow is the current code style. That is good indentations, sensible variable names and note the \_ on the end of variables that are class variables defined in the .h files. \textbf{Annotate} your code, for readability we encourage you to put comments around your code. On topping of annotating your code we encourage developers to add logs (print messages) in the source code. You will see in the source code this already. The purpose for logging out is for debugging purposes. By adding them and running \CNAME\ in log mode you can find out exactly where \CNAME\ is crashing. You can also output equations that would normally be too detailed for end users to be interested in, this is very useful for checking equations are correctly implemented.
\\
There are different levels of logging in \CNAME\ listed below.
\begin{itemize}
	\item LOG\_MEDIUM() 
	\item LOG\_FINE() 
	\item LOG\_FINEST() 
	\item LOG\_TRACE() 
\end{itemize}

To run \CNAME\ in log mode piping out any LOG\_FINEST and coarser logs (LOG\_MEDIUM and LOG\_FINE) you can use the following command,
\\
\texttt{casal2 -r --loglevel finest > MyReports.csl2 2> log.out}
\\
This will output all the logged information to \texttt{log.out}.
\subsection{Unit tests}
One of the key focusses in the \CNAME\ development is the emphasis on software integrity. It's hugely important to ensure results coming from user models are consistent and correct. As part of this we utilise unit tests to check individual components of the software and run entire models verifying results.
\CNAME\ uses:
\begin{itemize}
	\item Google testing framework
	\item Google mocking framework
\end{itemize}

It is advised when adding unit tests that they be validated outside of \CNAME\. For example in \R\ or another program like \CNAME\ e.g. CASAL, Stock Synthesis. An example of how to add a unit test for a process is shown in Section~\ref{sec:example}

\subsection{Reporting (Optional)}
Currently \CNAME\ has reports that are \R\ compatible, that is that all reports produced by \CNAME\ can be read into \R\ using the \textbf{CASAL2} \R\ package. If you create a report you most follow the standard so that the report is \R\ compatible.
\\\\
All reports must start with,
\\
\texttt{*label (type)}
\\ 
and end with,
\\
\texttt{*end}
\\\\
Depending on what type of information you wish to report, will depend on the syntax you need to use. For example
\paragraph*{$\{$d$\}$ (Dataframe)}
Report a dataframe
{\small{\begin{verbatim}
*estimates (estimate_value)
values {d}
process[Recrutiment_BOP].r0 process[Recrutiment_ENLD].r0 
2e+006 8e+006
*end
\end{verbatim}}}

\paragraph*{$\{$m$\}$ (Matrix)}
Report a matrix
{\small{\begin{verbatim}
*covar (covariance_matrix)
Covariance_Matrix {m}
2.29729e+010 -742.276 -70160.5
-110126 -424507 -81300 
-36283.4 955920 -52736.2 
*end
\end{verbatim}}}

\paragraph*{$\{$L$\}$ (List)}
Report a List
{\small{\begin{verbatim}
*weight_one (partition_mean_weight)
year: 1900
ENLD.EN.notag {L}
mean_weights {L}
0.0476604 0.111575 0.199705
end {L}
age_lengths {L}
12.0314 16.2808 20.0135
end {L}
end {L}
*end
\end{verbatim}}}



\subsection{Update manual}
Sections of the user manual are automatically generated from the source code. This is the Syntax section of the user manual. This means contributors need to manually update the sections of the user manual that their changes relate to. This is important if you want the end user to be able to use you're functionality with ease. It can be very difficult is users see functionality in the syntax section but no reference in text, so please keep this in mind.


\subsection{Builds to pass before merging changes}
Once you have made changes to your local repository you must run the following builds before your changes are considered for merging with the 'master' repository.
\\
build the unittest version see Section~\ref{sec:build_environment} for how to build unittest depending on you're system.
\\
\texttt{DoBuild test}
\\
run the unittest's check they all pass
\\
\texttt{casal2} run this command in the \path{CASAL2/BuildSystem/bin/<operating system>/test} folder
\\
\texttt{DoBuild debug}
\\
run the second phase of unitests which are complete model runs
\\
\texttt{DoBuild modelrunner}
\\
this is conditional on having a debug version previously built.
Build the archive which contains
\\
\texttt{DoBuild archive}


