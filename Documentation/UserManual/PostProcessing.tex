\section{\I{Post processing output using \R} \label{sec:post-processing}}\index{Post processing}\index{Post processing section}

\R\ is the main program used to process and visualise output from a \CNAME\ model. \R\ is free and can be downloaded from \url{https://cran.r-project.org/}. Once you have installed \R\ you can load the \cname\ \R\ package for investigating model runs. 

In the downloaded bundle there is a R-package (\texttt{casal2\_1.0.tar.gz}) that reads \CNAME\ output into \R. The \CNAME\ package has multiple functionalities, it can import \CNAME\ output into \R\ as a list. It also has plotting and summarising functions for \CNAME\ reports.

There are three types of output that \CNAME\ will produce and these depend on what type of analysis you are doing. These outputs are Standard output, MCMC output, Derived Quantity outputs. Standard output are the typical reports that are produced in most \CNAME\ runmodes (with the exception of -s and -m). This output can be split into furthur categories where you can have a single parameter run (\texttt{casal2 -r}) or multiparameter run e.g. \texttt{casal2 -r -i many\_pars.out} or running in projection mode (\texttt{-f 1}). Both of the standard output's can be read into \R\ using the \texttt{extract.mpd()} function. The second type of output if generated when doing an MCMC analysis (\texttt{casal2 -m}), which automatically generates two files \texttt{mcmc\_objective.out} and \texttt{mcmc\_sampeles.out}. These reports will be summarised to conclude convergence or chain behaviour, it will also be used to view marginal posteriors and quantify key model parameter uncertainty. The last output type is Derive Quanity outputs or sometimes referred to as tabular output. Usually this is generated after an MCMC is done, and the user wants to see the marginal posteriors for quantities that aren't explicit estimates. For example in fisheries a commonly reported quantity is the spawning stock biomass from the initial year to the final year. These aren't estimated paraemters and so you would need to run the following command to get posteriors for these derived quantities (\texttt{casal2 -r -i mcmc\_samples.out --tabular > Tabular\_report.out}). This can then be read into an \R\ environment using the \texttt{extract.tabular()} function.


\CNAME's reported output is written so that each \command{report} will start with a '*' and end with `*end'. Users can use this format as the basis to construct \R\ or other functions that read \CNAME\ output to identify and read individual reports for post-processing.


These extract functions differ by how the expected output is structured and they each create a different \texttt{casal2} object in \R\ that when run with \texttt{summary, plot} functions will generate different plots. To query an object from an \texttt{extract()} function just type \texttt{class(object)}. The following is a list of available \R\ functions with their descriptions that are available in the \texttt{casal2} R package.

\begin{itemize}
	\item \texttt{extract.mpd()} - import \CNAME\ default output into \R\ as a list
	\item \texttt{extract.mcmc()} - import \CNAME\ mcmc output into \R\ as a list
	\item \texttt{extract.tabular()} - import \CNAME\ tabular output into \R\ as a list
	\item \texttt{extract.parameters()} - import \CNAME\ parameter files into \R\ as a list
	\item \texttt{generate.starting.pars()} - Reads in a file that contains your \command{estimate} blocks and generates 'N' starting values to test 'global convergence'. Pretty handy little bugger I think, encouraged to use.	
	\item \texttt{burn.in.tabular()} - takes a \subcommand{casal2TAB} object and cuts off the first 'n' rows.
	\item \texttt{plot.derived\_quantities()} - plot derived quantities
	\item \texttt{plot.selectivities()} - plot selectivities
	\item \texttt{plot.ycs()} - plot true ycs strengths
	\item \texttt{plot.pressure()} - plot fishing pressures		
	\item \texttt{summary()} - summarise a model run	
	\item \texttt{extract.csl2.file} - read \CNAME\ .csl2 (configuration) file into \R\ as a list
	\item \texttt{write.csl2.file} - write \CNAME\ .csl2 (configuration) file from R
	\item \texttt{ReadSimulatedData} - read \CNAME\ output into \R\ from a \texttt{casal2 -s} run		
	\item \texttt{Method.TA1.8()} return a weighting factor for compositional data. See \cite{francis2011data} for more detail.
	\item \texttt{apply.dataweighting.to.csl2()} - takes a configuration that contains \command{observation} blocks, it applies a weighting factor to a compositional data set and generates a new \subcommand{.csl2} file, with modified effective sample size.
	
\end{itemize}

specific settings for these functions can be queried once you have loaded the \CNAME\ R library into your current R session via (\texttt{library(casal2)}), using the standard \R\ help syntax \texttt{?} (i.e. \texttt{?param.profile()}). Most of the help files should have example data that you can replicate to help understand with syntax.
\\
\paragraph*{Standard diagnostics and plots for model output}
An aspect of \CNAME\ that we have added to aid in userability is an \R\ package that has functions that can summarise and diagnose a model with ease. Stock synthesis is exceptional for having a well tested and used \R\ package, this our attempt of trying to follow suit.

\texttt{plot.derived\_quantity()}


Generally when comparing model outputs we are either comparing different parameters for the same model structure (Option 1) or comparing outputs from multiple model structures (Option 2). This function in theory should work for both of these.

\texttt{plot.selectivities()}
\texttt{plot.pressure()}
\texttt{plot.fit()}
\texttt{plot.ycs()}

\paragraph*{DataWeighting}
An important component of stock assessment is dealing with data conflicts through the use of dataweighting. There are a range of methods for this for a great summary see \cite{francis2011data}. The function implemented in the \CNAME\ R-library is TA1.8, which can be invoked using \texttt{Method.TA1.8()}. I have tried to automate this process for users by adding an assisting function named \subcommand{apply.dataweighting.to.csl2}. This function allows users to automatically apply a weighting factor to a specific compositional \command{observation} block, and generate a new \subcommand{.csl2} file to be re-estimated.

\begin{lstlisting}
library(casal2)

## read in a reported output from a casal2 -e run
## in a command prompt run 'casal2 -e > estimate.log'
## make sure there is @report block for the observation of interest.

Mpd = extract.mpd(estimate.log)
## calculate weighting factor from Francis method
WeightingFactor = Method.TA1.8(model = Mpd, observation_labels = "chatTANage")
## Apply weighting factor to your Observation.csl2 file
apply.dataweighting.to.csl2(weighting_factor = WeightingFactor, Observation_csl2_file = "Observations.csl2", Observation_label = "chatTANage", Observation_out_filename = "Observation.csl2.0")
## this will generate a new Observation.csl.0 file with the re-weighted effective sample size.
\end{lstlisting}
\paragraph*{Trouble Shooting R package}
This section documents some issues with solutions that we have come across when using the R package that may come in handy for future users.

if you get the following error when using one of the extract functions 
\begin{lstlisting}
Read 1 item
Warning messages:
1: In scan(filename, what = "", sep = "\n", fileEncoding = fileEncoding) :
embedded nul(s) found in input
2: In extract.mpd(file = "results.txt", fileEncoding = "") :
File is empty, no reports found
\end{lstlisting}

I have resolved the abover error by trying an alternative UTF formatting using the \subcommand{fileEncoding} parameter e.g,
\begin{lstlisting}
library(casal2)
MyOutput = extract.mpd(file = "Estimate.log", path = getwd(), , fileEncoding = "UTF-16LE")
\end{lstlisting}