\section{Population command and subcommand syntax\label{sec:population-syntax}}
For ease of reading \CNAME\ files in text editors, there exists a syntax highlighter \texttt{CASAL2.syn}
\subsection{\I{Model structure}}
\defComLab{model}{Define an object of type \emph{model}}

\defSub{start\_year} {Define the first year of the model, immediately following initialisation}
\defType{non-negative integer}
\defDefault{No Default}
\defValue{Defines the first year of the model, $\ge 1$, e.g. 1990}

\defSub{final\_year} {Define the final year of the model, excluding years in the projection period}
\defType{non-negative integer}
\defDefault{No Default}
\defValue{Defines the last year of the model, i.e., the model is run from start\_year to final\_year}

\defSub{min\_age} {Minimum age of individuals in the population}
\defType{non-negative integer}
\defDefault{0}
\defValue{$0 \le$ age\textlow{min} $\le$ age\textlow{max}}

\defSub{max\_age} {Maximum age of individuals in the population}
\defType{non-negative integer}
\defDefault{0}
\defValue{$0 \le$ age\textlow{min} $\le$ age\textlow{max}}

\defSub{age\_plus} {Define the oldest age as a plus group}
\defType{boolean}
\defDefault{false}
\defValue{true, false}

\defSub{initialisation\_phases} {Define the labels of the phases of the initialisation}
\defType{string vector}
\defDefault{true}
\defValue{A list of valid labels defined by \texttt{@initialisation\_phase}}

\defSub{time\_steps} {Define the labels of the time steps, in the order that they are applied, to form the annual cycle}
\defType{string vector}
\defDefault{No Default}
\defValue{A list of valid labels defined by \texttt{@time\_step}}

\defSub{projection\_final\_year} {Define the final year of the model in projection mode}
\defType{non-negative integer}
\defDefault{0}
\defValue{Defines the last year of the projection period, i.e., the projection period runs from \texttt{final\_year}$+1$ to \texttt{projection\_final\_year}. For the default, $0$, no projections are run.}

\defSub{length\_bins} {}
\defType{constant vector}
\defDefault{true}

\defSub{base\_weight\_units} {Define the units for the base weight. This will be the default unit of any weight input parameters}
\defType{string}
\defDefault{tonnes}
\defAllowedValues{grams, tonnes, kgs}

 

\subsection{\I{Initialisation}}
\defComLab{initialisation\_phase}{Define an object of type \emph{initialisation\_phase}}

\defSub{label} {The label of the initialisation phase}
\defType{string}
\defDefault{No Default}

\defSub{type} {The type of initialisation}
\defType{string}
\defDefault{iterative}

\subsubsection[Cinitial]{\commandlabsubarg{initialisation\_\_phase}{type}{cinitial}}

\defSub{categories} {The list of categories for the Cinitial initialisation}
\defType{string vector}
\defDefault{No Default}

\subsubsection[Derived]{\commandlabsubarg{initialisation\_\_phase}{type}{derived}}

\defSub{insert\_processes} {Additional processes not defined in the annual cycle, that are to beinserted into this initialisation phase}
\defType{string vector}
\defDefault{true}

\defSub{exclude\_processes} {Processes in the annual cycle to be excluded from this initialisation phase}
\defType{string vector}
\defDefault{true}

\defSub{casal\_intialisation\_switch} {Run an extra annual cycle to evalaute equilibrium SSB's. Warning - if true, this may not correctly evaluate the equilibrium state. Use true if attempting to replicate a legacy CASAL model}
\defType{boolean}
\defDefault{false}

\subsubsection[Iterative]{\commandlabsubarg{initialisation\_\_phase}{type}{iterative}}

\defSub{years} {The number of iterations (years) over which to execute this initialisation phase}
\defType{non-negative integer}
\defDefault{No Default}

\defSub{insert\_processes} {(years) over which to execute this initialisation phase}
\defType{string vector}
\defDefault{true}

\defSub{exclude\_processes} {Processes in the annual cycle to be excluded from this initialisation phase}
\defType{string vector}
\defDefault{true}

\defSub{convergence\_years} {The iteration (year) when the test for converegence (lambda) is evaluated}
\defType{non-negative integer vector}
\defDefault{true}

\defSub{lambda} {The maximum value of the absolute sum of differences (lambda) between the partition at year-1 and year that indicates successfull convergence}
\defType{constant}
\defDefault{0.0}

\subsubsection[State Category By Age]{\commandlabsubarg{initialisation\_\_phase}{type}{state\_category\_by\_age}}

\defSub{categories} {The list of categories for the category state initialisation}
\defType{string vector}
\defDefault{No Default}

\defSub{min\_age} {The minimum age of values supplied in the definition of the category state}
\defType{non-negative integer}
\defDefault{No Default}

\defSub{max\_age} {The minimum age of values supplied in the definition of the category state}
\defType{non-negative integer}
\defDefault{No Default}

 

\subsection{\I{Categories}}
\defComLab{categories}{Define an object of type \emph{categories}}

\defSub{format} {The format that the category names adhere too}
\defType{string}
\defDefault{No Default}

\defSub{names} {The names of the categories to be used in the model}
\defType{string vector}
\defDefault{No Default}

\defSub{years} {The years that individual categories will be active for. This overrides the model values}
\defType{string vector}
\defDefault{true}

\defSub{age\_lengths} {The labels of age\_length objects that are assigned to categories}
\defType{string vector}
\defDefault{true}

 

\subsection{\I{Time-steps}}
\defComLab{time\_step}{Define an object of type \emph{time\_step}}

\defSub{label} {The label of the timestep}
\defType{string}
\defDefault{No Default}

\defSub{processes} {The labels of the processes for this time step in the order that they occur}
\defType{string vector}
\defDefault{No Default}

 

\subsection{\I{Processes}}
\defComLab{process}{Define an object of type \emph{process}}

\defSub{label} {The label of the process}
\defType{string}
\defDefault{No Default}

\defSub{type} {The type of process}
\defType{string}
\defDefault{""}

\defSub{print\_report} {Indicates if a process report should be generated for this process}
\defType{boolean}
\defDefault{false}

\subsubsection[Ageing]{\commandlabsubarg{process}{type}{ageing}}

\defSub{categories} {The labels of the categories}
\defType{string vector}
\defDefault{No Default}

\subsubsection[Maturation]{\commandlabsubarg{process}{type}{maturation}}

\defSub{from} {List of categories to mature from}
\defType{string vector}
\defDefault{No Default}

\defSub{to} {List of categories to mature too}
\defType{string vector}
\defDefault{No Default}

\defSub{selectivities} {List of selectivities to use for maturation}
\defType{string vector}
\defDefault{No Default}

\defSub{years} {The years to be associated with rates}
\defType{non-negative integer vector}
\defDefault{No Default}

\defSub{rates} {The rates to mature for each year}
\defType{constant vector}
\defDefault{No Default}

\subsubsection[Mortality Constant Rate]{\commandlabsubarg{process}{type}{mortality\_constant\_rate}}

\defSub{categories} {List of categories labels}
\defType{string vector}
\defDefault{No Default}

\defSub{m} {Mortality rates}
\defType{constant vector}
\defDefault{No Default}

\defSub{time\_step\_ratio} {Time step ratios for the mortality rates}
\defType{constant vector}
\defDefault{true}

\defSub{selectivities} {List of selectivities for the categories}
\defType{string vector}
\defDefault{No Default}

\subsubsection[Mortality Event]{\commandlabsubarg{process}{type}{mortality\_event}}

\defSub{categories} {Categories}
\defType{string vector}
\defDefault{No Default}

\defSub{years} {Years in which to apply the mortality process}
\defType{non-negative integer vector}
\defDefault{No Default}

\defSub{catches} {The number of removals (catches) to apply for each year}
\defType{constant vector}
\defDefault{No Default}

\defSub{u\_max} {Maximum exploitation rate ($Umax$}
\defType{constant}
\defDefault{0.99}

\defSub{selectivities} {List of selectivities}
\defType{string vector}
\defDefault{No Default}

\defSub{penalty} {The label of the penalty to apply if the total number of removals cannot be taken}
\defType{string}
\defDefault{""}

\subsubsection[Mortality Event Biomass]{\commandlabsubarg{process}{type}{mortality\_event\_biomass}}

\defSub{categories} {Category labels}
\defType{string vector}
\defDefault{No Default}

\defSub{selectivities} {The labels of the selectivities for each of the categories}
\defType{string vector}
\defDefault{No Default}

\defSub{years} {Years in which to apply the mortality process}
\defType{non-negative integer vector}
\defDefault{No Default}

\defSub{catches} {The biomass of removals (catches) to apply for each year}
\defType{constant vector}
\defDefault{No Default}

\defSub{u\_max} {Maximum exploitation rate ($Umax$}
\defType{constant}
\defDefault{0.99}

\defSub{penalty} {The label of the penalty to apply if the total biomass of removals cannot be taken}
\defType{string}
\defDefault{""}

\subsubsection[Mortality Holling Rate]{\commandlabsubarg{process}{type}{mortality\_holling\_rate}}

\defSub{prey\_categories} {Prey Categories labels}
\defType{string vector}
\defDefault{No Default}

\defSub{predator\_categories} {Predator Categories labels}
\defType{string vector}
\defDefault{No Default}

\defSub{is\_abundance} {Is vulnerable amount of prey and predator an abundance [true] or biomass [false]}
\defType{boolean}
\defDefault{true}

\defSub{a} {parameter a}
\defType{constant}
\defDefault{No Default}
\defLowerBound{0.0 (inclusive)}

\defSub{b} {parameter b}
\defType{constant}
\defDefault{No Default}
\defLowerBound{0.0 (inclusive)}

\defSub{x} {This parameter controls the type of functional form, Holling function type 2 (x=2) or 3 (x=3), or generalised (Michaelis Menten, x,=1}
\defType{constant}
\defDefault{No Default}
\defLowerBound{1.0 (inclusive)}

\defSub{u\_max} {Maximum exploitation rate ($Umax$}
\defType{constant}
\defDefault{No Default}
\defLowerBound{0.0 (inclusive)}
\defUpperBound{1.0 (inclusive)}

\defSub{prey\_selectivities} {Selectivities for prey categories}
\defType{string vector}
\defDefault{No Default}

\defSub{predator\_selectivities} {Selectivities for predator categories}
\defType{string vector}
\defDefault{No Default}

\defSub{penalty} {Label of penalty to be applied}
\defType{string}
\defDefault{""}

\defSub{years} {Years in which to apply the mortality process}
\defType{non-negative integer vector}
\defDefault{No Default}

\subsubsection[Mortality Initialisation Event]{\commandlabsubarg{process}{type}{mortality\_initialisation\_event}}

\defSub{categories} {Categories}
\defType{string vector}
\defDefault{No Default}

\defSub{catch} {The number of removals (catches) to apply for each year}
\defType{constant}
\defDefault{No Default}

\defSub{u\_max} {Maximum exploitation rate ($Umax$}
\defType{constant}
\defDefault{0.99}

\defSub{selectivities} {List of selectivities}
\defType{string vector}
\defDefault{No Default}

\defSub{penalty} {The label of the penalty to apply if the total number of removals cannot be taken}
\defType{string}
\defDefault{""}

\subsubsection[Mortality Initialisation Event Biomass]{\commandlabsubarg{process}{type}{mortality\_initialisation\_event\_biomass}}

\defSub{categories} {Categories}
\defType{string vector}
\defDefault{No Default}

\defSub{catch} {The number of removals (catches) to apply for each year}
\defType{constant}
\defDefault{No Default}

\defSub{u\_max} {Maximum exploitation rate ($Umax$}
\defType{constant}
\defDefault{0.99}

\defSub{selectivities} {List of selectivities}
\defType{string vector}
\defDefault{No Default}

\defSub{penalty} {The label of the penalty to apply if the total number of removals cannot be taken}
\defType{string}
\defDefault{""}

\subsubsection[Mortality Instantaneous]{\commandlabsubarg{process}{type}{mortality\_instantaneous}}

\defSub{categories} {Categories for instantaneous mortality}
\defType{string vector}
\defDefault{No Default}

\defSub{m} {Natural mortality rates for each category}
\defType{constant vector}
\defDefault{No Default}
\defLowerBound{0.0 (inclusive)}
\defUpperBound{1.0 (inclusive)}

\defSub{time\_step\_ratio} {Time step ratios for natural mortality}
\defType{constant vector}
\defDefault{true}

\defSub{selectivities} {The selectivities to apply on the categories for natural mortality}
\defType{string vector}
\defDefault{No Default}

\subsubsection[Mortality Prey Suitability]{\commandlabsubarg{process}{type}{mortality\_prey\_suitability}}

\defSub{prey\_categories} {Prey Categories labels}
\defType{string vector}
\defDefault{No Default}

\defSub{predator\_categories} {Predator Categories labels}
\defType{string vector}
\defDefault{No Default}

\defSub{consumption\_rate} {Predator consumption rate}
\defType{constant}
\defDefault{No Default}
\defLowerBound{0.0 (inclusive)}
\defUpperBound{1.0 (inclusive)}

\defSub{electivities} {Prey Electivities}
\defType{constant vector}
\defDefault{No Default}
\defLowerBound{0.0 (inclusive)}
\defUpperBound{1.0 (inclusive)}

\defSub{u\_max} {Umax}
\defType{constant}
\defDefault{No Default}
\defLowerBound{0.0 (inclusive)}
\defUpperBound{1.0 (inclusive)}

\defSub{prey\_selectivities} {Selectivities for prey categories}
\defType{string vector}
\defDefault{No Default}

\defSub{predator\_selectivities} {Selectivities for predator categories}
\defType{string vector}
\defDefault{No Default}

\defSub{penalty} {Label of penalty to be applied}
\defType{string}
\defDefault{""}

\defSub{years} {Year that process occurs}
\defType{non-negative integer vector}
\defDefault{No Default}

\subsubsection[Nop]{\commandlabsubarg{process}{type}{nop}}

\subsubsection[Recruitment Beverton Holt]{\commandlabsubarg{process}{type}{recruitment\_beverton\_holt}}

\defSub{categories} {Category labels}
\defType{string vector}
\defDefault{No Default}

\defSub{r0} {R0}
\defType{constant}
\defDefault{false}

\defSub{b0} {B0}
\defType{constant}
\defDefault{false}

\defSub{proportions} {Proportions}
\defType{constant vector}
\defDefault{No Default}

\defSub{age} {Age to recruit at}
\defType{non-negative integer}
\defDefault{true}

\defSub{ssb\_offset} {Spawning biomass year offset}
\defType{non-negative integer}
\defDefault{true}

\defSub{steepness} {Steepness}
\defType{constant}
\defDefault{1.0}

\defSub{ssb} {SSB Label (derived quantity}
\defType{string}
\defDefault{No Default}

\defSub{b0\_intialisation\_phase} {Initialisation phase Label that b0 is from}
\defType{string}
\defDefault{""}

\defSub{ycs\_values} {YCS Values}
\defType{constant vector}
\defDefault{No Default}

\defSub{ycs\_years} {Recruitment years. A vector of years that relates to the year of the spawning event that created this cohort}
\defType{non-negative integer vector}
\defDefault{false}

\defSub{standardise\_ycs\_years} {Years that are included for year class standardisation}
\defType{non-negative integer vector}
\defDefault{true}

\subsubsection[Recruitment Constant]{\commandlabsubarg{process}{type}{recruitment\_constant}}

\defSub{categories} {Categories}
\defType{string vector}
\defDefault{No Default}

\defSub{proportions} {Proportions}
\defType{constant vector}
\defDefault{true}

\defSub{age} {Age}
\defType{non-negative integer}
\defDefault{No Default}

\defSub{r0} {R0}
\defType{constant}
\defDefault{No Default}
\defLowerBound{0.0 (exclusive)}

\subsubsection[Survival Constant Rate]{\commandlabsubarg{process}{type}{survival\_constant\_rate}}

\defSub{categories} {List of categories}
\defType{string vector}
\defDefault{No Default}

\defSub{s} {Survival rates}
\defType{constant vector}
\defDefault{No Default}

\defSub{time\_step\_ratio} {Time step ratios for S}
\defType{constant vector}
\defDefault{true}

\defSub{selectivities} {Selectivity label}
\defType{string vector}
\defDefault{No Default}

\subsubsection[Tag By Age]{\commandlabsubarg{process}{type}{tag\_by\_age}}

\defSub{from} {Categories to transition from}
\defType{string vector}
\defDefault{No Default}

\defSub{to} {Categories to transition to}
\defType{string vector}
\defDefault{No Default}

\defSub{min\_age} {Minimum age to transition}
\defType{non-negative integer}
\defDefault{No Default}

\defSub{max\_age} {Maximum age to transition}
\defType{non-negative integer}
\defDefault{No Default}

\defSub{penalty} {Penalty label}
\defType{string}
\defDefault{""}

\defSub{u\_max} {U Max}
\defType{constant}
\defDefault{0.99}

\defSub{years} {Years to execute the transition in}
\defType{non-negative integer vector}
\defDefault{No Default}

\defSub{initial\_mortality} {}
\defType{constant}
\defDefault{0}

\defSub{initial\_mortality\_selectivity} {}
\defType{string}
\defDefault{""}

\defSub{loss\_rate} {}
\defType{constant vector}
\defDefault{No Default}

\defSub{loss\_rate\_selectivities} {}
\defType{string vector}
\defDefault{true}

\defSub{selectivities} {}
\defType{string vector}
\defDefault{No Default}

\defSub{n} {}
\defType{constant vector}
\defDefault{true}

\subsubsection[Tag By Length]{\commandlabsubarg{process}{type}{tag\_by\_length}}

\defSub{from} {Categories to transition from}
\defType{string vector}
\defDefault{No Default}

\defSub{to} {Categories to transition to}
\defType{string vector}
\defDefault{No Default}

\defSub{plus\_group} {Use plus group for last length bin}
\defType{boolean}
\defDefault{false}

\defSub{maximum\_length} {The upper length when there is no plus group}
\defType{constant}
\defDefault{0}

\defSub{penalty} {Penalty label}
\defType{string}
\defDefault{""}

\defSub{u\_max} {U Max}
\defType{constant}
\defDefault{0.99}

\defSub{years} {Years to execute the transition in}
\defType{non-negative integer vector}
\defDefault{No Default}

\defSub{initial\_mortality} {}
\defType{constant}
\defDefault{0}

\defSub{initial\_mortality\_selectivity} {}
\defType{string}
\defDefault{""}

\defSub{selectivities} {}
\defType{string vector}
\defDefault{No Default}

\defSub{n} {}
\defType{constant vector}
\defDefault{true}

\subsubsection[Tag Loss]{\commandlabsubarg{process}{type}{tag\_loss}}

\defSub{categories} {List of categories}
\defType{string vector}
\defDefault{No Default}

\defSub{tag\_loss\_rate} {Tag Loss rates}
\defType{constant vector}
\defDefault{No Default}

\defSub{time\_step\_ratio} {Time step ratios for Tag Loss}
\defType{constant vector}
\defDefault{true}

\defSub{tag\_loss\_type} {Type of tag loss}
\defType{string}
\defDefault{No Default}

\defSub{selectivities} {Selectivities}
\defType{string vector}
\defDefault{No Default}

\defSub{year} {The year the first tagging release process was executed}
\defType{non-negative integer}
\defDefault{No Default}

\subsubsection[Transition Category]{\commandlabsubarg{process}{type}{transition\_category}}

\defSub{from} {From}
\defType{string vector}
\defDefault{No Default}

\defSub{to} {To}
\defType{string vector}
\defDefault{No Default}

\defSub{proportions} {Proportions}
\defType{constant vector}
\defDefault{No Default}

\defSub{selectivities} {Selectivity names}
\defType{string vector}
\defDefault{No Default}

\subsubsection[Transition Category By Age]{\commandlabsubarg{process}{type}{transition\_category\_by\_age}}

\defSub{from} {Categories to transition from}
\defType{string vector}
\defDefault{No Default}

\defSub{to} {Categories to transition to}
\defType{string vector}
\defDefault{No Default}

\defSub{min\_age} {Minimum age to transition}
\defType{non-negative integer}
\defDefault{No Default}

\defSub{max\_age} {Maximum age to transition}
\defType{non-negative integer}
\defDefault{No Default}

\defSub{penalty} {Penalty label}
\defType{string}
\defDefault{""}

\defSub{u\_max} {U Max}
\defType{constant}
\defDefault{0.99}

\defSub{years} {Years to execute the transition in}
\defType{non-negative integer vector}
\defDefault{No Default}

 

\subsection{\I{Time varying parameters}}
\defComLab{time\_varying}{Define an object of type \emph{time\_varying}}

\defSub{label} {The time-varying label}
\defType{string}
\defDefault{No Default}

\defSub{type} {The time-varying type}
\defType{string}
\defDefault{""}

\defSub{years} {Years in which to vary the values}
\defType{non-negative integer vector}
\defDefault{No Default}

\defSub{parameter} {The name of the parameter to time vary}
\defType{string}
\defDefault{No Default}

\subsubsection[Annual Shift]{\commandlabsubarg{time\_\_varying}{type}{annual\_shift}}

\defSub{values} {}
\defType{constant vector}
\defDefault{No Default}

\defSub{a} {}
\defType{constant}
\defDefault{No Default}

\defSub{b} {}
\defType{constant}
\defDefault{No Default}

\defSub{c} {}
\defType{constant}
\defDefault{No Default}

\defSub{scaling\_years} {}
\defType{non-negative integer vector}
\defDefault{true}

\subsubsection[Constant]{\commandlabsubarg{time\_\_varying}{type}{constant}}

\defSub{values} {Value to assign to addressable}
\defType{constant vector}
\defDefault{No Default}

\subsubsection[Exogenous]{\commandlabsubarg{time\_\_varying}{type}{exogenous}}

\defSub{a} {Shift parameter}
\defType{constant}
\defDefault{No Default}

\defSub{exogeneous\_variable} {Values of exogeneous variable for each year}
\defType{constant vector}
\defDefault{No Default}

\subsubsection[Linear]{\commandlabsubarg{time\_\_varying}{type}{linear}}

\defSub{slope} {The slope of the linear trend (additive unit per year}
\defType{constant}
\defDefault{No Default}

\defSub{intercept} {The intercept of the linear trend value for the first year}
\defType{constant}
\defDefault{No Default}

\subsubsection[Random Draw]{\commandlabsubarg{time\_\_varying}{type}{random\_draw}}

\defSub{mean} {Mean}
\defType{constant}
\defDefault{0}

\defSub{sigma} {Standard deviation}
\defType{constant}
\defDefault{1}

\defSub{distribution} {distribution}
\defType{string}
\defDefault{normal}
\defAllowedValues{normal, lognormal}

\subsubsection[Random Walk]{\commandlabsubarg{time\_\_varying}{type}{random\_walk}}

\defSub{mean} {Mean}
\defType{constant}
\defDefault{0}

\defSub{sigma} {Standard deviation}
\defType{constant}
\defDefault{1}

\defSub{upper\_bound} {Upper bound for the random walk}
\defType{constant}
\defDefault{1}

\defSub{upper\_bound} {Lower bound for the random walk}
\defType{constant}
\defDefault{1}

\defSub{rho} {Auto Correlation parameter}
\defType{constant}
\defDefault{1}

\defSub{distribution} {distribution}
\defType{string}
\defDefault{normal}



\subsection{\I{Derived quantities}}
\defComLab{derived\_quantity}{Define an object of type \emph{derived\_quantity}}

\defSub{label} {Label of the derived quantity}
\defType{string}
\defDefault{No Default}

\defSub{type} {Type of derived quantity}
\defType{string}
\defDefault{No Default}

\defSub{time\_step} {The time step in which to calculate the derived quantity after}
\defType{string}
\defDefault{No Default}

\defSub{categories} {The list of categories to use when calculating the derived quantity}
\defType{string vector}
\defDefault{No Default}

\defSub{selectivities} {A list of one selectivity}
\defType{string vector}
\defDefault{No Default}

\defSub{time\_step\_proportion} {Proportion through the mortality block of the time step when calculated}
\defType{constant}
\defDefault{0.5}
\defLowerBound{0.0 (inclusive)}
\defUpperBound{1.0 (inclusive)}

\defSub{time\_step\_proportion\_method} {Method for interpolating for the proportion through the mortality block}
\defType{string}
\defDefault{weighted\_sum}
\defAllowedValues{weighted\_sum, weighted\_product}

\subsubsection[Abundance]{\commandlabsubarg{derived\_\_quantity}{type}{abundance}}

\subsubsection[Biomass]{\commandlabsubarg{derived\_\_quantity}{type}{biomass}}

 

\subsection{\I{Age-length relationship}}
\defComLab{age\_length}{Define an object of type \emph{age\_length}}

\defSub{label} {Label of the age length relationship}
\defType{string}
\defDefault{No Default}

\defSub{type} {Type of age length relationship}
\defType{string}
\defDefault{No Default}

\defSub{time\_step\_proportions} {the fraction of the year applied in each time step that is added to the age for the purposes of evaluating the length, i.e., a value of 0.5 for a time step will evaluate the length of individuals at age+0.5 in that time step}
\defType{constant vector}
\defDefault{true}

\defSub{distribution} {The assumed distribution for the growth curve}
\defType{string}
\defDefault{normal}

\defSub{cv\_first} {CV for the first age class}
\defType{estimable}
\defDefault{0.0}
\defLowerBound{0.0 (inclusive)}

\defSub{cv\_last} {CV for last age class}
\defType{estimable}
\defDefault{0.0}
\defLowerBound{0.0 (inclusive)}

\defSub{casal\_switch} {If true, use the (less accurate) equation for the cumulative normal function as was used in the legacy version of CASAL.}
\defType{boolean}
\defDefault{false}

\subsubsection[Data]{\commandlabsubarg{age\_\_length}{type}{data}}

\defSub{external\_gaps} {}
\defType{string}
\defDefault{mean}
\defAllowedValues{mean, nearest\_neighbour}

\defSub{internal\_gaps} {}
\defType{string}
\defDefault{mean}
\defAllowedValues{mean, nearest\_neighbour, interpolate}

\defSub{length\_weight} {The label from an associated length-weight block}
\defType{string}
\defDefault{No Default}

\defSub{by\_length} {Specifies if the linear interpolation of CV's is a linear function of mean length at age. Default is just by age}
\defType{boolean}
\defDefault{true}

\subsubsection[None]{\commandlabsubarg{age\_\_length}{type}{none}}

\subsubsection[Schnute]{\commandlabsubarg{age\_\_length}{type}{schnute}}

\defSub{y1} {Define the y1 parameter of the Schnute relationship}
\defType{estimable}
\defDefault{No Default}

\defSub{y2} {Define the y2 parameter of the Schnute relationship}
\defType{estimable}
\defDefault{No Default}

\defSub{tau1} {Define the $\tau_1$ parameter of the Schnute relationship}
\defType{estimable}
\defDefault{No Default}

\defSub{tau2} {Define the $\tau_2$ parameter of the Schnute relationship}
\defType{estimable}
\defDefault{No Default}

\defSub{a} {Define the $a$ parameter of the Schnute relationship}
\defType{estimable}
\defDefault{No Default}
\defLowerBound{0.0 (inclusive)}

\defSub{b} {Define the $b$ parameter of the Schnute relationship}
\defType{estimable}
\defDefault{No Default}
\defLowerBound{0.0 (exclusive)}

\defSub{length\_weight} {Define the label of the associated length-weight relationship}
\defType{string}
\defDefault{No Default}

\defSub{by\_length} {Specifies if the linear interpolation of CV's is a linear function of mean length at age. Default is just by age}
\defType{boolean}
\defDefault{true}

\subsubsection[Von Bertalanffy]{\commandlabsubarg{age\_\_length}{type}{von\_bertalanffy}}

\defSub{linf} {Define the $L_{infinity}$ parameter of the von Bertalanffy relationship}
\defType{estimable}
\defDefault{No Default}
\defLowerBound{0.0 (inclusive)}

\defSub{k} {Define the $k$ parameter of the von Bertalanffy relationship}
\defType{estimable}
\defDefault{No Default}
\defLowerBound{0.0 (inclusive)}

\defSub{t0} {Define the $t_0$ parameter of the von Bertalanffy relationship}
\defType{estimable}
\defDefault{No Default}

\defSub{length\_weight} {Define the label of the associated length-weight relationship}
\defType{string}
\defDefault{No Default}

\defSub{by\_length} {Specifies if the linear interpolation of CV's is a linear function of mean length at age. Default is just by age}
\defType{boolean}
\defDefault{true}

 

\subsection{\I{Length-weight}}
\defComLab{length\_weight}{Define an object of type \emph{length\_weight}}

\defSub{label} {The label of the length-weight relationship}
\defType{string}
\defDefault{No Default}

\defSub{type} {The type of the length-weight relationship}
\defType{string}
\defDefault{No Default}

\subsubsection[Basic]{\commandlabsubarg{length\_\_weight}{type}{basic}}

\defSub{a} {The $a$ parameter in the basic length-weight relationship}
\defType{constant}
\defDefault{No Default}

\defSub{b} {The $b$ parameter in the basic length-weight relationship}
\defType{constant}
\defDefault{No Default}

\defSub{units} {Units of measure (tonnes, kgs, grams}
\defType{string}
\defDefault{No Default}
\defAllowedValues{tonnes, kgs, grams}

\subsubsection[None]{\commandlabsubarg{length\_\_weight}{type}{none}}



\subsection{\I{Selectivities}}
\defComLab{Selectivity}{Define an object of type \emph{Selectivity}}.
\defRef{sec:Selectivity}
\label{syntax:Selectivity}

\defSub{label}{The label for the selectivity}
\defType{String}
\defDefault{No default}

\defSub{type}{The type of selectivity}
\defType{String}
\defDefault{No default}

\defSub{length\_based}{Is the selectivity length based?}
\defType{Boolean}
\defDefault{false}

\defSub{intervals}{The number of quantiles to evaluate a length-based selectivity over the age-length distribution}
\defType{Non-negative integer}
\defDefault{5}

\defSub{values}{}
\defType{Vector of addressables}
\defDefault{No default}

\defSub{length\_values}{}
\defType{Vector of addressables}
\defDefault{No default}

\subsubsection{Selectivity of type All\_Values}
\commandlabsubarg{Selectivity}{type}{All\_Values}.
\defRef{sec:Selectivity-AllValues}
\label{syntax:Selectivity-AllValues}

\defSub{v}{The v parameter}
\defType{Vector of real numbers (estimable)}
\defDefault{No default}

\subsubsection{Selectivity of type All\_Values\_Bounded}
\commandlabsubarg{Selectivity}{type}{All\_Values\_Bounded}.
\defRef{sec:Selectivity-AllValuesBounded}
\label{syntax:Selectivity-AllValuesBounded}

\defSub{l}{The low value (L)}
\defType{Non-negative integer}
\defDefault{No default}

\defSub{h}{The high value (H)}
\defType{Non-negative integer}
\defDefault{No default}

\defSub{v}{The v parameter}
\defType{Vector of real numbers (estimable)}
\defDefault{No default}

\subsubsection{Selectivity of type Compound\_All}
\commandlabsubarg{Selectivity}{type}{Compound\_All}.
\defRef{sec:Selectivity-CompoundAll}
\label{syntax:Selectivity-CompoundAll}

\defSub{a50}{The mean (mu)}
\defType{Real number (estimable)}
\defDefault{No default}

\defSub{ato95}{The sigma L parameter}
\defType{Real number (estimable)}
\defDefault{No default}
\defLowerBound{0.0 (exclusive)}

\defSub{a\_min}{The sigma R parameter}
\defType{Real number (estimable)}
\defDefault{No default}
\defLowerBound{0.0 (exclusive)}

\defSub{alpha}{The maximum value of the selectivity}
\defType{Real number (estimable)}
\defDefault{1.0}
\defLowerBound{0.0 (exclusive)}

\subsubsection{Selectivity of type Compound\_Left}
\commandlabsubarg{Selectivity}{type}{Compound\_Left}.
\defRef{sec:Selectivity-CompoundLeft}
\label{syntax:Selectivity-CompoundLeft}

\defSub{a50}{The mean (mu)}
\defType{Real number (estimable)}
\defDefault{No default}

\defSub{ato95}{The sigma L parameter}
\defType{Real number (estimable)}
\defDefault{No default}
\defLowerBound{0.0 (exclusive)}

\defSub{a\_min}{The sigma R parameter}
\defType{Real number (estimable)}
\defDefault{No default}
\defLowerBound{0.0 (exclusive)}

\defSub{left\_mean}{The maximum value of the selectivity}
\defType{Real number (estimable)}
\defDefault{1.0}
\defLowerBound{0.0 (exclusive)}

\defSub{sigma}{The maximum value of the selectivity}
\defType{Real number (estimable)}
\defDefault{1.0}
\defLowerBound{0.0 (exclusive)}

\defSub{alpha}{The maximum value of the selectivity}
\defType{Real number (estimable)}
\defDefault{1.0}
\defLowerBound{0.0 (exclusive)}

\subsubsection{Selectivity of type Compound\_Middle}
\commandlabsubarg{Selectivity}{type}{Compound\_Middle}.
\defRef{sec:Selectivity-CompoundMiddle}
\label{syntax:Selectivity-CompoundMiddle}

\defSub{a50}{The mean (mu)}
\defType{Real number (estimable)}
\defDefault{No default}

\defSub{ato95}{The sigma L parameter}
\defType{Real number (estimable)}
\defDefault{No default}
\defLowerBound{0.0 (exclusive)}

\defSub{a\_min}{The sigma R parameter}
\defType{Real number (estimable)}
\defDefault{No default}
\defLowerBound{0.0 (exclusive)}

\defSub{left\_mean}{The maximum value of the selectivity}
\defType{Real number (estimable)}
\defDefault{1.0}
\defLowerBound{0.0 (exclusive)}

\defSub{to\_right\_mean}{The maximum value of the selectivity}
\defType{Real number (estimable)}
\defDefault{1.0}
\defLowerBound{0.0 (exclusive)}

\defSub{sigma}{The maximum value of the selectivity}
\defType{Real number (estimable)}
\defDefault{1.0}
\defLowerBound{0.0 (exclusive)}

\defSub{alpha}{The maximum value of the selectivity}
\defType{Real number (estimable)}
\defDefault{1.0}
\defLowerBound{0.0 (exclusive)}

\subsubsection{Selectivity of type Compound\_Right}
\commandlabsubarg{Selectivity}{type}{Compound\_Right}.
\defRef{sec:Selectivity-CompoundRight}
\label{syntax:Selectivity-CompoundRight}

\defSub{a50}{The mean (mu)}
\defType{Real number (estimable)}
\defDefault{No default}

\defSub{ato95}{The sigma L parameter}
\defType{Real number (estimable)}
\defDefault{No default}
\defLowerBound{0.0 (exclusive)}

\defSub{a\_min}{The sigma R parameter}
\defType{Real number (estimable)}
\defDefault{No default}
\defLowerBound{0.0 (exclusive)}

\defSub{left\_mean}{The maximum value of the selectivity}
\defType{Real number (estimable)}
\defDefault{1.0}
\defLowerBound{0.0 (exclusive)}

\defSub{to\_right\_mean}{The maximum value of the selectivity}
\defType{Real number (estimable)}
\defDefault{1.0}
\defLowerBound{0.0 (exclusive)}

\defSub{sigma}{The maximum value of the selectivity}
\defType{Real number (estimable)}
\defDefault{1.0}
\defLowerBound{0.0 (exclusive)}

\defSub{alpha}{The maximum value of the selectivity}
\defType{Real number (estimable)}
\defDefault{1.0}
\defLowerBound{0.0 (exclusive)}

\subsubsection{Selectivity of type Constant}
\commandlabsubarg{Selectivity}{type}{Constant}.
\defRef{sec:Selectivity-Constant}
\label{syntax:Selectivity-Constant}

\defSub{c}{The constant value}
\defType{Real number (estimable)}
\defDefault{No default}

\subsubsection{Selectivity of type Constant.Mock}
\commandlabsubarg{Selectivity}{type}{Constant.Mock}.
\defRef{sec:Selectivity-Constant.Mock}
\label{syntax:Selectivity-Constant.Mock}

The Constant.Mock type has no additional subcommands.
\subsubsection{Selectivity of type Decreasing}
\commandlabsubarg{Selectivity}{type}{Decreasing}.
\defRef{sec:Selectivity-Decreasing}
\label{syntax:Selectivity-Decreasing}

\defSub{decreasing\_parameter}{Decreasing parameter}
\defType{Real number (estimable)}
\defDefault{0.0}

\defSub{l}{The low value (L)}
\defType{Non-negative integer}
\defDefault{No default}

\defSub{h}{The high value (H)}
\defType{Non-negative integer}
\defDefault{No default}

\defSub{v}{The v parameter}
\defType{Vector of real numbers (estimable)}
\defDefault{No default}

\defSub{alpha}{The maximum value of the selectivity}
\defType{Real number (estimable)}
\defDefault{1.0}
\defLowerBound{0.0 (exclusive)}

\subsubsection{Selectivity of type Double\_Exponential}
\commandlabsubarg{Selectivity}{type}{Double\_Exponential}.
\defRef{sec:Selectivity-DoubleExponential}
\label{syntax:Selectivity-DoubleExponential}

\defSub{x0}{The X0 parameter}
\defType{Real number (estimable)}
\defDefault{No default}

\defSub{x1}{The X1 parameter}
\defType{Real number (estimable)}
\defDefault{No default}

\defSub{x2}{The X2 parameter}
\defType{Real number (estimable)}
\defDefault{No default}

\defSub{y0}{The Y0 parameter}
\defType{Real number (estimable)}
\defDefault{No default}
\defLowerBound{0.0 (exclusive)}

\defSub{y1}{The Y1 parameter}
\defType{Real number (estimable)}
\defDefault{No default}
\defLowerBound{0.0 (exclusive)}

\defSub{y2}{The Y2 parameter}
\defType{Real number (estimable)}
\defDefault{No default}
\defLowerBound{0.0 (exclusive)}

\defSub{alpha}{The maximum value of the selectivity}
\defType{Real number (estimable)}
\defDefault{1.0}
\defLowerBound{0.0 (exclusive)}

\subsubsection{Selectivity of type Double\_Normal}
\commandlabsubarg{Selectivity}{type}{Double\_Normal}.
\defRef{sec:Selectivity-DoubleNormal}
\label{syntax:Selectivity-DoubleNormal}

\defSub{mu}{The mean (mu)}
\defType{Real number (estimable)}
\defDefault{No default}

\defSub{sigma\_l}{The sigma L parameter}
\defType{Real number (estimable)}
\defDefault{No default}
\defLowerBound{0.0 (exclusive)}

\defSub{sigma\_r}{The sigma R parameter}
\defType{Real number (estimable)}
\defDefault{No default}
\defLowerBound{0.0 (exclusive)}

\defSub{alpha}{The maximum value of the selectivity}
\defType{Real number (estimable)}
\defDefault{1.0}
\defLowerBound{0.0 (exclusive)}

\subsubsection{Selectivity of type Double\_Normal\_Plateau}
\commandlabsubarg{Selectivity}{type}{Double\_Normal\_Plateau}.
\defRef{sec:Selectivity-DoubleNormalPlateau}
\label{syntax:Selectivity-DoubleNormalPlateau}

\defSub{sigma\_l}{The sigma L parameter}
\defType{Real number (estimable)}
\defDefault{No default}
\defLowerBound{0.0 (exclusive)}

\defSub{sigma\_r}{The sigma R parameter}
\defType{Real number (estimable)}
\defDefault{No default}
\defLowerBound{0.0 (exclusive)}

\defSub{a1}{The a1 parameter}
\defType{Real number (estimable)}
\defDefault{No default}
\defLowerBound{0.0 (exclusive)}

\defSub{a2}{The a2 parameter}
\defType{Real number (estimable)}
\defDefault{No default}
\defLowerBound{0.0 (exclusive)}

\defSub{alpha}{The maximum value of the selectivity}
\defType{Real number (estimable)}
\defDefault{1.0}
\defLowerBound{0.0 (inclusive)}

\subsubsection{Selectivity of type Double\_Normal\_SS3}
\commandlabsubarg{Selectivity}{type}{Double\_Normal\_SS3}.
\defRef{sec:Selectivity-DoubleNormalSS3}
\label{syntax:Selectivity-DoubleNormalSS3}

\defSub{peak}{beginning size (or age) for the plateau (in cm or year).}
\defType{Real number (estimable)}
\defDefault{No default}
\defLowerBound{0.0 (exclusive)}

\defSub{y0}{Selectivity at first bin, as logistic between 0 and 1}
\defType{Real number (estimable)}
\defDefault{No default}
\defLowerBound{-20 (inclusive)}
\defUpperBound{0.0 (inclusive)}

\defSub{y1}{Selectivity at last bin, as logistic between 0 and 1}
\defType{Real number (estimable)}
\defDefault{No default}
\defLowerBound{-20 (inclusive)}
\defUpperBound{10.0 (inclusive)}

\defSub{ascending}{Ascending width: parameter value is ln(width).}
\defType{Real number (estimable)}
\defDefault{No default}

\defSub{descending}{Descending width: parameter value is ln(width).}
\defType{Real number (estimable)}
\defDefault{No default}

\defSub{width}{width of plateau, as logistic between peak and maximum length (or age)}
\defType{Real number (estimable)}
\defDefault{No default}

\defSub{l}{The low value (L)}
\defType{Real number (estimable)}
\defDefault{No default}
\defLowerBound{0.0 (exclusive)}

\defSub{h}{The high value (H)}
\defType{Real number (estimable)}
\defDefault{No default}
\defLowerBound{1.0 (exclusive)}

\defSub{alpha}{The maximum value of the selectivity}
\defType{Real number (estimable)}
\defDefault{1.0}
\defLowerBound{0.0 (inclusive)}

\subsubsection{Selectivity of type Increasing}
\commandlabsubarg{Selectivity}{type}{Increasing}.
\defRef{sec:Selectivity-Increasing}
\label{syntax:Selectivity-Increasing}

\defSub{l}{The low value (L)}
\defType{Non-negative integer}
\defDefault{No default}

\defSub{h}{The high value (H)}
\defType{Non-negative integer}
\defDefault{No default}

\defSub{v}{The v parameter}
\defType{Vector of real numbers (estimable)}
\defDefault{No default}

\defSub{alpha}{The maximum value of the selectivity}
\defType{Real number (estimable)}
\defDefault{1.0}
\defLowerBound{0.0 (exclusive)}

\subsubsection{Selectivity of type Inverse\_Logistic}
\commandlabsubarg{Selectivity}{type}{Inverse\_Logistic}.
\defRef{sec:Selectivity-InverseLogistic}
\label{syntax:Selectivity-InverseLogistic}

\defSub{a50}{a50}
\defType{Real number (estimable)}
\defDefault{No default}

\defSub{ato95}{ato95}
\defType{Real number (estimable)}
\defDefault{No default}
\defLowerBound{0.0 (exclusive)}

\defSub{alpha}{The maximum value of the selectivity}
\defType{Real number (estimable)}
\defDefault{1.0}
\defLowerBound{0.0 (exclusive)}

\subsubsection{Selectivity of type Knife\_Edge}
\commandlabsubarg{Selectivity}{type}{Knife\_Edge}.
\defRef{sec:Selectivity-KnifeEdge}
\label{syntax:Selectivity-KnifeEdge}

\defSub{e}{The edge value}
\defType{Real number (estimable)}
\defDefault{No default}

\defSub{alpha}{The maximum value of the selectivity}
\defType{Real number (estimable)}
\defDefault{1.0}

\subsubsection{Selectivity of type Logistic}
\commandlabsubarg{Selectivity}{type}{Logistic}.
\defRef{sec:Selectivity-Logistic}
\label{syntax:Selectivity-Logistic}

\defSub{a50}{The a50 parameter}
\defType{Real number (estimable)}
\defDefault{No default}

\defSub{ato95}{The ato95 parameter}
\defType{Real number (estimable)}
\defDefault{No default}
\defLowerBound{0.0 (exclusive)}

\defSub{alpha}{The maximum value of the selectivity}
\defType{Real number (estimable)}
\defDefault{1.0}
\defLowerBound{0.0 (exclusive)}

\subsubsection{Selectivity of type Logistic\_Producing}
\commandlabsubarg{Selectivity}{type}{Logistic\_Producing}.
\defRef{sec:Selectivity-LogisticProducing}
\label{syntax:Selectivity-LogisticProducing}

\defSub{l}{The low value (L)}
\defType{Non-negative integer}
\defDefault{No default}

\defSub{h}{The high value (H)}
\defType{Non-negative integer}
\defDefault{No default}

\defSub{a50}{the a50 parameter}
\defType{Real number (estimable)}
\defDefault{No default}

\defSub{ato95}{The ato95 parameter}
\defType{Real number (estimable)}
\defDefault{No default}
\defLowerBound{0.0 (exclusive)}

\defSub{alpha}{The maximum value of the selectivity}
\defType{Real number (estimable)}
\defDefault{1.0}
\defLowerBound{0.0 (exclusive)}

 

\section{Estimation command and subcommand syntax\label{sec:estimation-syntax}}

\subsection{\I{Estimation methods}}
\defComLab{estimate}{Define an object of type \emph{estimate}}

\defSub{label} {The label of the estimate}
\defType{string}
\defDefault{""}

\defSub{type} {The prior type for the estimate}
\defType{string}
\defDefault{No Default}

\defSub{parameter} {The name of the parameter to estimate in the model}
\defType{string}
\defDefault{No Default}

\defSub{lower\_bound} {The lower bound for the parameter}
\defType{constant}
\defDefault{No Default}

\defSub{upper\_bound} {The upper bound for the parameter}
\defType{constant}
\defDefault{No Default}

\defSub{same} {List of parameters that are constrained to have the same value as this parameter}
\defType{string vector}
\defDefault{""}

\defSub{mcmc} {Indicates if this parameter is fixed at the point estimate during an MCMC run}
\defType{boolean}
\defDefault{false}

\subsubsection[Beta]{\commandlabsubarg{estimate}{type}{beta}}

\defSub{mu} {Beta prior  mean (mu) parameter}
\defType{estimable}
\defDefault{No Default}

\defSub{sigma} {Beta prior variance (sigma) parameter}
\defType{estimable}
\defDefault{No Default}
\defLowerBound{0.0 (exclusive)}

\defSub{a} {Beta prior lower bound of the range (A) parameter}
\defType{constant}
\defDefault{No Default}

\defSub{b} {Beta prior upper bound of the range (B) parameter}
\defType{constant}
\defDefault{No Default}

\subsubsection[Lognormal]{\commandlabsubarg{estimate}{type}{lognormal}}

\defSub{mu} {The lognormal prior mean (mu) parameter}
\defType{estimable}
\defDefault{No Default}
\defLowerBound{0.0 (exclusive)}

\defSub{cv} {The Lognormal variance (CV) parameter}
\defType{estimable}
\defDefault{No Default}
\defLowerBound{0.0 (exclusive)}

\subsubsection[Normal]{\commandlabsubarg{estimate}{type}{normal}}

\defSub{mu} {The normal prior mean (mu) parameter}
\defType{estimable}
\defDefault{No Default}

\defSub{cv} {The normal variance (standard devation) parameter}
\defType{estimable}
\defDefault{No Default}
\defLowerBound{0.0 (exclusive)}

\subsubsection[Normal By Stdev]{\commandlabsubarg{estimate}{type}{normal\_by\_stdev}}

\defSub{mu} {The normal prior mean (mu) parameter}
\defType{estimable}
\defDefault{No Default}

\defSub{sigma} {The normal variance (standard devation) parameter}
\defType{estimable}
\defDefault{No Default}
\defLowerBound{0.0 (exclusive)}

\subsubsection[Normal Log]{\commandlabsubarg{estimate}{type}{normal\_log}}

\defSub{mu} {The normal-log prior mean (mu) parameter}
\defType{estimable}
\defDefault{No Default}

\defSub{sigma} {The normal-log prior variance (standard deviation) parameter}
\defType{estimable}
\defDefault{No Default}
\defLowerBound{0.0 (exclusive)}

\subsubsection[Uniform]{\commandlabsubarg{estimate}{type}{uniform}}

\subsubsection[Uniform Log]{\commandlabsubarg{estimate}{type}{uniform\_log}}



\subsection{\I{Point estimation}}
\defComLab{minimiser}{Define an object of type \emph{minimiser}}

\defSub{label} {The minimiser label}
\defType{string}
\defDefault{No Default}

\defSub{type} {The type of minimiser to use}
\defType{string}
\defDefault{No Default}

\defSub{active} {Indicates if this minimiser is active}
\defType{boolean}
\defDefault{false}

\defSub{covariance} {Indicates if a covariance matrix should be generated}
\defType{boolean}
\defDefault{true}

\subsubsection[A D O L C]{\commandlabsubarg{minimiser}{type}{adolc}}

\defSub{iterations} {Maximum number of iterations}
\defType{integer}
\defDefault{1000}

\defSub{evaluations} {Maximum number of evaluations}
\defType{integer}
\defDefault{4000}

\defSub{tolerance} {Tolerance of the gradient for convergence}
\defType{constant}
\defDefault{0.02}

\defSub{step\_size} {Minimum Step-size before minimisation fails}
\defType{constant}
\defDefault{1e-7}

\subsubsection[Beta Diff]{\commandlabsubarg{minimiser}{type}{betadiff}}

\defSub{iterations} {Maximum number of iterations}
\defType{integer}
\defDefault{1000}

\defSub{evaluations} {Maximum number of evaluations}
\defType{integer}
\defDefault{4000}

\defSub{tolerance} {Tolerance of the gradient for convergence}
\defType{constant}
\defDefault{2e-3}

\subsubsection[C P P A D]{\commandlabsubarg{minimiser}{type}{cppad}}

\subsubsection[D E Solver]{\commandlabsubarg{minimiser}{type}{de\_solver}}

\defSub{population\_size} {The number of candidate solutions to have in the population}
\defType{non-negative integer}
\defDefault{No Default}

\defSub{crossover\_probability} {Define the minimisers crossover probability}
\defType{constant}
\defDefault{0.9}

\defSub{difference\_scale} {The scale to apply to new solutions when comparing candidates}
\defType{constant}
\defDefault{0.02}

\defSub{max\_generations} {The maximum number of iterations to run}
\defType{non-negative integer}
\defDefault{No Default}

\defSub{tolerance} {The total variance between the population and best candidate before acceptance}
\defType{constant}
\defDefault{0.01}

\defSub{method} {The type of candidate generation method to use}
\defType{string}
\defDefault{""}
\defValue{not\_yet\_implemented}

\subsubsection[D Lib]{\commandlabsubarg{minimiser}{type}{d\_lib}}

\subsubsection[Gamma Diff]{\commandlabsubarg{minimiser}{type}{numerical\_differences}}

\defSub{iterations} {Maximum number of iterations}
\defType{integer}
\defDefault{1000}

\defSub{evaluations} {Maximum number of evaluations}
\defType{integer}
\defDefault{4000}

\defSub{tolerance} {Tolerance of the gradient for convergence}
\defType{constant}
\defDefault{0.02}

\defSub{step\_size} {Minimum Step-size before minimisation fails}
\defType{constant}
\defDefault{1e-7}



\subsection{\I{Monte Carlo Markov Chain (MCMC)}\label{sec:estimation-syntax-MCMC}}
\defComLab{mcmc}{Define an object of type \emph{mcmc}}

\defSub{label} {The label of the MCMC}
\defType{string}
\defDefault{No Default}

\defSub{type} {The type of MCMC}
\defType{string}
\defDefault{""}

\defSub{length} {The number of iterations in for the MCMC chain}
\defType{non-negative integer}
\defDefault{No Default}

\defSub{active} {Indicates if this is the active MCMC algorithm}
\defType{boolean}
\defDefault{true}

\defSub{print\_default\_reports} {Indicates if the output prints the default reports}
\defType{boolean}
\defDefault{true}

\defSub{step\_size} {Initial stepsize (as a multiplier of the approximate covariance matrix}
\defType{constant}
\defDefault{0.02}

\subsubsection[Independence Metropolis]{\commandlabsubarg{mcmc}{type}{independence\_metropolis}}

\defSub{start} {Covariance multiplier for the starting point of the MCMC}
\defType{constant}
\defDefault{0.0}

\defSub{keep} {Spacing between recorded values in the MCMC}
\defType{non-negative integer}
\defDefault{1u}

\defSub{max\_correlation} {Maximum absolute correlation in the covariance matrix of the proposal distribution}
\defType{constant}
\defDefault{0.8}

\defSub{covariance\_adjustment\_method} {Method for adjusting small variances in the covariance proposal matrix}
\defType{string}
\defDefault{covariance}

\defSub{correlation\_adjustment\_diff} {Minimum non-zero variance times the range of the bounds in the covariance matrix of the proposal distribution}
\defType{constant}
\defDefault{0.0001}

\defSub{proposal\_distribution} {The shape of the proposal distribution (either the t or the normal distribution}
\defType{string}
\defDefault{t}

\defSub{df} {Degrees of freedom of the multivariate t proposal distribution}
\defType{non-negative integer}
\defDefault{4}

\defSub{adapt\_stepsize\_at} {Iterations in the chain to check and resize the MCMC stepsize}
\defType{non-negative integer vector}
\defDefault{true}

\defSub{adapt\_covariance\_matrix\_at} {Iterations in the chain to check and resize the MCMC stepsize}
\defType{non-negative integer vector}
\defDefault{true}

\defSub{adapt\_stepsize\_method} {Method to adapt step size.}
\defType{string}
\defDefault{ratio}
\defAllowedValues{ratio, double\_half}



\subsection{\I{Profiles}}
\defComLab{profile}{Define an object of type \emph{profile}}

\defSub{label} {Label}
\defType{string}
\defDefault{""}

\defSub{steps} {The number of steps to take between the lower and upper bound}
\defType{non-negative integer}
\defDefault{No Default}

\defSub{lower\_bound} {The lower bounds}
\defType{constant}
\defDefault{No Default}

\defSub{upper\_bound} {The upper bounds}
\defType{constant}
\defDefault{No Default}

\defSub{parameter} {The system parameter to profile}
\defType{string}
\defDefault{No Default}



\subsection{\I{Defining catchability constants}}
\defComLab{catchability}{Define an object of type \emph{catchability}}

\defSub{label} {Label of the catchability}
\defType{string}
\defDefault{No Default}

\defSub{type} {Type of catchability}
\defType{string}
\defDefault{No Default}

\subsubsection[Free]{\commandlabsubarg{catchability}{type}{free}}

\defSub{q} {value of the catchability}
\defType{constant}
\defDefault{No Default}

\subsubsection[Nuisance]{\commandlabsubarg{catchability}{type}{nuisance}}



\subsection{\I{Defining penalties}}
\defComLab{penalty}{Define an object of type \emph{penalty}}

\defSub{label} {The label of the penalty}
\defType{string}
\defDefault{No Default}

\defSub{type} {The type of penalty}
\defType{string}
\defDefault{No Default}

\subsubsection[Process]{\commandlabsubarg{penalty}{type}{process}}

\defSub{multiplier} {The penalty multiplier}
\defType{constant}
\defDefault{1.0}

\defSub{log\_scale} {Indicates if the sums of squares is calculated on the log scale}
\defType{boolean}
\defDefault{false}



\subsection{\I{Defining priors on parameter ratios, differences and means}}
\defComLab{additional\_prior}{Define an object of type \emph{additional\_prior}}

\defSub{parameter} {Name of the parameter to generate additional prior on}
\defType{string}
\defDefault{No Default}

\defSub{label} {Label for teh additional prior}
\defType{string}
\defDefault{No Default}

\defSub{type} {Type of additional prior}
\defType{string}
\defDefault{No Default}

\subsubsection[Beta]{\commandlabsubarg{additional\_\_prior}{type}{beta}}

\defSub{mu} {Beta distribution mean (mu) parameter}
\defType{constant}
\defDefault{No Default}

\defSub{sigma} {Beta distribution variance (sigma) parameter}
\defType{constant}
\defDefault{No Default}
\defLowerBound{0.0 (inclusive)}

\defSub{a} {Beta distribution lower bound of the range (A) parameter}
\defType{constant}
\defDefault{No Default}

\defSub{b} {Beta distribution upper bound of the range (B) parameter}
\defType{constant}
\defDefault{No Default}

\subsubsection[Vector Average]{\commandlabsubarg{additional\_\_prior}{type}{vector\_average}}

\defSub{method} {What calculation method to use, either k, l, or m}
\defType{string}
\defDefault{k}

\defSub{k} {K Value to use in the calculation}
\defType{constant}
\defDefault{No Default}

\defSub{multiplier} {Multiplier for the penalty amount}
\defType{constant}
\defDefault{1}

\subsubsection[Vector Smoothing]{\commandlabsubarg{additional\_\_prior}{type}{vector\_smoothing}}

\defSub{log\_scale} {Should sums of squares be calculated on the log scale?}
\defType{boolean}
\defDefault{false}

\defSub{multiplier} {Multiply the penalty by this factor}
\defType{constant}
\defDefault{1}

\defSub{lower\_bound} {First element to apply the penalty to in the vector}
\defType{non-negative integer}
\defDefault{0u}

\defSub{upper\_bound} {Last element to apply the penalty to in the vector}
\defType{non-negative integer}
\defDefault{0u}

\defSub{r} {Penalty applied to rth differences}
\defType{non-negative integer}
\defDefault{2u}



\section{Observation command and subcommand syntax\label{sec:observation-syntax}}

%Note: this code auto generated by the Casal2 build system. 

\subsection{\I{Observation types}}

The observation types available are,

\begin{description}
  \item Observations of proportions of individuals by age class
  \item Observations of proportions of individuals between categories within each age class
  \item Relative and absolute abundance observations
  \item Relative and absolute biomass observations
\end{description}

Each type of observation requires a set of subcommands and arguments specific to that process.

\section{The observation section\label{sec:observation-section}}

\subsection{\I{Observations}\label{sec:Observations}\index{Observations}}

The objective function is based on the goodness-of-fit of the model to your observations. Observations are typically supplied at an instance in time, over a group of aggregated categories. Most observations are different kinds of time series, i.e., data which were recorded for one or more years, in the same format each year. Examples of time series data types include relative abundance indices, commercial catch length frequencies, and survey numbers-at-age.

The definitions for each type of observation are described below, including how the observed values should be formatted, how \CNAME\ calculates the expected values, and the likelihoods that are available for each type of observation.

There are two types of observations available in \CNAME. The first are observations that are associated with a \textbf{mortality block} and secondly observations that are associated with a specific process. These can be distinguished by the type definition. If an observation type begins with \texttt{process} it is an observation that is associated with a process. If a type does \textbf{not} begin with \texttt{process} it is associated with the mortality block of the time step that you define. For example the observation type \texttt{process\_abundance} is a process based observation vs \texttt{process\_abundance} \texttt{abundance}, which is an observation that is associated with a mortality block.

Process specific observations can also be broken into two types. \textbf{Specific process observations} are observations that are associated to a specific process (e.g. \texttt{process\_proportions\_migrating}), and \textbf{general process observations} are observations that can be associated with any process (e.g. \texttt{process\_proportions\_at\_age}). These tiers of observations have been separated in different sections as to reduce the confusion.

\subsubsection{\I{Mortality block associated observations}}

All observations within this class are calculated in a similar fashion. That is an expectation is calculated at the beginning of the mortality block and at the end of the mortality block. \CNAME\ then uses a linear interpolation to approximate an expectation part way through a mortality block usign the subcommand \texttt{time\_step\_proportion}. This could be useful if a survey occurs part-way through an exploitation phase. For example for if modelling a fish population this may be part-way through a fishing season. Each observation in this class will evaluate different expectations of the partition which will be explained in the following descriptions. A list of observation \texttt{types} that are available with this class of observations are as follows,
\begin{itemize}
	\item \texttt{abundance}
	\item \texttt{biomass}
	\item \texttt{proportions\_at\_age}
	\item \texttt{proportions\_at\_length}	
	\item \texttt{proportions\_by\_category}	
	\item \texttt{tag\_recapture\_by\_length}		
	\item \texttt{tag\_recapture\_by\_age}														
\end{itemize}

\paragraph*{\I{Abundance or biomass observations}}
Abundance (or biomass) observations are observations of either a relative or absolute number (or biomass) of individuals from a set of categories after applying a selectivity. The observations classes are the same, except that a biomass observation will use the biomass as the observed (and expected) value (calculated from mean weight of individuals within each age and category) while an abundance observation is just the number of individuals. 

Each observation is for a given year and time-step, for some selected age classes of the population (i.e., for a range of ages multiplied by a selectivity), for aggregated categories. Further, you need to provide the label of the catchability coefficient $q$, which can either be estimated of fixed. For absolute abundance or absolute biomass observations, define a catchability where $q=1$.

The observations can be supplied for any set of categories. For example, for a model with the two categories \emph{male} and \emph{female}, we might supply an observation of the total abundance/biomass (male $+$ female) or just male abundance/biomass. The subcommand \subcommand{categories} defines the categories used to aggregate the abundance/biomass. In addition, each category must have an associated selectivity, defined by \subcommand{selectivities}. For example,  

{\small{\begin{verbatim}
		categories male
		selectivities male-selectivity
		\end{verbatim}}}

defines an observation for males after applying the selectivity male-selectivity. \CNAME\ then expects that there will be a single observation supplied. The expected values for the observations will be the expected abundance (or biomass) of males, after applying the selectivities, at the year and time-step specified. 

\CNAME\ calculates the expected values by summing over the defined ages (via the age range and selectivity) and categories at both the beginning and end of a mortality block. You can prompt \CNAME\ to approximate the expectation part way through the mortality block using the \texttt{time\_step\_proportion}. The default value \CNAME\ uses us 0.5, which does linear interpolation between the start and end abundance (or biomass) from the mortality block.

For an abundance observation the expectation is calculated as follows,
\begin{equation}\label{eq:expec_1}
E_{i,1} = \sum_{c=1}^{} \sum_{a=1}^{A} S_{a,c} N_{a,c,i,1}
\end{equation}

\begin{equation}\label{eq:expec_2}
E_{i,2} = \sum_{c=1}^{} \sum_{a=1}^{A} S_a N_{a,c,i,2}
\end{equation}

Where $E_{i,1}$ is the expectation at the beginning of time step and $E_{i,2}$ is the expectation at the end of the time-step. $S_a$ is the selectivity for age $a$ and category $c$. If there is no mortality related to this observation then $E_i$ which is used in the likelihood contribution is $E_{i,1}$. If this was a biomass observation we would replace $N_{a,c,i,1}$ in Equation~\eqref{eq:expec_1} and~\eqref{eq:expec_2} with $N_{a,c,i,1} \bar{w}_{a,c}$, where $\bar{w}_{a,c}$ is the mean weight of category $c$ at age $a$. If the user wishes to apply 100\% mortality then $E_i = E_{i,2}$. For applying quantities of mortality between these values ($M_i$), \CNAME\ does the following linear interpolation.
\begin{equation}
E_{i} = |E_{i,1} - E_{i,2}|  M_i
\end{equation}


{\small{\begin{verbatim}
		@observation MyAbundance
		type abundance
		years 1999
		...
		categories male 
		obs 1000
		...
		\end{verbatim}}}

Or, for an observation aggregated over multiple categories,

{\small{\begin{verbatim}
		@observation MyAbundance
		type abundance
		years 1990 1991
		...
		categories male+female
		table obs
		1990 1000
		1991 1200
		end_table
		...
		\end{verbatim}}}


Note that, to define a biomass observation instead of an abundance observation, use 

{\small{\begin{verbatim}
		@observation MyBiomass
		type biomass
		...
		\end{verbatim}}}

\paragraph*{\I{Proportions-at-age}}
Proportions-at-age observations are observations of the relative number of individuals at age, via some selectivity. 

The observation is supplied for a given year and time-step, for some selected age classes of the population (i.e., for a range of ages multiplied by a selectivity), for categories aggregated over a set of spatial cells. Note that the categories defined in the observations must have an associated selectivity, defined by \subcommand{selectivities}.

The age range must be ages defined in the partition (i.e., between \commandsub{model}{min\_age} and \commandsub{model}{max\_age} inclusive), but the upper end of the age range can optionally be a plus group --- which must be either the same or less than the plus group defined for the partition. 


Proportions-at-age observations can be supplied as; 
\begin{enumerate}
	\item a set of proportions for a single category, 
	\item a set of proportions for multiple categories, or
	\item a set of proportions across aggregated categories.
\end{enumerate}

The method of evaluating expectation are the same for all three of these sceneries. We will describe how you define these different scenarios and the expected dimensions of observation and error inputs that \CNAME\ expects for each respective scenario with examples.

Like all types of observations that are associated with the mortality block, \CNAME\ will evaluate the numbers at age before the mortality block (after taking into account a selectivity that the user defines) and after for the specified time step of the observation. \CNAME\ will generate expectations from the partition part way through the mortality block using the subcommand \texttt{time\_step\_proportion}. This approximation is an linear interpolation of the numbers at age over the mortality block. 

Once the interpolation is evaluated \CNAME\ will apply ageing error if the user has specified it. \CNAME\ finally converts numbers at age to proportions at age by dividing all numbers at age bin by the total and sending that to the likelihood to be evaluated.

Defining an observation for a single category is the simplest, and is used to model a set of proportions of a single category by age class. For example, to specify that the observations are of the proportions of male within each age class, then the subcommand \subcommand{categories} for the \commandlabsubarg{observation}{type}{proportion\_by\_age} command is,

{\small{\begin{verbatim}
		categories male
		\end{verbatim}}}

\CNAME\ then expects that there will be a single vector of proportions supplied, with one proportion for each age class within the defined age range, and that these proportions sum to one. 

For example, if the age range was 3 to 10, then 8 proportions should be supplied (one proportion for each of the the ages 3, 4, 5, 6, 7, 8, 9, and 10). The expected values will be the expected proportions of males within each of these age classes (after ignoring any males aged less than 3 or older than 10), after applying a selectivity at the year and time-step specified. The supplied vector of proportions (i.e., in this example, the 8 proportions) must sum to one, which is evaluated with a default tolerance of 0.001. 


{\small{\begin{verbatim}
		@observation MyProportions
		type proportions_at_age
		...
		categories male
		min_age 3
		max_age 9
		years 1990
		table obs
		1990 0.01 0.09 0.20 0.20 0.35 0.10 0.05
		end_table
		...
		\end{verbatim}}}


Defining an observation for multiple categories extends on the single category implementation. It is used to model a set of proportions over several categories by age class. For example, to specify that the observations are of the proportions of male or females within each age class, then the subcommand \subcommand{categories} for the \commandlabsubarg{observation}{type}{proportion\_by\_age} command is,

{\small{\begin{verbatim}
		categories male female
		\end{verbatim}}}

\CNAME\ then expects that there will be a single vector of proportions supplied, with one proportion for each category and age class combination, and that these proportions sum to one across all ages and categories.

For example, if there were two categories and the age range was 3 to 10, then 16 proportions should be supplied (one proportion for each of the the ages 3, 4, 5, 6, 7, 8, 9, and 10, for each category male and female). The expected values will be the expected proportions of males and within each of these age classes (after ignoring those aged less than 3 or older than 10), after applying a selectivity at the year and time-step specified. The supplied vector of proportions (i.e., in this example, the 16 proportions) must sum to one, which is evaluated with a default tolerance of 0.001. 

For example,

{\small{\begin{verbatim}
		@observation MyProportions
		type proportions_at_age
		...
		categories male female
		min_age 1
		max_age 5
		years 1990 1991
		table obs
		1990 0.01 0.05 0.10 0.20 0.20 0.01 0.05 0.15 0.20 0.03
		1991 0.02 0.06 0.10 0.21 0.18 0.02 0.03 0.17 0.20 0.01
		end_table
		...
		\end{verbatim}
		
		
Defining an observation across aggregated categories allows categories to be aggregated before the proportions are calculated. It is used to model a set of proportions from several categories that have been combined by age class. To indicate that two (or more) categories are to be aggregated, separate them with a '+' symbol. For example, to specify that the observations are of the proportions of male and females combined within each age class, then the subcommand \subcommand{categories} for the \commandlabsubarg{observation}{type}{proportion\_by\_age} command is,
		
		{\small{\begin{verbatim}
				categories male + female
				\end{verbatim}}}
		
\CNAME\ then expects that there will be a single vector of proportions supplied, with one proportion for each age class, and that these proportions sum to one. 
		
For example, if there were two categories and the age range was 3 to 10, then 8 proportions should be supplied (one proportion for each of the the ages 3, 4, 5, 6, 7, 8, 9, and 10, for the sum of males and females within each age class). The expected values will be the expected proportions of males + females within each of these age classes (after ignoring those aged less than 3 or older than 10), after applying a selectivity at the year and time-step specified. The supplied vector of proportions (i.e., in this example, the 16 proportions) must sum to one, which is evaluated with a default tolerance of 0.001. 
		
For example,
		
{\small{\begin{verbatim}
				@observation MyProportions
				type proportions_at_age 
				...
				years 1990 1991
				categories male+female
				min_age 1
				max_age 5
				table obs
				1990 0.02 0.13 0.25 0.30 0.30
				1991 0.02 0.06 0.18 0.35 0.39
				end_table
				...
				\end{verbatim}
				
The later form can then be extended to include multiple categories, or multiple aggregated categories. For example, to describe proportions for the three groups: immature males, mature males, and all females (immature and mature females added together) for ages 1--4, a total of 12 proportions are required 
				
{\small{\begin{verbatim}
@observation MyProportions
type proportions_at_age
...
categories male_immature male_mature female_immature+female_mature
min_age 1
max_age 4
years 1990
table obs
year 1990 0.05 0.15 0.15 0.05 0.02 0.03 0.08 0.04 0.05 0.15 0.15 0.08
end_table
...
\end{verbatim}}}


\paragraph*{\I{Proportions-at-length}}
Functionality regarding defining combinations of categories and aggregated categories directly translates over from proportions at age to proportions at length. The difference is the observation is over length bins instead of age-classes. \CNAME\ calculates expectations of numbers at length by converting numbers at age to numbers by length by using the age-length relationship and distribution specified for the category specified in the \command{age\_length} block. Commands that are different are instead of supplying a minimum and maximum age users must supply a vector of length bins. If there is no plus group i.e \texttt{length\_plus\_group=false} \CNAME\ expects a vector of proportions for each year that is $n - 1$, where $n$ is the number of lengths supplied. If \texttt{length\_plus\_group=true} \CNAME\ expects a vector of proportions for each year that is $n$. The last proportion represents the numbers from the last length bin to the maximum length the age-length relationship allows.


{\small{\begin{verbatim}
@observation Observed_Length_frequency_Chat_east
type process_removals_by_length
years 1991 1992
likelihood multinomial
time_step Summer
fishery EastChathamRise
process instant_mort
categories male
length_plus_group false
length_bins 0 20 40 60 80 110
table obs
1991    0.2   0.25    0.15     0.2     0.2 
1992    0.12  0.25    0.28    0.25    0.1 
end_table
table error_values
1991 25
1992 37
end_table  
\end{verbatim}}}
 
\paragraph*{\I{Proportions-by-category observations}\label{sec:proportions-by-category}}
Proportions-by-category observations are observations of either the relative number of individuals between categories within age classes, or relative biomass between categories within age classes. 

The observation is supplied for a given year and time-step, for some selected age classes of the population (i.e., for a range of ages multiplied by a selectivity).

The age range must be ages defined in the partition (i.e., between \commandsub{model}{min\_age} and \commandsub{model}{max\_age} inclusive), but the upper end of the age range can optionally be a plus group --- which may or may not be the same as the plus group defined for the partition. 

Proportions-by-category observations can be supplied for any set of categories as a proportion of themselves and any set of additional categories. For example, for a model with the two categories \emph{male} and \emph{female}, we might supply observations of the proportions of males in the population at each age class. The subcommand \subcommand{categories} defines the categories for the numerator in the calculation of the proportion, and the subcommand \subcommand{categories2} supplies the additional categories to be used in the denominator of the calculation. In addition, each category must have an associated selectivity, defined by \subcommand{selectivities} for the numerator categories and \subcommand{selectivities2} for the additional categories used in the denominator, e.g., 

{\small{\begin{verbatim}
		categories male
		categories2 female
		selectivities male-selectivity
		selectivities2 female-selectivity
		\end{verbatim}}}

defines that the proportion of males in each age class as a proportion of males $+$ females. \CNAME\ then expects that there will be a vector of proportions supplied, with one proportion for each age class within the defined age range, i.e., if the age range was 3 to 10, then 8 proportions should be supplied (one proportion for each of the the ages 3, 4, 5, 6, 7, 8, 9, and 10). The expected values will be the expected proportions of male to male $+$ female within each of these age classes, after applying the selectivities at the year and time-step specified. 

The observations must be supplied using all or some of the values defined by a categorical layer. \CNAME\ calculates the expected values by summing over the ages (via the age range and selectivity) and categories for those spatial cells where the categorical layer has the same value as defined for each vector of observations i.e.,

{\small{\begin{verbatim}
		@observation MyProportions
		type proportions_by_category
		years 1990 1991
		...
		categories male
		categories2 female
		min_age 1
		max_age 5
		table obs
		1990 0.01 0.05 0.10 0.20 0.20
		1991 0.02 0.06 0.10 0.21 0.18
		end_table
		...
		\end{verbatim}}}

\paragraph*{\I{Tag Recapture by length}\label{sec:tag-recapture-by-length}}

Tag data is primarily used to estimate the population abundance of fish. In some models, this estimation can only be made outside the model and the result is used as an estimate of abundance in the model. But in CASAL the tagging data can, alternatively, be fitted within the model.
\\\\
Before adding a tag-recapture time series, you will need to define a tag-release process (Section~\ref{sub:tag_release}). Tagging events list the labels of the tags which are modelled, and define the events where fish are tagged (i.e., \CNAME moves fish into the section of the partition corresponding to a specific tag).
\\\\
The observations are divided into two parts: (i) the number of fish that were scanned, and (ii) the number of tags that were recaptured. Each can be specified by categories, or for combinations of categories. The precise content of the scanned and recaptured observations depends on the sampling method, and the available options are:

\begin{enumerate}
	\item age: both scanned and recaptured are vectors containing numbers-at-age. Only available in an age-based model. The selectivity ogive is redundant and cannot be supplied.
	\item size: both scanned and recaptured are vectors containing numbers-at-size. Can be used in either an age- or size-based model. The selectivity ogive is redundant and cannot be supplied.
\end{enumerate}
When defining the tag-recapture time series, you also need to specify:
\begin{itemize}
	\item the time step,
	\item the years (unlike a tag-release process, the tag-recapture observations can occur over several years),
	\item the probability that each scanned tagged fish is detected as tagged (may be less than 1 if the observers are not infallible). The expected number of tags detected is calculated by multiplying this number by the number of tagged fish in the sample,
	\item the tagged category or categories (Make up the recaptures),
	\item the categories scanned (All the fish sampled for tags),
	\item A selectivity used in the recapture process,
	\item the size classes if the observations are size-based in an age-based model.
\end{itemize}


An example of a tag recapture observation applied in \CNAME\ is shown, below
{\small{\begin{verbatim}
		## For the following partition
		@categories
		format sex.area.tag
		names  		male.Area1.2011,notag female.Area1.2011,notag
		
		@observation Tag_2011_Area1_recap_2012 ## individuals tagged in 2011 and recaptured in 2012
		## in Area1
		type tag_recapture_by_length
		categories *.Area1.2011+  ## male and femaled tagged categories
		categories2 format=*.Area1.*+ ## scanned categories in Area1
		detection 0.85 ## detection probability
		likelihood binomial ## likelihood choice
		selectivities One ## label of selectivity for tagged
		selectivities2 One ## label of selectivity for scanned
		years 2012  ## years to apply observation
		time_step step2  ## time_step to apply observation	
		time_step_proportion 0.5 ## proportion of mortality applied before observation is calculated 
		length_bins 21 30 40 50	## size bins
		plus_group true ## is the last bin a plus group i.e. 50cm +
		
		table scanned
		2012 281271 41360 30239 12234
		end_table
		
		table recaptured
		2012 15 20 12 2
		end_table
		
		delta 1e-11 ## robustification value
		dispersion 6.3	## dispersion factor
		
		\end{verbatim}}}



The tag-recapture likelihood (binomial) is specified below as it is a modified version of the more general binomial. Note that this likelihood does not have any user-set precision parameters such as $N$ or $c.v.$ (though there are user-specified robustification and dispersion parameters available). Note that factorials are calculated using the log-gamma function, to allow for non-integer arguments where necessary (and avoid overflow errors).


\subsubsection{\I{General process observations}}
A list of \texttt{types} that are associated with this set of observations.
\begin{itemize}
	\item \texttt{process\_abundance}
	\item \texttt{process\_biomass}
	\item \texttt{process\_proportions\_at\_age}
	\item \texttt{process\_proportions\_at\_length}	
	\item \texttt{process\_proportions\_by\_category}				
\end{itemize}

These observations have the same expectations as the mortality block versions described in Section~\ref{sec:mortality_block}. With the exception that instead of wrapping a mortality block they can wrap any process type available in \CNAME.

\subsubsection{\I{Specific process observations}}
A list of \texttt{types} that are associated with this set of observations are;
\begin{itemize}
	\item \texttt{process\_removals\_by\_age},
	\item \texttt{process\_removals\_by\_length},
	\item \texttt{process\_proportions\_migrating}.	
\\\\	
\paragraph*{\I{Process removals by age}\label{sec:removals-by-age}}	
Removals at age observations are observations of the relative number of individuals at age, partway through a process of type \texttt{mortality\_instantaneous}. This observation is exclusively associated with the process of type \texttt{mortality\_instantaneous}, and will error out if associated with any other process type.

The observation is supplied for a given year and time-step, for some selected age classes of the population (i.e., for a range of ages multiplied by a selectivity that is associated with the process).

The age range must be ages defined in the partition (i.e., between \commandsub{model}{min\_age} and \commandsub{model}{max\_age} inclusive), but the upper end of the age range can optionally be a plus group --- which must be either the same or less than the plus group defined for the partition. 

The expectations from this observation are generated whilst the process is being executed. The expectation of numbers at age $a$ for category $c$ from exploitation method $m$ ($E[N_{a,c,m}]$) are defined as,


\begin{equation}
E[N_{a,c,m}] = N_{a,c} U_{a,m} S_{a,c,m} 0.5 M_{a,c}
\end{equation}

where, $N_{a,c}$ are the numbers at age in category $c$ before the process is executed, $U_{a,m}$ is the exploitation rate for age $a$ from method $m$. $S_{a,c,m}$ is the selectivity and $M$ is the natural mortality. These are all relevant to the time step which the user defines.
\\\\
The observation class then acquires the variable $E[N_{a,c,m}]$ and applies ageing error if the user has specified it. Then it amalgamates the observations by method and category depending on how the user specifies the observation, before converting numbers at age to proportions and sending them to the likelihood to be evaluated.
\\\\
Likelihoods that are available for this observation class are the mulitnomial, dirichlet and the lognormal. See Section~\ref{sec:likelihood-observations} for information on the respected likelihood.
\\\\
\paragraph*{\I{Process removals by length}\label{sec:removals-by-length}}
Removals by length observations are observations of the relative number of individuals at length, partway through a process of type \texttt{mortality\_instantaneous}. This observation is exclusively associated with the process of type \texttt{mortality\_instantaneous}, and will error out if associated with any other process type.
\\\\
The observation is supplied for a given year and time-step, for some selected age classes of the population (i.e., for a range of ages multiplied by a selectivity that is associated with the process).

The expectations from this observation are generated whilst the process is being executed. The expectation of numbers at age $a$ for category $c$ from exploitation method $m$ ($E[N_{a,c,m}]$) are defined as,


\begin{equation}
E[N_{a,c,m}] = N_{a,c} U_{a,m} S_{a,c,m} 0.5 M_{a,c}
\end{equation}

where, $N_{a,c}$ are the numbers at age in category $c$ before the process is executed, $U_{a,m}$ is the exploitation rate for age $a$ from method $m$. $S_{a,c,m}$ is the selectivity and $M$ is the natural mortality. These are all relevant to the time step which the user defines.

The observation class acquires the variable $E[N_{a,c,m}]$ from the process and applies the age-length relationship specified in the model. This converts numbers at age to numbers at age and length, where \CNAME\ then converts to numbers at length. Then it amalgamates the observations by method and category depending on how the user specifies the observation, before converting numbers at age to proportions and sending them to the likelihood to be evaluated.
{\small{\begin{verbatim}
@observation observation_fishery_LF
type process_removals_by_length
...
years  1993 1994 1995 
method_of_removal FishingEast
mortality_instantaneous_process instant_mort
length_plus_group false
length_bins 0 20 40 60 80 110
delta 1e-5
table obs
1993    0.0   0.05    0.05    0.10    0.80  
1994    0.05  0.1     0.05    0.05    0.75  
1995    0.3   0.4     0.2     0.05    0.05  
end_table

table error_values
1993 31
1994 34
1995 22
end_table   
\end{verbatim}}}

Likelihoods that are available for this observation are the mulitnomial, dirichlet and the lognormal. See Section~\ref{sec:likelihood-observations} for information on the respected likelihood.

\paragraph*{\I{Proportions migrating}\label{sec:Proportions-migrating}}
This observation is of the proportion migrating from one area to another. This observation is exclusively associated with the process type \texttt{transition\_category}, and will error out when trying to associate with any other process type. This observation is used to inform migration rates in migration processes. This observation class is used in the Hoki stock assessment see~\cite{francis_03} for more information on how these observations are collected and the situation you would use it. This observation calculates an expectation $E_a$ of proportions for each age class $a$ that have migrated, by evaluating the following,

\begin{equation}
E_a = \frac{N_a - N_a'}{N_a}
\end{equation}
where, $N_a$ are the numbers of individuals in age $a$ before the migration process occurs and $N_a'$ is the number of individuals after the migration process occurs.
\\
The likelihoods that are allowed for this observation are the lognormal, multinomial and dirichlet.
\\\\
An extract of the Hoki stock assessment is as follows,
{\small{\begin{verbatim}
		@observation pspawn_1993
		type process_proportions_migrating
		years 1993
		time_step step4
		process Wspmg ## migration process that the observation is associated with
		age_plus true
		min_age 4
		max_age 9
		likelihood lognormal
		categories male.west+female.west ## Categories to evaluate the prportion for
		ageing_error Normal_offset ## label for an @ageing_error block
		table obs
		#age    4    5    6    7    8    9
		1993 0.64 0.58 0.65 0.66 0.71 0.60
		end_table
		
		table error_values
		## if lognormal these are c.v.'s
		1993 0.25
		end_table
		\end{verbatim}}}
			
\end{itemize}

\subsection{\I{Likelihoods}\label{sec:likelihood-observations}\index{Likelihoods}}
\subsubsection{Likelihoods for proportions-at-age observations}
\CNAME\ implements three likelihoods for proportions-at-age observations, the multinomial likelihood, dirichlet, and the lognormal likelihood. 

\subsubsection*{The multinomial likelihood\index{Multinomial likelihood}}
For the observed proportions at age $O_i$ for age classes $i$, with sample size $N$, and the expected proportions at the same age classes $E_i$, the negative log-likelihood is defined as; 

\begin{equation}
-\log \left(L \right) =  -\log \left(N! \right) + \sum\limits_i \log \left( \left(NO_i \right)! \right) - NO_i \log \left(Z \left(E_i,\delta \right) \right)
\end{equation}

where $\sum\limits_i O_i = 1$ and $\sum\limits_i E_i = 1$. $Z \left(\theta,\delta \right)$ is a robustifying function to prevent division by zero errors, with parameter $\delta>0$. $Z \left(\theta,\delta \right)$ is defined as,

\begin{equation}
Z \left(\theta,\delta \right) = \begin{cases}
\theta, & \text{where $\theta \ge r$} \\
\delta/\left( 2-\theta/\delta \right), & \text{otherwise} \\  
\end{cases}
\end{equation}

The default value of $\delta$ is $1 \times 10^{-11}$.
\subsubsection*{The dirichlet likelihood\index{Dirichlet likelihood}}

For the observed proportions at age $O_i$ for age classes $i$, with sample size $N$, and the expected proportions at the same age classes $E_i$, the negative log-likelihood is defined as; 

\begin{equation}
-\log \left(L \right) = -\log(\Gamma \sum\limits_i (\alpha_i)) + \sum\limits_i \log(\Gamma (\alpha_i)) - \sum\limits_i (\alpha_i-1) \log(Z(O_i,\delta))
\end{equation}

where $\alpha_i = Z \left(N E_i,\delta \right)$, $\sum\limits_i O_i = 1$, and $\sum\limits_i E_i = 1$. $Z \left(\theta,\delta \right)$ is a robustifying function to prevent division by zero errors, with parameter $\delta>0$. $Z \left(\theta,\delta \right)$ is defined as,

\begin{equation}
Z \left(\theta,\delta \right) = \begin{cases}
\theta, & \text{where $\theta \ge r$} \\
\delta/\left( 2-\theta/\delta \right), & \text{otherwise} \\  
\end{cases}
\end{equation}

The default value of $\delta$ is $1 \times 10^{-11}$.

\subsubsection*{The lognormal likelihood\index{Lognormal likelihood}}

For the observed proportions at age $O_i$ for age classes $i$, with c.v. $c_i$, and the expected proportions at the same age classes $E_i$, the negative log-likelihood is defined as; 

\begin{equation}
- \log \left(L \right) = \sum\limits_i \left( \log \left( \sigma _i \right) + 0.5\left( \frac{\log \left(O_i / Z \left(E_i,\delta \right) \right)}{\sigma_i} + 0.5 \sigma_i \right)^2 \right)
\end{equation}

where 

\begin{equation}
\sigma_i  = \sqrt{\log \left(1+c_i^2 \right)}
\end{equation}

and the $c_i$'s are the c.v.s for each age class $i$, and $Z \left(\theta,\delta \right)$ is a robustifying function to prevent division by zero errors, with parameter $\delta>0$. $Z \left(\theta,\delta \right)$ is defined as,

\begin{equation}
Z \left(\theta,\delta \right) = \begin{cases}
\theta, & \text{where $\theta \ge r$} \\
\delta/\left( 2-\theta/\delta \right), & \text{otherwise} \\  
\end{cases}
\end{equation}

The default value of $\delta$ is $1 \times 10^{-11}$.

\subsubsection{Likelihoods for abundance and biomass observations}\label{Obs:biomass}
Abundance and biomass observations are expected as an annual time series in \CNAME, where they select the same categories over that time series. The parameters and inputs needed to use this observation class are: a observation $O_i$, c.v. $c_i$, catchability coefficient $q$, where $i$ indexed the year. \CNAME\ calculates an expectation $E_i$ and scales it by $q$ before comparing it to $O_i$. This means that the value chosen for $q$ will determine whether the observation is relative ($q\neq 1$) or absolute $q = 1$. Before we describe each of the likelihoods we will discuss the methods available to handle $q's$:

\begin{enumerate}
	\item The $q's$ can be treated as ‘nuisance’ parameters. For each set of values of the free parameters, the model uses the values of the $q's$which minimise the objective function. These optimal $q's$ are calculated algebraically (see Section~\ref{subsec:nuisance}). If one of the $q's$ falls outside the bounds specified by the user, it is set equal to the closest bound. This approach reduces the size of the parameter vector and hence should improve the performance of the estimation method. However, it is not correct when calculating a sample from the posterior in a Bayesian analysis (except asymptotically, see \cite{Walters_ludwig_94}) and we offer the following alternative;
	
	\item The $q's$ can be treated as ordinary free parameters.
\end{enumerate}	
	
For both options, it is necessary to evaluate the contribution of $O_i$ to the negative loglikelihood for a given value of $q$. Each observation $O_i$ varies about $qE_i$ — express the variability of $O_i$ in terms of its c.v. $c_i$ (or in one case, its standard deviation si). Here are the likelihoods, which are expressed on the objective-function scale of -log(L):


\subsubsection*{The lognormal likelihood\index{Lognormal likelihood}\index{Lognormal likelihood}}

The negative log likelihood for a the lognormal is as follows,

\begin{equation}
- \log \left(L \right) = \sum\limits_i \left( \log \left( \sigma _i \right) + 0.5\left( \frac{\log \left(O_i / q Z \left(E_i,\delta \right) \right)}{\sigma_i} + 0.5 \sigma_i \right)^2 \right)
\end{equation}

where 

\begin{equation}
\sigma_i  = \sqrt{\log \left(1+c_i^2 \right)}
\end{equation}

and $Z \left(\theta,\delta \right)$ is a robustifying function to prevent division by zero errors, with parameter $\delta>0$. $Z \left(\theta,\delta \right)$ is defined as,

This reflects the distributional assumptions that  $O_i$ has the lognormal distribution, that the mean of $O_i$ is $qE_i$  and the c.v. of $O_i$ is $c_i$.

\begin{equation}
Z \left(\theta,\delta \right) = \begin{cases}
\theta, & \text{where $\theta \ge r$} \\
\delta/\left( 2-\theta/\delta \right), & \text{otherwise} \\  
\end{cases}
\end{equation}

The default value of $\delta$ is $1 \times 10^{-11}$.

\subsubsection*{The normal likelihood\index{Normal likelihood}\index{Normal likelihood}}

For observations $O_i$, c.v. $c_i$, and expected values $qE_i$, the negative log-likelihood is defined as;

\begin{equation}
- \log \left(L \right) = \sum\limits_i \left( \log \left( c_i E_i \right) +0.5 \left( \frac{O_i-E_i}{Z\left(c_i E_i,\delta \right)}\right)^2\right)
\end{equation}

and $Z \left(\theta,\delta \right)$ is a robustifying function to prevent division by zero errors, with parameter $\delta>0$. $Z \left(\theta,\delta \right)$ is defined as,

\begin{equation}
Z \left(\theta,\delta \right) = \begin{cases}
\theta, & \text{where $\theta \ge r$} \\
\delta/\left( 2-\theta/\delta \right), & \text{otherwise} \\  
\end{cases}
\end{equation}

The default value of $\delta$ is $1 \times 10^{-11}$.

This reflects the distributional assumptions that  $O_i$ has the normal distribution, that the mean of $O_i$ is $qE_i$  and the c.v. of $O_i$ is $c_i$.

\subsubsection{Likelihoods for tag recapture by age and length observations}
\paragraph*{The binomial likelihood\index{Binomial likelihood ! tag-recapture-by-length}}
Designed for situations where the size frequencies or age frequencies of the recaptured tagged fish and of the scanned fish are known. Available in both age or size based models.
\\\\
Here we define the likelihood as a binomial, but based on sizes, rather than ages,
\begin{equation}
\begin{split}
-\log \left(L \right)'= -\sum\limits_i & \left[ \right. \log \left(n_i! \right) - \log \left(\left(n_i - m_i \right)! \right) - \log \left(\left(m_i \right)! \right) + m_i \log \left(Z\left(\frac{M_i}{N_i},\delta \right) \right) \\
&+  \left(n_i - m_i \right)\log \left(Z\left(1 - \frac{M_i}{N_i},\delta\right) \right) \left. \right]
\end{split}
\end{equation}
where 
\\
$n_i$ = number of fish at size or age $i$ that were scanned
\\
$m_i$ = number of fish at size or age $i$ that were recaptured
\\
$N_i$ = number of fish at size or age $i$ in the available population (tagged and untagged)
\\
$M_i$ = number of fish at size or age $i$ in the available population that have the tag after a detection probability $p_d$ has been applied, $M_i = M_i'p_d$, where $M_i'$ is the expected available population that have the tag.
\\\\
where $Z(x,\delta)$ is a robustifying function with parameter $r > 0$ (to prevent division by zero errors), defined as


\[ Z(x,\delta) =
\begin{cases}
x       & \text{where } x \geq \delta\\
\frac{\delta}{(2 - x / \delta)}  & \text{otherwise}\\
\end{cases}
\]

Finally if a dispersion parameter ($\tau$) is described in the observation then the final negative log likelihood $-log(L)$ contribution is,

$$-log(L) = -log(L)' / \tau$$


\subsubsection{Likelihoods for proportions-by-category observations}
\CNAME\ implements two likelihoods for proportions-by-category observations, the binomial likelihood, and the normal approximation to the binomial (binomial-approx). 

\subsubsection*{The binomial likelihood\index{Binomial likelihood ! proportions-by-category}}

For observed proportions $O_i$ for age class $i$, where $E_i$ are the expected proportions for age class $i$, and $N_i$ is the effective sample size for age class $i$, then the negative log-likelihood is defined as;  

\begin{equation}
\begin{split}
-\log \left(L \right)= -\sum\limits_i & \left[ \right. \log \left(N_i! \right) - \log \left(\left(N_i \left(1 - O_i \right) \right)! \right) - \log \left(\left(N_i O_i \right)! \right) + N_i O_i \log \left(Z\left(E_i,\delta \right) \right) \\
&+ N_i \left(1 - O_i \right)\log \left(Z\left(1 - E_i,\delta\right) \right) \left. \right]
\end{split}
\end{equation}


where $Z \left(\theta,\delta \right)$ is a robustifying function to prevent division by zero errors, with parameter $\delta>0$. $Z \left(\theta,\delta \right)$ is defined as,

\begin{equation}
Z \left(\theta,\delta \right) = \begin{cases}
\theta, & \text{where $\theta \ge r$} \\
\delta/\left( 2-\theta/\delta \right), & \text{otherwise} \\  
\end{cases}
\end{equation}

The default value of $\delta$ is $1 \times 10^{-11}$.

\subsubsection*{The normal approximation to the binomial likelihood\index{Binomial likelihood (normal approximation) ! proportions-by-category}}

For observed proportions $O_i$ for age class $i$, where $E_i$ are the expected proportions for age class $i$, and $N_i$ is the effective sample size for age class $i$, then the negative log-likelihood is defined as;  

\begin{equation}
-\log \left(L \right)= \sum\limits_i \log \left( \sqrt{Z\left(E_i,\delta \right)Z\left(1-E_i,\delta\right)/N_i} \right)     + \frac{1}{2} \left( \frac{O_i-E_i}{\sqrt{Z\left(E_i,\delta\right)Z\left(1-E_i,\delta \right)/N_i}} \right)^2
\end{equation}

where $Z \left(\theta,\delta \right)$ is a robustifying function to prevent division by zero errors, with parameter $\delta>0$. $Z \left(\theta,\delta \right)$ is defined as,

\begin{equation}
Z \left(\theta,\delta \right) = \begin{cases}
\theta, & \text{where $\theta \ge r$} \\
\delta/\left( 2-\theta/\delta \right), & \text{otherwise} \\  
\end{cases}
\end{equation}

The default value of $\delta$ is $1 \times 10^{-11}$.

\subsection{\I{Process error}}

Additional `process error' can be defined for each set of observations. Additional process error has the effect of increasing the observation error in the data, and hence of decreasing the relative weight given to the data in the fitting process. 

For observations where where the likelihood is parameterised by the c.v., you can specify the process error for a given set of observations as a c.v., in which case all the c.v.s $c_i$ are changed to

\begin{equation}
  c'_i  = \sqrt {c_i^2  + c_{process\_error}^2 } 
\end{equation}

Note that $c_{process\_ error} \ge 0$, and that $c_{process\_ error} = 0$ is equivalent to no process error.

Similarly, if the likelihood is parameterised by the effective sample size $N$,

\begin{equation}
 N'_i  = \frac{1}{1 / {N_i}+ 1 / N_{process\_error}}
\end{equation}

Note that this requires that $N_{process\_ error} > 0$, but we allow the special case of $N_{process\_ error}=0$, and define $N_{process\_ error}=0$ as no process error (i.e., defined to be equivalent to $N_{process\_ error}=\infty$). 

For both the c.v. and $N$ process errors, the process error has more effect on small errors than on large ones. Be clear that a large value for the $N$ process error means a small process error.

\subsection{\I{Calculating nuisance q's}}\label{subsec:nuisance}
This section describes the equations used to calculate nuisance catchability coefficients $q’s$ (see Section~\ref{Obs:biomass}). From the user's point of view, the essence is that you can use nuisance $q’s$ in the following situations:
\begin{enumerate}
	\item With maximum likelihood.
	\item With Bayesian estimation, providing that your prior on the q is one of the following:
		\begin{itemize} 
			\item Uniform
			\item Uniform-log
			\item Lognormal with observations distributed lognormal, robustified lognormal
		\end{itemize}
\end{enumerate}
Table~\ref{tab:nus_overview} displays the scenarios when the nuisance catchability can be used for a Bayesian analysis.

\begin{table}[h!]
	\caption{\textbf{Equations used to calculate nuisance $q$'s. (*=no analytic solution found.)}}\label{tab:nus_overview}
	\begin{tabular}{cccccc}
		Distribution of observations & Maximum Likelihood & Uniform & Uniform-log & Normal & lognormal\\
		\hline
		Normal & \eqref{EQ:1} & \eqref{EQ:1} & \eqref{EQ:3} & \textbf{*} & \textbf{*} \\
		Lognormal & \eqref{EQ:4} & \eqref{EQ:4} & \eqref{EQ:8} & \textbf{*} & \eqref{EQ:9} \\				
	\end{tabular}
\end{table}
%% Insert table 4 here
Note that $q’s$ are calculated for robustified lognormal likelihoods as if they were ordinary lognormal likelihoods.
\\\\
The equations and their derivations follow. Let $\sigma_i = \sqrt{log(1 + c_i^2)}$ throughout, and let $n$ be the number of observations in the time series. The case of multiple time series sharing the same $q$, and the modifications required for the assumption of curvature, are addressed at the end of this subsection.
\\\\
First, consider maximum likelihood estimation. When the ($Oi$) are assumed to be normally
distributed,
\begin{equation}\label{EQ:1}
-log(L) = \sum_i log (c_iqE_i) + 0.5\sum_i \bigg(\frac{O_i - qE_i}{c_iqE_i} \bigg)^2
\end{equation}
The value of $q$ which minimises the objective function is found by solving $\partial/\partial q(-log(L))$

\begin{equation}\label{EQ:2}
\frac{\partial }{\partial q}(-log(L)) = \frac{n}{q} + \frac{1}{q^2} \sum_i \frac{O_i}{c_i^2E_i} - \frac{1}{q^3} \sum_i \bigg(\frac{O_i}{c_iE_i}\bigg)^2
\end{equation}
hence
\begin{equation}\label{EQ:3}
\hat q = \frac{-S_1 + \sqrt{S_1^2 + 4nS_2}}{2n} 
\end{equation}
where $S_1 = \sum_i (O_i/c_i^2E_i)$ and $S_2 = \sum_i (O_i/c_iE_i)^2$
\\\\
When the ($O_i$) are assumed to be lognormally distributed,
\begin{equation}\label{EQ:4}
-log(L) = \sum_i log (\sigma_i) + 0.5\sum_i \bigg(\frac{log(O_i) - log(qE_i) + 0.5\sigma_i^2}{\sigma_i} \bigg)^2
\end{equation}
\begin{equation}\label{EQ:5}
\frac{\partial }{\partial q}(-log(L)) = \frac{-1}{q} \sum_i\bigg( \frac{log(O_i/E_i) - log(q) + 0.5\sigma_i^2}{\sigma_i^2}\bigg)
\end{equation}

\begin{equation}\label{EQ:6}
\hat q = exp\frac{0.5n + S_3}{S_4} 
\end{equation}

where $S_3 = \sum_i (log(O_i /E_I)/\sigma_i^2)$ and $S_4 = \sum_i(1/\sigma_i^2)$ 
\\\\
Next consider Bayesian estimation, where we must also specify a prior for $q$.
\\\\
The effects of the prior on the equations are to replace likelihood $L$ by posterior $P$ throughout, to add $-log(\pi(q))$ to the equation for $-log(P)$ and $\partial/\partial q(-log(-\pi(q)))$ to the equation for $\partial/\partial q(-log(P))$
\\\\
This last term is 0 for a uniform prior on $q$, $1/q$ for a log-uniform prior, and $\frac{1}{q}\bigg( 1.5 + \frac{log(q) - log(\mu_q)}{\sigma_q^2}$ for a lognormal prior, 
\\\\
where $\mu_q$ and $c_q$ are the mean and c.v of the prior on $q$ and $\sigma_q = \sqrt{log(1+c_q^2)}$. Clearly, if the prior is uniform, the equation for $\hat q$ is teh same as teh maximum likelihood estimation.
\\\\
When the $(O_i)$ are assumed to be normally distributed and teh prior is log-uniform equation~\eqref{EQ:3} becomes,

\begin{equation}\label{EQ:7}
\hat q = \frac{-S_1 + \sqrt{S_1^2 + 4(n + 1)S_2}}{2(n+1)} 
\end{equation}

but we cannot solve for $\hat q$ with either a normal or lognormal prior.
\\\\
When the $O_i$ are assumed to be lognormally distributed and the prior is log-uniform, equation~\eqref{EQ:6} becomes 


\begin{equation}\label{EQ:8}
\hat q = exp\frac{0.5n -1 + S_3}{S_4} 
\end{equation}

and if the prior is lognormal,

\begin{equation}\label{EQ:9}
\hat q = exp\frac{0.5n -1.5 + log(\mu_q)/\sigma_q^2 + S_3}{S_4 + 1 / \sigma_q^2} 
\end{equation}
but it is not possible to solve for $\hat q$ with a normal prior.
\subsection{\I{Ageing error}}

\CNAME\ can apply ageing error to expected age frequency generated by the model. The ageing error is applied as a misclassification matrix, which has the effect of 'smearing' the expected age frequencies. This is mimicking the error involved in identifying the age of individuals. For example fish species are aged by reading the ear bones (otoliths) which can be quite difficult depending on the species. These are used in calculating the fits to the observed values, and hence the contribution to the total objective function. 

Ageing error is optional, and if it is used, it may be omitted for any individual time series. Different ageing error models may be applied for different observation commands. See Section \ref{sec:ageingerrorreport} for reporting the misclassification matrix at the end of model run.

The ageing error models implemented are,
\begin{enumerate}
  \item{None}: The default model is to apply no ageing error.
  \item{Off by one}: Proportion $p_1$ of individuals of each age $a$ are misclassified as age $a-1$ and proportion $p_2$ are misclassified as age $a+1$. Individuals of age $a < k$ are not misclassified. If there is no plus group in the population model, then proportion $p_2$ of the oldest age class will 'fall off the edge' and disappear. 
  \item{Normal}: Individuals of age $a$ are classified as ages which are normally distributed with mean $a$ and constant c.v. $c$. As above, if there is no plus group in the population model, some individuals of the older age classes may disappear. If $c$ is high enough, some of the younger age classes may 'fall off the other edge'. Individuals of age $a < k$ are not misclassified.
\end{enumerate}

Note that the expected values (fits) reported by \CNAME\ for observations with ageing error will have had the ageing error applied. 



\subsection{\I{Simulating observations}\label{sec:simulation-observations}}

\CNAME\ can generate simulated observations for a given model with given parameter values using \texttt{casal2 -s 1} (To simulate one set of simulated observations). Simulated observations are randomly distributed values, generated according to the error assumptions defined for each observation, around fits calculated from one or more sets of the 'true' parameter values. Simulating from a set of parameters can be used to generate observations from an operating model or as a form of parametric bootstrap. 

The procedure \CNAME\ uses for simulating observations is to first run using the `true' parameter values and generate the expected values. Then, if a set of observations uses ageing error, ageing error is applied. Finally a random value for each observed value is generated based on (i) the expected values, (ii) the type of likelihood specified, and (iii) the variability parameters (e.g., \subcommand{error\_value} and \subcommand{process\_error}). 

Methods for generating the random error, and hence simulated values, depend on the specific likelihood type of each observation. 

\begin{enumerate}
  \item{} Normal likelihood parameterised by c.v.: Let $E_{i}$ be the fitted value for observation $i$, and $c_i$ be the corresponding c.v. (adjusted by the process error if applicable). Each simulated observation value $S_i$ is generated as an independent normal deviate with mean $E_i$ and standard deviation $E_i c_i$.
  \item{} Log-normal likelihood: Let $E_i$ be the fitted value for observation $i$ and $c_i$ be the corresponding c.v. (adjusted by the process error if applicable). Each simulated observation value $S_i$ is generated as an independent lognormal deviate with mean and standard deviation (on the natural scale, not the log-scale) of $E_i$ and $E_i c_i$ respectively. The robustification parameter $\delta$ is ignored.
  \item{} Multinomial likelihood: Let $E_i$ be the fitted value for observation $i$, for $i$ between $1$ and $n$, and let $N$ be the sample size (adjusted by process error if applicable, and then rounded up to the next whole number). The robustification parameter $\delta$ is ignored. Then, 
  \begin{enumerate}
    \item{} A sample of $N$ values from $1$ to $n$ is generated using the multinomial distribution, using sample probabilities proportional to the values of $E_i$.
    \item{} Each simulated observation value $S_i$ is calculated as the proportion of the $N$ sampled values equalling $i$
    \item{} The simulated observation values $S_i$ are then rescaled so that their sum is equal to $1$
  \end{enumerate}
\item{} Binomial and the normal approximation to the binomial likelihoods: Let $E_i$ be the fitted value for observation $i$, for $i$ between $1$ and $n$, and $N_i$ the corresponding equivalent sample size (adjusted by process error if applicable, and then rounded up to the next whole number). The robustification parameter $\delta$ is ignored. Then, 
  \begin{enumerate}
    \item{} A sample of $N_i$ independent binary variates is generated, equalling $1$ with probability $E_i$ 
    \item{}	The simulated observation value $S_i$ is calculated as the sum of these binary variates divided by $N_i$
  \end{enumerate}
\end{enumerate}

\textbf{An important note:} \CNAME\ will not automatically report simulated observations, the user must write a report using the \texttt{simulated\_observation} report (\commandlabsubarg{report}{type}{observation}). See Section \ref{sec:report-section} for more information on how to write this report.




\subsection{\I{Pseudo-observations}}
\CNAME\ can generate expected values for observations without them contributing to the total objective function. These are called pseudo-observations, and can be used to either generate the expected values from \CNAME\ for reporting or diagnostic purposes. To define an observation as a pseudo-observation, use the command \commandlabsubarg{observation}{likelihood}{none}. Any observation type can be used as a pseudo-observation. \CNAME\ can also generate simulated observations from pseudo-observations. Note that;

\begin{itemize}
  \item Output will only be generated if a report command \commandlabsubarg{report}{type}{observation} is specified.
  \item The observed values should be supplied (even if they are `dummy' observation). These will be processed by \CNAME\ as if they were actual observation values, and must conform to the validations carried out for the other types of likelihood. 
  \item The subcommands \subcommand{likelihood}, \subcommand{obs}, \subcommand{error\_value} and \subcommand{process\_error} have no effect when generating the expected values for the pseudo-observation.   
  \item When simulating observations, \CNAME\ needs the subcommand \subcommand{simulation\_likelihood} to tell it what sort of likelihood to use. In this case, the \subcommand{obs}, \subcommand{error\_value} and \subcommand{process\_error} are used to determine the appropriate terms to use for the likelihood when simulating.
\end{itemize}

\subsection{\I{Residuals}}\label{sec:Residuals}}

\CNAME\ will only print the usual residual (i.e. observed less fitted) using the report type \command{report}.type=observation. For an observation \textit{O} and \textit{F} the corresponding fit (=\textit{qE} for relative observations), then
\begin{itemize}
	\item Residuals = \textit{O} - \textit{F}
\end{itemize}

Pearson and Normalised residuals can be generated using \CNAME\ \textbf{R} package with-in the \textbf{R} environment. For specific R functions see Section~\ref{sec:post-processing}. The definitions used in the calculations are as follows,

\begin{enumerate}
	\item \textit{Pearson residuals} attempt to express the residual relative to the variability of the observation, and are defined as (\textit{O}-\textit{F})/std.dev.(\textit{O}), where std.dev.(\textit{O}) is calculated as
	\begin{itemize}
			\item F $\times$ cv for normal, lognormal, robustified lognormal, and normal-log error distributions.
			\item s for normal-by-standard deviation error distributions.
			\item $\sqrt{\frac{Z(\textit{F},r)(1 - Z(\textit{F},r))}{N}}$ for multinomial or binomial likelihoods.
			\item $\sqrt{\frac{(\textit{F} + r)(1 - \textit{F} + r)}{N}}$ for binomial-approx likelihood likelihoods.
	\end{itemize}
	\item \textit{Normalised residuals} to express the residual on a standard normal scale, and are defined as:
	\begin{itemize}
		\item Equal to the Pearson residuals for normal error distributions.
		\item (log(\textit{O}/\textit{F})+0.5$\sigma^2$)/$\sigma$ for lognormal (including robustified lognormal) error distributions, where $\sigma= \sqrt{log(1 + cv^2)}$.
		\item  log(\textit{O}/\textit{F})/$\sigma$ for normal-log error distributions, again with $\sigma= \sqrt{log(1 + cv^2)}$.
		\item And are otherwise undefined.
	\end{itemize}	
\end{enumerate}

where $Z(\textit{F},r)$ is the robustifying term on \textit{F} (fit or expectation of the observation). This robustifying is described earlier in the likelihood section.




\subsection{\I{Likelihoods}}
\defComLab{likelihood}{Define an object of type \emph{likelihood}}

\subsubsection[Binomial]{\commandlabsubarg{likelihood}{type}{binomial}}

\subsubsection[Binomial Approx]{\commandlabsubarg{likelihood}{type}{binomial\_approx}}

\subsubsection[Dirichlet]{\commandlabsubarg{likelihood}{type}{dirichlet}}

\subsubsection[Log Normal]{\commandlabsubarg{likelihood}{type}{log\_normal}}

\subsubsection[Log Normal With Q]{\commandlabsubarg{likelihood}{type}{log\_normal\_with\_q}}

\subsubsection[Logistic Normal]{\commandlabsubarg{likelihood}{type}{logistic\_normal}}

\defSub{label} {Label for Logisitic Normal Likelihood}
\defType{string}
\defDefault{No Default}

\defSub{type} {Type of likelihood}
\defType{string}
\defDefault{No Default}

\defSub{rho} {The auto-correlation parameter $\rho$}
\defType{constant vector}
\defDefault{No Default}

\defSub{sigma} {Sigma parameter in the likelihood}
\defType{constant}
\defDefault{No Default}

\defSub{arma} {Defines if two rho parameters supplied then covar is assumed to have the correlation matrix of an ARMA(1,1) process}
\defType{boolean}
\defDefault{No Default}

\defSub{bin\_labels} {If no covariance matrix parameter then list a vector of bin labels that the covariance matrix will be built for, can be ages or lengths.}
\defType{non-negative integer vector}
\defDefault{false}

\defSub{sexed} {Will the observation be split by sex?}
\defType{boolean}
\defDefault{false}

\defSub{seperate\_by\_sex} {If data is sexed, shold the correlation structure be seperated by sex?}
\defType{boolean}
\defDefault{false}

\subsubsection[Multinomial]{\commandlabsubarg{likelihood}{type}{multinomial}}

\subsubsection[Normal]{\commandlabsubarg{likelihood}{type}{normal}}

\subsubsection[Pseudo]{\commandlabsubarg{likelihood}{type}{pseudo}}



\subsection{\I{Defining ageing error}}

Three methods for including ageing error into estimation with observations are,

\begin{itemize}
	\item None
	\item Normal
	\item Off-by-one
\end{itemize}

Each type of ageing error requires a set of subcommands and arguments specific to its type.

\defComLab{ageing\_error}{Define an object of type \emph{ageing\_error}}

\defSub{label} {Label of the ageing error}
\defType{string}
\defDefault{No Default}

\defSub{type} {Type of ageing error}
\defType{string}
\defDefault{No Default}

\subsubsection[Data]{\commandlabsubarg{ageing\_\_error}{type}{data}}

\subsubsection[None]{\commandlabsubarg{ageing\_\_error}{type}{none}}

\subsubsection[Normal]{\commandlabsubarg{ageing\_\_error}{type}{normal}}

\defSub{cv} {CV of the misclassification matrix}
\defType{constant}
\defDefault{No Default}
\defLowerBound{0.0 (inclusive)}

\defSub{k} {k defines the minimum age of individuals which can be misclassified, e.g., individuals of age less than k have no ageing error}
\defType{non-negative integer}
\defDefault{0u}

\subsubsection[Off By One]{\commandlabsubarg{ageing\_\_error}{type}{off\_by\_one}}

\defSub{p1} {proportion misclassified as one year younger, e.g., the proportion of age 3 individuals that were misclassified as age 2}
\defType{constant}
\defDefault{No Default}
\defLowerBound{0.0 (inclusive)}
\defUpperBound{1.0 (inclusive)}

\defSub{p2} {proportion misclassified as one year older, e.g., the proportion of age 3 individuals that were misclassified as age 4}
\defType{constant}
\defDefault{No Default}
\defLowerBound{0.0 (inclusive)}
\defUpperBound{1.0 (inclusive)}

\defSub{k} {The minimum age of fish which can be misclassified, i.e., fish of age less than k are assumed to be correctly classified}
\defType{non-negative integer}
\defDefault{0u}
\defLowerBound{0.0 (inclusive)}
\defUpperBound{1.0 (inclusive)}



\section{Report command and subcommand syntax\label{sec:report-syntax}}
\subsection{\I{Report commands and subcommands}}

\section{\I{The report section}\label{sec:report-section}}\index{Reports}\index{Reports section}
The report section specifies the printouts and other outputs from the model. \CNAME\ does not, in general, produce any output unless requested by a valid \command{report} block. 

Reports from \CNAME\ can be defined to print partition and states objects at a particular point in time, observation summaries, estimated parameters and objective function values. See below for a more extensive list, and an example of an observation report.

{\small{\begin{verbatim}
		@report observation_age ## label of report
		type observation		## Type of report
		observation age_1990	## label corresponding to an @observation as shown below
		
		@observation age_1990
		type proportion_at_age
		year 1990
		plus_group
		etc ...
		\end{verbatim}}}

Reports from \CNAME\ all conform to a standard style\index{Reports ! standard style} (with one exception --- the \texttt{output\_parameters} report, see below). The standard style is that reports are prefixed with an aster-ix followed by a user-defined label and type of report in brackets (e.g., \texttt{*label (type)}), with the report ending with the line \texttt{*end}. For example,

\begin{verbatim} 
*My_report(type)
...
*end
\end{verbatim}


This syntax should make it easier for external packages to be configured to read \CNAME\ output. The \texttt{extract} functions in the \R\ \texttt{CASAL2} package uses this information to identify and read \CNAME\ output within an \R\ environment.

Note that the \texttt{output\_parameters} report does not print either a header or \texttt{*end} at the end of the report. This is as the \texttt{output\_parameters} report is designed to provide a single line (or multi-line for more than one set) vector of the estimated parameter values, suitable for reading by \CNAME\ (with the command \texttt{casal2 -i}). This is a specialised report for \texttt{casal2 -o filename} command. For estimate values in standard output users are recommended to use \texttt{type=estimate\_value}.

Reports can be defined in an \command{report} but may not be generated. For example printing the partition for a year and/or time-step that does not exist or reporting the covariance matrix when not estimating. Certain reports are associated with certain \CNAME\ run modes. Such reports are ignored by \CNAME\ and the program will not generate any output for these reports --- although they must still conform to \CNAME s syntax requirements.

Not all reports will be generated in all run modes. Some reports are only available in some run modes. For example, when simulating, only simulation reports will be output.

\subsection{\I{Print the partition at the end of an initialisation}}\index{Reports ! Initialisation}

Print the partition following an initialisation phase. This prints out, the numbers of individuals in each age class and category in the partition following an initialisation phase. This report will print out in the following runmodes \texttt{-r, -e, -f}.

\subsection{\I{Print the partition}}\index{Reports ! Partition}

Print the partition for a given year or given years and time-step. This prints out, the numbers of individuals in each age class and category in the partition for each year. Note that this report is evaluated at the end of the time-step in the given year(s). This report will print out in the following runmodes \texttt{-r, -e, -f}.


\subsection{\I{Print the age length and length weight values}}\index{Reports ! Partition}

Print the length and weight for an age of the partition for a given year or given years and time-step. This prints out, the length and weight value for each age class and category in the partition for each year and time step. Note that this report is evaluated at the end of the time-step in the given year(s). This report will print out in the following runmodes \texttt{-r, -e, -f}.

{\small{\begin{verbatim}
		@report length_weight_at_age
		type partition_mean_weight
		time_step step2
		years 1900:2013
		\end{verbatim}}}

\subsection{\I{Print a process summary}}\index{Reports ! Processes}
Print a summary of a process. Depending on the process, different summaries are produced. These typically detail the type of process, its parameters and other options, and any associated details. This report will print out in the following runmodes \texttt{-r, -e, -f}.

\subsection{\I{Print derived quantities}}\index{Reports ! Derived quantities}

Print out the description of the derived quantity, and the values of the derived quantity as recorded in the model state, for each year of the model. and for all years in the  initialisation phases. This report will print out in the following runmodes \texttt{-r, -e, -f}.

\subsection{\I{Print the estimated parameters}}\index{Reports ! Estimated parameters}

Print a summary of the estimated parameters using the following type \texttt{estimate\_summary}, including the parameter name, lower and upper bounds, the label of the prior, and its value. This report will print out in the following runmodes \texttt{-r, -e}.

\subsection{\I{Print the estimated parameters in a vector format}}\index{Reports ! Estimated parameters}

Print the estimated parameter values out as a vector. The \texttt{estimate\_values} report prints the name of the parameter, followed by the value of that run.  This report will print out in the following runmodes \texttt{-r, -e}.

\subsection{\I{Print the objective function}}\index{Reports ! Objective function}

Print the total objective function value, and the value of all observations, the values of all priors, and the value of any penalties that have been incurred in the model. Note that if an individual model run does not incur a penalty, then the penalty will not be reported. This report will print out in the following runmodes \texttt{-r, -e, -f}.

\subsection{\I{Print the covariance matrix}}\index{Reports ! Covariance Matrix}\index{Reports ! Hessian}

Print the Hessian and covariance matrices if estimating and if the covariance has been requested by\commandlabsubarg{minimiser}{covariance}{true}.

\subsection{\I{Print observations, fits, and residuals}}\index{Reports ! Observations}

Prints out for each category or combination of categories, expected values as calculated by the model, residuals (observed $-$ expected), the error value, process error, and the total error (i.e., the error value as modified by any additional process error), and the contribution to the total objective function of that individual point in the observation. 

Note that constants in likelihoods are often ignored in the objective function score of individual points. Hence, the total score from an observation equals the contribution of the objective function scores from each individual point plus a constant term (if applicable). In likelihoods without a constant term, then the total score from an observation will equal the contribution of the objective function scores from each individual point.

If simulating, then the contribution to the objective function of each observation is reported as zero. 

{\small{\begin{verbatim}
		@report Tan_at_age_obs
		type observation
		observation TAN_AT_AGE
		\end{verbatim}}}

\subsection{\I{Print simulated observations}}\index{Reports ! Simulated observations}

Prints out a complete observation definition (i.e., in the form defined by \commandlabsubarg{report}{type}{observation}), but with observed values replaced by randomly generated simulated values. The output is in a form  suitable for use within a \CNAME\ \config, reproducing the command and subcommands from the \config. This report will print out in the following runmodes \texttt{-s}.

\subsection{\I{Print the ageing error misclassification matrix}}\index{Reports ! Ageing error misclassification matrix}\label{sec:ageingerrorreport}

Prints out the ageing error misclassification matrix used to offset observations within during model the model fitting procedure.

\subsection{\I{Print selectivities}}\index{Reports ! Selectivities}

Prints the values of a selectivity for each age in the partition, for a given year and at then end of a given time-step.

\subsection{\I{Print the random number seed}}\index{Reports ! Random number seed}

Prints the random number seed used by \CNAME\ to generate the random number sequence. Future runs made with the same random number seed and the same model will produce identical outputs.

\subsection{\I{Print the results of an MCMC}}\index{Reports ! MCMC}

Print the MCMC samples, objective function values, and proposal covariance matrix following an MCMC. This report will print out in the following runmode \texttt{-m}.

\subsection{\I{Print the MCMC samples as they are calculated}}\index{Reports ! MCMC samples}

Print the MCMC samples for each new \textit{i}th sample as they are calculated while doing an MCMC. The output file will be updated with each new sample as it is calculated by \CNAME. This report will print out in the following runmodes \texttt{-m}.

\subsection{\I{Print the MCMC objective function values as they are calculated}}\index{Reports ! MCMC objective functions}

Print the MCMC objective function values (along with the proposal covariance matrix) for each new \textit{i}th sample as they are calculated while doing an MCMC. The output file will be updated with each new set of objective function values as it is calculated by \CNAME. This report will print out in the following runmodes \texttt{-m}.

\subsection{\I{Print time varying parameters}}\index{Reports ! time varying}

Print all \command{time\_varying} blocks with the values and years that they were implemented in. This report will print out in the following runmodes \texttt{-r, -e, -m}.

{\small{\begin{verbatim}
		@report time_varying_parameters
		type time_varying
		\end{verbatim}}}

\subsection{\I{Tabular or derived quantity reporting}}\index{Reports ! Tabular}
An alternative reporting framework to the standard output is the tabular reporting. Tabular reporting is used with multiline \texttt{-i} input files (like the MCMC sample or -o outputs). Tabular reports will print out a row that will correspond with each row of the \texttt{-i} input files. Tabular reporting is in invoked at the command line using the following command \texttt{casal2 -r --tabular -i file\_name}.

\command{report}'s that have \texttt{tabular} functionality, these are derived quantities, observations, selectivities, processes and estimate\_values. For each input file the output will begin with the names of each column followed by a multiline report ending with the \texttt{*end} syntax. These tables can be easily read into \R\ using the \texttt{CASAL2} package and for the example of MCMC multi-line files posteriors of derived quantities can be plotted. This functionality is analagous to running CASAL with the \texttt{-v} mode. The main use is extracting SSB's, exploitation by year, residuals from a MCMC run.




\section{Other commands and subcommands\label{sec:general-syntax}}

\defComArg{include}{file}{\I{Include an external file}}

\defArg{file}{The name of the external file to include}
\defType{string}
\defDefault{No default}
\defValue{A valid external file}
\defCondition{The file name must be enclosed in double quotes}
\defExample{\command{include} \argument{\ "my\_file.txt"}}
\defNote{\command{include} does not denote the end of the previous command block as is the case for all other commands}
