\section{Running \CNAME\label{sec:running}\index{Running \CNAME}}

\CNAME\ is run from the console window (i.e., the command line) on \I{Microsoft Windows} or from a terminal window on \I{Linux}. \CNAME\ gets its information from input data files, the key one of which is the \config\index{Input configuration file}. 

The \config\ is compulsory and defines the model structure, processes, observations, parameters (both the fixed parameters and the parameters to be estimated)\index{Estimable parameters}, and the reports (outputs) requested. The following sections  describe how to construct the \CNAME\ configuration file. By convention, the name of the \config\ ends with the suffix \texttt{.csl2}, however, any file name is acceptable. Note that the \config\ can `include' other files as a part of its syntax. Collectively, these are called the \config.

Other input files can, in some circumstances, be supplied, depending on what is required. For example, a file can be supplied that defines the starting point for estimation, as points from which to simulate observations, or as points from which to run projections.

Simple command line arguments\index{Command line arguments} are used to determine the actions or \emph{tasks}\index{Tasks} of \CNAME, i.e., to run a model with a set of parameter values, estimate parameter values (either point estimates or MCMC), project quantities into the future, simulate observations, etc,. Hence, the \emph{command line arguments} define the \emph{task}. For example, \texttt{-r} is the \emph{run}, \texttt{-e} is the \emph{estimation}, and \texttt{-m} is the \emph{MCMC} task. The \emph{command line arguments} are described in Section \ref{sec:command-line-arguments}.

\subsection{\I{Using \CNAME}}

To use \CNAME, open a console (i.e. the command prompt) window (Microsoft Windows) or a terminal window (Linux). Navigate to a directory of your choice, where your \config s are located. Then type \cname\ with any arguments (see Section \ref{sec:command-line-arguments} for the the list of possible arguments). \CNAME\ will print output to the screen and return you to the command prompt when it completes its task. Note that the \CNAME\ executable (binary) and shared libraries (extension \texttt{.dll}) must be either in the directory where you run it or in your systems \texttt{PATH}. The \CNAME\ installer should update your path on Windows in any case, but see your operating system documentation for help on identifying or modifying your \texttt{PATH}.

\subsection{The \config\label{sec:config-files}}\index{Input configuration file}

The \config\ is made up of four broad sections; the description of the population structure and parameters (the population section), the estimation methods and variables (the estimation section), the observations and their associated likelihoods (the observation section), and the outputs and reports that \CNAME\ will return (the report section). The \config\ is made up of a number of commands (many with subcommands) which specify various options for each of these components.

The command and subcommand definitions in the \config\ can be extensive (especially when you have a model that has many observations), and can result in a \config\ that is long and difficult to navigate. To aid readability and flexibility, we can use the \config\ command !\texttt{\emph{include}} \texttt{\emph{file}}. The command causes an external file, \argument{\emph{file}}, to be read and processed, exactly as if its contents had been inserted in the main \config\ at that point\index{Including external files}. The file name must be a complete file name with extension, but can use either a relative or absolute path as part of its name. Note that included files can also contain !\texttt{\emph{include}} commands. See Section \ref{sec:general-syntax} for more detail.

\subsection{\I{Redirecting standard output}\label{sec:redirecting-stdout}}

\CNAME\ uses the standard output stream \texttt{standard output}\index{standard output} to display run-time information. The \I{standard error} stream is used by \CNAME\ to output the program exit status and run-time errors. We suggest redirecting both the standard output and standard error into files\index{Redirecting standard out}\index{Redirecting standard error}. With the bash shell (on Linux systems), you can do this using the command structure,

\begin{verbatim} (casal2 [arguments] > out) >& err &\end{verbatim}

It may be useful to redirect the standard input, especially if you're using \CNAME\ inside a batch job software, i.e. 

\begin{verbatim} (casal2 [arguments] > out < /dev/null) >& err &\end{verbatim}

On Microsoft Windows systems, you can redirect to standard output using,

\begin{verbatim} casal2 [arguments] > out\end{verbatim}

And, on some Microsoft Windows systems (e.g., Windows10), you can redirect to both standard output and standard error, using the syntax, 

\begin{verbatim} casal2 [arguments] > out 2> err\end{verbatim}

Note that \CNAME\ outputs a few lines of header information to the output. The header\index{Output header information} consists of the program name and version, the arguments passed to \CNAME\ from the command line, the date and time that the program was called (derived from the system time), the user name, and the machine name (including the operating system and the process identification number). These can be used to track outputs as well as identifying the version of \CNAME\ used to run the model.

\subsection{\I{Command line arguments}\label{sec:command-line-arguments}}

The call to \CNAME\ is of the following form: 

\texttt{\cname [-c \emph{config\_file}] [\emph{task}] [\emph{options}]}

\begin{description}
  \item [\texttt{-c \emph{config\_file}}] Define the \config\ for \CNAME. If omitted, then \CNAME\ looks for a file named \texttt{config.csl2}.
\end{description}

and where \emph{task} is one of, if there are square brackets \textbf{[]} this indicates a secondary label to call the same task for example \textbf{\texttt{-h}} will execute the same task as \textbf{\texttt{--help}};

\begin{description}
\item [\texttt{-h [--help]}] Display help (this page).
\item [\texttt{-l [--licence]}] Display the reference for the software license (GPL v2).
\item [\texttt{-v [--version]}] Display the \CNAME\ version number.

\item [\texttt{-r [--run]}] \emph{Run} the model once using the parameter values in the \config, or optionally, with the values from the file denoted with the command line argument \texttt{-i \emph{file}}.

\item [\texttt{-e [--estimate]}] Do a point \emph{estimate} using the values in the \config\ as the starting point for the parameters to be estimated, or optionally, with the start values from the file denoted with the command line argument \texttt{-i \emph{file}}.

\item [\texttt{-p [--profiling]}] Do a likelihood \emph{profile} using the parameter values in the \config\ as the starting point, or optionally, with the start values from the file denoted with the command line argument \texttt{-i \emph{file}}.

\item [\texttt{-m [--mcmc]}] Do an \emph{MCMC} estimate using the values in the \config\ as the starting point for the parameters to be estimated, or optionally, with the start values from the file denoted with the command line argument \texttt{-i \emph{file}}. 

\item [\texttt{-f [--projection]}] Project the model \emph{forward} in time using the parameter values in the \config\ as the starting point for the estimation, or optionally, with the start values from the file denoted with the command line argument \texttt{-i \emph{file}}. 

\item [\texttt{-s [--simulation]} \emph{number}] \emph{Simulate} the \emph{number} of observation sets using values in the \config\ as the parameter values, or optionally, with the values for the parameters denoted as estimated from the file with the command line argument \texttt{-i \emph{file}}.
\end{description}

In addition, the following are optional arguments\index{Optional command line arguments} [\emph{options}],

\begin{description}
\item [\texttt{-i [--input] \emph{file}}] \emph{Input} one or more sets of free (estimated) parameter values from \texttt{\emph{file}}. See Section \ref{sec:report-syntax} for details about the format of \texttt{\emph{file}}.

\item [\texttt{-o [--output]\emph{file}}] \emph{Output} a report of the free (estimated) parameter values in a format suitable for \texttt{-i \emph{file}}. See Section \ref{sec:report-syntax} for details about the format of \texttt{\emph{file}}.

\item [\texttt{-g [--seed]\emph{seed}}]  Seed the random number \emph{generator} with \texttt{\emph{seed}}, a positive (long) integer value. Note, if \texttt{-g} is not specified, then \CNAME\ will  generate a random number seed based on the computer clock time.

\item [\texttt{--loglevel}] arg = \{trace, finest, fine, medium\} See Section \ref{sec:report-section}.

\item [\texttt{--tabular}] Run with \texttt{-r} or \texttt{-f}  command it will print \command{report} in tabular format. See Section \ref{sec:report-section}.

\item [\texttt{--single-step}] Run with \texttt{-r}, this additional option will pause the model and ask the user to specify parameters and their values to use for the next iteration. See Section \ref{sec:singlestepping}.

\item [\texttt{-q [--query]}] Query an object type to see it's description and parameter definitions. This will print out an extract of the object. An object can be defined as block.type. For example \texttt{casal2 --query process.recruitment\_constant}, will query the constant recruitment block.

\end{description}

\subsection{Constructing a \CNAME\ \config s \label{constructing-config}}\index{Input configuration file syntax}

The model definition, parameters, observations, and reports are specified in an \config s. The  population section is described in Section \ref{sec:population-section} and the population commands in Section \ref{sec:population-syntax}. Similarly, the estimation section is described in Section \ref{sec:estimation-section} and its commands in Section \ref{sec:estimation-syntax}, and in Section \ref{sec:report-section} and Section \ref{sec:report-syntax} for the report and report commands. 

\subsubsection{Commands}\index{Commands}

\CNAME\ has a range of commands that define the model structure, processes, observations, and how tasks are carried out. There are three types of commands, 

\begin{enumerate}
\item Commands that have an argument and do not have subcommands (for example, !\texttt{\emph{include}}\ \argument{\emph{file}})
\item Commands that have a label and subcommands (for example \command{process} must have a label, and has subcommands)
\item Commands that do not have either a label or argument, but have subcommands (for example \command{model})
\end{enumerate}

Commands that have a label must have a unique label, i.e., the label cannot be used on more than one command of that type. The labels can contain alpha numeric characters, period (`.'), underscore (`\_') and dash (`-'). Labels must not contain white-space, or other characters that are not letters, numbers, dash, period or an underscore. For example,

{\small{\begin{verbatim}
@process NaturalMortality
or
!include MyModelSpecification.csl2
		\end{verbatim}}}

\subsubsection{Subcommands}\index{Commands ! Subcommands}

Subcommands in \CNAME\ are for defining options and parameter values for commands. They always take an argument which is one of a specific \emph{type}. The types acceptable for each subcommand are defined in Section \ref{sec:syntax}, and are summarised below. 

Like commands (\command{command}), subcommands and their arguments are not order specific --- except that that all subcommands of a given command must appear before the next \command{command} block. \CNAME\ may report an error if they are not supplied in this way, however, in some circumstances a different order may result in a valid, but unintended set of actions, leading to possible errors in your expected results.  

The arguments for a subcommand are either\index{Subcommand argument type}:

\begin{tabular}{ll}
\textbf{switch} & true/false\\ 
\textbf{integer}& an integer number,\\
\textbf{integer vector} & a vector of integer numbers,\\
\textbf{integer range} & a range of integer numbers separated by a colon (:), e.g. 1994:1996 is \\ & expanded to an integer vector of values 1994 1995 1996),\\
\textbf{constant} & a real number (i.e. double),\\
\textbf{constant vector} & a vector of real numbers (i.e. vector of doubles),\\
\textbf{estimable} & a real number that can be estimated (i.e. estimable double),\\
\textbf{estimable vector} & a vector of real numbers that can be estimated (i.e. vector of estimable \\ & doubles),\\
\textbf{string} & a categorical (string) value, or\\
\textbf{string vector} & a vector of categorical values.
\end{tabular}

Switches are parameters which are either true or false. Enter \emph{true} as \argument{true} or \argument{t}, and \emph{false} as \argument{false} or \argument{f}. 

Integers must be entered as integers (i.e., if \subcommand{year}\ is an integer then use 2008, not 2008.0)

Arguments of type integer vector, integer range, constant vector, estimable vector, or categorical vector contain one or more entries on a row, separated by white space (tabs or spaces). 

\emph{Estimable} parameters are those parameters that \CNAME\ can estimate, if requested. If a particular parameter is not being estimated in a particular model run, then it acts as a constant.  Within \CNAME\, only estimable parameters can be estimated. And, you have to tell \CNAME\ those that are to be estimated in any particular model. Estimable parameters that are being estimated within a particular model run are called the \emph{estimated parameters}\index{Estimated parameters}.

\subsubsection{The command-block format}\index{Command block format}
Each command-block either consists of a single command (starting with the symbol \command{}) and, for most commands, a unique label or an argument. Each command is then followed by its subcommands and their arguments, e.g., 

\begin{description}
\item \command{command}, or 
\item \command{command} \subcommand{argument}, or
\item \command{command} \subcommand{\emph{label}}
\end{description}

and then
\begin{description}
\item \subcommand{subcommand} \subcommand{argument}
\item \subcommand{subcommand} \subcommand{argument}
\item etc,.
\end{description}

Blank lines are ignored, as is extra white space (i.e., tabs and spaces) between arguments. But don't put extra white space before a \command{} character (which must also be the first character on the line), and make sure the file ends with a carriage return.

There is no need to mark the end of a command block. This is automatically recognized by either the end of the file, section, or the start of the next command block (which is marked by the \command{} on the first character of a line). Note, however, that the !\texttt{\emph{include}} is the only exception to this rule. See Section \ref{sec:general-syntax})\index{Command ! Include files} for details of the use of !\texttt{\emph{include}}. 

Note that in the \config, commands, sub-commands, and arguments are not case sensitive. However, labels and variable values are case sensitive. Also note that if you are on a Linux system then external calls to files are case sensitive (i.e., when using !\texttt{\emph{include}} \subcommand{\emph{file}}, the argument \subcommand{\emph{file}} will be case sensitive). 


\subsubsection{\I{Commenting out lines}}\index{Comments}
Text that follows a \commentline\ on a line are considered to be comments and are ignored. If you want to remove a group of commands or subcommands using \commentline, then comment out all lines in the block, not just the first line. 

Alternatively, you can comment out an entire block or section by placing curly brackets around the text that you want to comment out. Put in a \commentstart\ as the first character on the line to start the comment block, then end it with \commentend. All lines (including line breaks) between \commentstart\ and \commentend\ inclusive are ignored. 
{\small{\begin{verbatim}
		# This is a comment and will be ignored
		@process NaturalMortality
		m 0.2
		{ 
		This block of code 
		is a comment and
		will be ignored
		}
		\end{verbatim}}}

\subsubsection{Determining parameter names\label{sec:parameter-names}\index{Determining parameter names}\index{Parameter names}}

When \CNAME\ processes a \config, it translates each command and each subcommand into a parameter with a unique name. For commands, this parameter name is simply the command label. For subcommands, the parameter name format is either 

\begin{description}
\item \texttt{command[label].subcommand} if the command has a label, or
\item \texttt{command.subcommand} if the command has no label, or
\item \texttt{command[label].subcommand(i)} if the command has a label and the subcommand arguments are a vector, and we are accessing the  \emph{i}th element of that vector. 
\item \texttt{command[label].subcommand(i:j)} if the command has a label, and the subcommand arguments are a vector, and we are accessing the elements from $i$ to $j$ (inclusive) of that vector.
\end{description} 

The unique parameter name is used to reference the parameter when estimating, applying a penalty, projecting, time varying or applying a profile. For example, the parameter name of subcommand \subcommand{m} of the command \command{process} with the label \argument{NaturalMortality} is

\texttt{process[NaturalMortality].m}

\subsection{\I{Single stepping \CNAME}\label{sec:singlestepping}}\index{Single\_stepping}\index{single\_stepping section}

Single stepping in \CNAME\ gives it the ability to write reports and `pause' after each year in the annual cycle during a run, and then wait and process user input of updated estimable parameters for the next year. 

This can allow \CNAME\ to be used for implementing models that require feedback management simulations or scenarios, for example for use in operational management procedures (OMPs). This can be automated using \R, where \CNAME\ may be controlled by \R\ to update input harvest values (for examples, catches in a fisheries model) to evaluate a particular harvest control rule. 

\subsection{\CNAME\ exit status values\index{Exit status value}}
When \CNAME\ completes its task successfully or errors out gracefully, it returns a single exit status value 'completed' to the standard output. Error messages will be printed to the console. If configuration errors are found, \CNAME\ will print an error messages along with the associated files and line numbers where the errors were identified.
