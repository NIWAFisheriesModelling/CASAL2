\section{The observation section\label{sec:observation-section}}

\subsection{\I{Observations}\label{sec:Observations}\index{Observations}}

The objective function is based on the goodness-of-fit of the model to your observations. Observations are typically supplied at an instance in time, over a group of aggregated categories. \hl{Most observations are different kinds of time series} \textbf{Re-Word}, i.e., data which were recorded for one or more years, in the same format each year. Examples of time series data types include relative abundance indices, commercial catch length frequencies, and survey numbers-at-age.

The definitions for each type of observation are described below, including how the observed values should be formatted, how \CNAME\ calculates the expected values, and the likelihoods that are available for each type of observation.

There are two main types of observations available in \CNAME. The first are observations that are associated with a \hyperref[sec:mortality_block]{\textbf{mortality block}} and secondly observations that are associated with a specific process. These can be distinguished by the \subcommand{type} subcommand. If an observation type begins with \texttt{process} it is an observation that is associated with a process. If a type does \textbf{not} begin with \texttt{process} it is associated with the mortality block of the time step that you define. For example the observation type \texttt{process\_abundance} is a process based observation vs \texttt{process\_abundance} \texttt{abundance}, which is an observation that is associated with a mortality block.

Process specific observations can also be broken into two types. \textbf{Specific process observations} are observations that are associated to a specific process (e.g. \texttt{process\_proportions\_migrating}), and \textbf{general process observations} are observations that can be associated with any process (e.g. \texttt{process\_proportions\_at\_age}). These tiers of observations have been separated in different sections as to reduce the confusion.

\subsubsection{\I{Mortality block associated observations}}

All observations within this class are calculated in a similar fashion. That is an expectation is calculated at the beginning of the mortality block and at the end of the mortality block. \CNAME\ then uses a linear interpolation to approximate an expectation part way through a mortality block usign the subcommand \texttt{time\_step\_proportion}. This could be useful if a survey occurs part-way through an exploitation phase. For example for if modelling a fish population this may be part-way through a fishing season. Each observation in this class will evaluate different expectations of the partition which will be explained in the following descriptions. A list of observation \texttt{types} that are available with this class of observations are as follows,
\begin{itemize}
	\item \texttt{abundance}
	\item \texttt{biomass}
	\item \texttt{proportions\_at\_age}
	\item \texttt{proportions\_at\_length}	
	\item \texttt{proportions\_by\_category}	
	\item \texttt{tag\_recapture\_by\_length}		
	\item \texttt{tag\_recapture\_by\_age}														
\end{itemize}

\paragraph*{\I{Abundance or biomass observations}}
Abundance (or biomass) observations are observations of either a relative or absolute number (or biomass) of individuals from a set of categories after applying a selectivity. The observations classes are the same, except that a biomass observation will use the biomass as the observed (and expected) value (calculated from mean weight of individuals within each age and category) while an abundance observation is just the number of individuals. 

Each observation is for a given year and time-step, for some selected age classes of the population (i.e., for a range of ages multiplied by a selectivity), for aggregated categories. Further, you need to provide the label of the catchability coefficient $q$, which can either be estimated of fixed. For absolute abundance or absolute biomass observations, define a catchability where $q=1$.

The observations can be supplied for any set of categories. For example, for a model with the two categories \emph{male} and \emph{female}, we might supply an observation of the total abundance/biomass (male $+$ female) or just male abundance/biomass. The subcommand \subcommand{categories} defines the categories used to aggregate the abundance/biomass. In addition, each category must have an associated selectivity, defined by \subcommand{selectivities}. For example,  

{\small{\begin{verbatim}
		categories male
		selectivities male-selectivity
		\end{verbatim}}}

defines an observation for males after applying the selectivity male-selectivity. \CNAME\ then expects that there will be a single observation supplied. The expected values for the observations will be the expected abundance (or biomass) of males, after applying the selectivities, at the year and time-step specified. 

\CNAME\ calculates the expected values by summing over the defined ages (via the age range and selectivity) and categories at both the beginning and end of a mortality block. You can prompt \CNAME\ to approximate the expectation part way through the mortality block using the \texttt{time\_step\_proportion}. The default value \CNAME\ uses us 0.5, which does linear interpolation between the start and end abundance (or biomass) from the mortality block.

For an abundance observation the expectation is calculated as follows,
\begin{equation}\label{eq:expec_1}
E_{i,1} = \sum_{c=1}^{} \sum_{a=1}^{A} S_{a,c} N_{a,c,i,1}
\end{equation}

\begin{equation}\label{eq:expec_2}
E_{i,2} = \sum_{c=1}^{} \sum_{a=1}^{A} S_a N_{a,c,i,2}
\end{equation}

Where $E_{i,1}$ is the expectation at the beginning of time step and $E_{i,2}$ is the expectation at the end of the time-step. $S_a$ is the selectivity for age $a$ and category $c$. If there is no mortality related to this observation then $E_i$ which is used in the likelihood contribution is $E_{i,1}$. If this was a biomass observation we would replace $N_{a,c,i,1}$ in Equation~\eqref{eq:expec_1} and~\eqref{eq:expec_2} with $N_{a,c,i,1} \bar{w}_{a,c}$, where $\bar{w}_{a,c}$ is the mean weight of category $c$ at age $a$. If the user wishes to apply 100\% mortality then $E_i = E_{i,2}$. For applying quantities of mortality between these values ($M_i$), \CNAME\ does the following linear interpolation.
\begin{equation}
E_{i} = |E_{i,1} - E_{i,2}|  M_i
\end{equation}


{\small{\begin{verbatim}
		@observation MyAbundance
		type abundance
		years 1999
		...
		categories male 
		obs 1000
		...
		\end{verbatim}}}

Or, for an observation aggregated over multiple categories,

{\small{\begin{verbatim}
		@observation MyAbundance
		type abundance
		years 1990 1991
		...
		categories male+female
		table obs
		1990 1000
		1991 1200
		end_table
		...
		\end{verbatim}}}


Note that, to define a biomass observation instead of an abundance observation, use 

{\small{\begin{verbatim}
		@observation MyBiomass
		type biomass
		...
		\end{verbatim}}}

\paragraph*{\I{Proportions-at-age}}
Proportions-at-age observations are observations of the relative number of individuals at age, via some selectivity. 

The observation is supplied for a given year and time-step, for some selected age classes of the population (i.e., for a range of ages multiplied by a selectivity), for categories aggregated over a set of spatial cells. Note that the categories defined in the observations must have an associated selectivity, defined by \subcommand{selectivities}.

The age range must be ages defined in the partition (i.e., between \commandsub{model}{min\_age} and \commandsub{model}{max\_age} inclusive), but the upper end of the age range can optionally be a plus group --- which must be either the same or less than the plus group defined for the partition. 


Proportions-at-age observations can be supplied as; 
\begin{enumerate}
	\item a set of proportions for a single category, 
	\item a set of proportions for multiple categories, or
	\item a set of proportions across aggregated categories.
\end{enumerate}

The method of evaluating expectation are the same for all three of these sceneries. We will describe how you define these different scenarios and the expected dimensions of observation and error inputs that \CNAME\ expects for each respective scenario with examples.

Like all types of observations that are associated with the mortality block, \CNAME\ will evaluate the numbers at age before the mortality block (after taking into account a selectivity that the user defines) and after for the specified time step of the observation. \CNAME\ will generate expectations from the partition part way through the mortality block using the subcommand \texttt{time\_step\_proportion}. This approximation is an linear interpolation of the numbers at age over the mortality block. 

Once the interpolation is evaluated \CNAME\ will apply ageing error if the user has specified it. \CNAME\ finally converts numbers at age to proportions at age by dividing all numbers at age bin by the total and sending that to the likelihood to be evaluated.

Defining an observation for a single category is the simplest, and is used to model a set of proportions of a single category by age class. For example, to specify that the observations are of the proportions of male within each age class, then the subcommand \subcommand{categories} for the \commandlabsubarg{observation}{type}{proportion\_by\_age} command is,

{\small{\begin{verbatim}
		categories male
		\end{verbatim}}}

\CNAME\ then expects that there will be a single vector of proportions supplied, with one proportion for each age class within the defined age range, and that these proportions sum to one. 

For example, if the age range was 3 to 10, then 8 proportions should be supplied (one proportion for each of the the ages 3, 4, 5, 6, 7, 8, 9, and 10). The expected values will be the expected proportions of males within each of these age classes (after ignoring any males aged less than 3 or older than 10), after applying a selectivity at the year and time-step specified. The supplied vector of proportions (i.e., in this example, the 8 proportions) must sum to one, which is evaluated with a default tolerance of 0.001. 


{\small{\begin{verbatim}
		@observation MyProportions
		type proportions_at_age
		...
		categories male
		min_age 3
		max_age 9
		years 1990
		table obs
		1990 0.01 0.09 0.20 0.20 0.35 0.10 0.05
		end_table
		...
		\end{verbatim}}}


Defining an observation for multiple categories extends on the single category implementation. It is used to model a set of proportions over several categories by age class. For example, to specify that the observations are of the proportions of male or females within each age class, then the subcommand \subcommand{categories} for the \commandlabsubarg{observation}{type}{proportion\_by\_age} command is,

{\small{\begin{verbatim}
		categories male female
		\end{verbatim}}}

\CNAME\ then expects that there will be a single vector of proportions supplied, with one proportion for each category and age class combination, and that these proportions sum to one across all ages and categories.

For example, if there were two categories and the age range was 3 to 10, then 16 proportions should be supplied (one proportion for each of the the ages 3, 4, 5, 6, 7, 8, 9, and 10, for each category male and female). The expected values will be the expected proportions of males and within each of these age classes (after ignoring those aged less than 3 or older than 10), after applying a selectivity at the year and time-step specified. The supplied vector of proportions (i.e., in this example, the 16 proportions) must sum to one, which is evaluated with a default tolerance of 0.001. 

For example,

{\small{\begin{verbatim}
		@observation MyProportions
		type proportions_at_age
		...
		categories male female
		min_age 1
		max_age 5
		years 1990 1991
		table obs
		1990 0.01 0.05 0.10 0.20 0.20 0.01 0.05 0.15 0.20 0.03
		1991 0.02 0.06 0.10 0.21 0.18 0.02 0.03 0.17 0.20 0.01
		end_table
		...
		\end{verbatim}
		
		
Defining an observation across aggregated categories allows categories to be aggregated before the proportions are calculated. It is used to model a set of proportions from several categories that have been combined by age class. To indicate that two (or more) categories are to be aggregated, separate them with a '+' symbol. For example, to specify that the observations are of the proportions of male and females combined within each age class, then the subcommand \subcommand{categories} for the \commandlabsubarg{observation}{type}{proportion\_by\_age} command is,
		
		{\small{\begin{verbatim}
				categories male + female
				\end{verbatim}}}
		
\CNAME\ then expects that there will be a single vector of proportions supplied, with one proportion for each age class, and that these proportions sum to one. 
		
For example, if there were two categories and the age range was 3 to 10, then 8 proportions should be supplied (one proportion for each of the the ages 3, 4, 5, 6, 7, 8, 9, and 10, for the sum of males and females within each age class). The expected values will be the expected proportions of males + females within each of these age classes (after ignoring those aged less than 3 or older than 10), after applying a selectivity at the year and time-step specified. The supplied vector of proportions (i.e., in this example, the 16 proportions) must sum to one, which is evaluated with a default tolerance of 0.001. 
		
For example,
		
{\small{\begin{verbatim}
				@observation MyProportions
				type proportions_at_age 
				...
				years 1990 1991
				categories male+female
				min_age 1
				max_age 5
				table obs
				1990 0.02 0.13 0.25 0.30 0.30
				1991 0.02 0.06 0.18 0.35 0.39
				end_table
				...
				\end{verbatim}
				
The later form can then be extended to include multiple categories, or multiple aggregated categories. For example, to describe proportions for the three groups: immature males, mature males, and all females (immature and mature females added together) for ages 1--4, a total of 12 proportions are required 
				
{\small{\begin{verbatim}
@observation MyProportions
type proportions_at_age
...
categories male_immature male_mature female_immature+female_mature
min_age 1
max_age 4
years 1990
table obs
year 1990 0.05 0.15 0.15 0.05 0.02 0.03 0.08 0.04 0.05 0.15 0.15 0.08
end_table
...
\end{verbatim}}}


\paragraph*{\I{Proportions-at-length}}
Functionality regarding defining combinations of categories and aggregated categories directly translates over from proportions at age to proportions at length. The difference is the observation is over length bins instead of age-classes. \CNAME\ calculates expectations of numbers at length by converting numbers at age to numbers by length by using the age-length relationship and distribution specified for the category specified in the \command{age\_length} block. Commands that are different are instead of supplying a minimum and maximum age users must supply a vector of length bins. If there is no plus group i.e \texttt{length\_plus\_group=false} \CNAME\ expects a vector of proportions for each year that is $n - 1$, where $n$ is the number of lengths supplied. If \texttt{length\_plus\_group=true} \CNAME\ expects a vector of proportions for each year that is $n$. The last proportion represents the numbers from the last length bin to the maximum length the age-length relationship allows.


{\small{\begin{verbatim}
@observation Observed_Length_frequency_Chat_east
type process_removals_by_length
years 1991 1992
likelihood multinomial
time_step Summer
fishery EastChathamRise
process instant_mort
categories male
length_plus_group false
length_bins 0 20 40 60 80 110
table obs
1991    0.2   0.25    0.15     0.2     0.2 
1992    0.12  0.25    0.28    0.25    0.1 
end_table
table error_values
1991 25
1992 37
end_table  
\end{verbatim}}}
 
\paragraph*{\I{Proportions-by-category observations}\label{sec:proportions-by-category}}
Proportions-by-category observations are observations of either the relative number of individuals between categories within age classes, or relative biomass between categories within age classes. 

The observation is supplied for a given year and time-step, for some selected age classes of the population (i.e., for a range of ages multiplied by a selectivity).

The age range must be ages defined in the partition (i.e., between \commandsub{model}{min\_age} and \commandsub{model}{max\_age} inclusive), but the upper end of the age range can optionally be a plus group --- which may or may not be the same as the plus group defined for the partition. 

Proportions-by-category observations can be supplied for any set of categories as a proportion of themselves and any set of additional categories. For example, for a model with the two categories \emph{male} and \emph{female}, we might supply observations of the proportions of males in the population at each age class. The subcommand \subcommand{categories} defines the categories for the numerator in the calculation of the proportion, and the subcommand \subcommand{categories2} supplies the additional categories to be used in the denominator of the calculation. In addition, each category must have an associated selectivity, defined by \subcommand{selectivities} for the numerator categories and \subcommand{selectivities2} for the additional categories used in the denominator, e.g., 

{\small{\begin{verbatim}
		categories male
		categories2 female
		selectivities male-selectivity
		selectivities2 female-selectivity
		\end{verbatim}}}

defines that the proportion of males in each age class as a proportion of males $+$ females. \CNAME\ then expects that there will be a vector of proportions supplied, with one proportion for each age class within the defined age range, i.e., if the age range was 3 to 10, then 8 proportions should be supplied (one proportion for each of the the ages 3, 4, 5, 6, 7, 8, 9, and 10). The expected values will be the expected proportions of male to male $+$ female within each of these age classes, after applying the selectivities at the year and time-step specified. 

The observations must be supplied using all or some of the values defined by a categorical layer. \CNAME\ calculates the expected values by summing over the ages (via the age range and selectivity) and categories for those spatial cells where the categorical layer has the same value as defined for each vector of observations i.e.,

{\small{\begin{verbatim}
		@observation MyProportions
		type proportions_by_category
		years 1990 1991
		...
		categories male
		categories2 female
		min_age 1
		max_age 5
		table obs
		1990 0.01 0.05 0.10 0.20 0.20
		1991 0.02 0.06 0.10 0.21 0.18
		end_table
		...
		\end{verbatim}}}

\paragraph*{\I{Tag Recapture by length}\label{sec:tag-recapture-by-length}}

Tag data is primarily used to estimate the population abundance of fish. In some models, this estimation can only be made outside the model and the result is used as an estimate of abundance in the model. But in CASAL the tagging data can, alternatively, be fitted within the model.
\\\\
Before adding a tag-recapture time series, you will need to define a tag-release process (Section~\ref{sub:tag_release}). Tagging events list the labels of the tags which are modelled, and define the events where fish are tagged (i.e., \CNAME moves fish into the section of the partition corresponding to a specific tag).
\\\\
The observations are divided into two parts: (i) the number of fish that were scanned, and (ii) the number of tags that were recaptured. Each can be specified by categories, or for combinations of categories. The precise content of the scanned and recaptured observations depends on the sampling method, and the available options are:

\begin{enumerate}
	\item age: both scanned and recaptured are vectors containing numbers-at-age. Only available in an age-based model. The selectivity ogive is redundant and cannot be supplied.
	\item size: both scanned and recaptured are vectors containing numbers-at-size. Can be used in either an age- or size-based model. The selectivity ogive is redundant and cannot be supplied.
\end{enumerate}
When defining the tag-recapture time series, you also need to specify:
\begin{itemize}
	\item the time step,
	\item the years (unlike a tag-release process, the tag-recapture observations can occur over several years),
	\item the probability that each scanned tagged fish is detected as tagged (may be less than 1 if the observers are not infallible). The expected number of tags detected is calculated by multiplying this number by the number of tagged fish in the sample,
	\item the tagged category or categories (Make up the recaptures),
	\item the categories scanned (All the fish sampled for tags),
	\item A selectivity used in the recapture process,
	\item the size classes if the observations are size-based in an age-based model.
\end{itemize}


An example of a tag recapture observation applied in \CNAME\ is shown, below
{\small{\begin{verbatim}
		## For the following partition
		@categories
		format sex.area.tag
		names  		male.Area1.2011,notag female.Area1.2011,notag
		
		@observation Tag_2011_Area1_recap_2012 ## individuals tagged in 2011 and recaptured in 2012
		## in Area1
		type tag_recapture_by_length
		categories *.Area1.2011+  ## male and femaled tagged categories
		categories2 format=*.Area1.*+ ## scanned categories in Area1
		detection 0.85 ## detection probability
		likelihood binomial ## likelihood choice
		selectivities One ## label of selectivity for tagged
		selectivities2 One ## label of selectivity for scanned
		years 2012  ## years to apply observation
		time_step step2  ## time_step to apply observation	
		time_step_proportion 0.5 ## proportion of mortality applied before observation is calculated 
		length_bins 21 30 40 50	## size bins
		plus_group true ## is the last bin a plus group i.e. 50cm +
		
		table scanned
		2012 281271 41360 30239 12234
		end_table
		
		table recaptured
		2012 15 20 12 2
		end_table
		
		delta 1e-11 ## robustification value
		dispersion 6.3	## dispersion factor
		
		\end{verbatim}}}



The tag-recapture likelihood (binomial) is specified below as it is a modified version of the more general binomial. Note that this likelihood does not have any user-set precision parameters such as $N$ or $c.v.$ (though there are user-specified robustification and dispersion parameters available). Note that factorials are calculated using the log-gamma function, to allow for non-integer arguments where necessary (and avoid overflow errors).


\subsubsection{\I{General process observations}}
A list of \texttt{types} that are associated with this set of observations.
\begin{itemize}
	\item \texttt{process\_abundance}
	\item \texttt{process\_biomass}
	\item \texttt{process\_proportions\_at\_age}
	\item \texttt{process\_proportions\_at\_length}	
	\item \texttt{process\_proportions\_by\_category}				
\end{itemize}

These observations have the same expectations as the mortality block versions described in Section~\ref{sec:mortality_block}. With the exception that instead of wrapping a mortality block they can wrap any process type available in \CNAME.

\subsubsection{\I{Specific process observations}}
A list of \texttt{types} that are associated with this set of observations are;
\begin{itemize}
	\item \texttt{process\_removals\_by\_age},
	\item \texttt{process\_removals\_by\_length},
	\item \texttt{process\_proportions\_migrating}.	
\\\\	
\paragraph*{\I{Process removals by age}\label{sec:removals-by-age}}	
Removals at age observations are observations of the relative number of individuals at age, partway through a process of type \texttt{mortality\_instantaneous}. This observation is exclusively associated with the process of type \texttt{mortality\_instantaneous}, and will error out if associated with any other process type.

The observation is supplied for a given year and time-step, for some selected age classes of the population (i.e., for a range of ages multiplied by a selectivity that is associated with the process).

The age range must be ages defined in the partition (i.e., between \commandsub{model}{min\_age} and \commandsub{model}{max\_age} inclusive), but the upper end of the age range can optionally be a plus group --- which must be either the same or less than the plus group defined for the partition. 

The expectations from this observation are generated whilst the process is being executed. The expectation of numbers at age $a$ for category $c$ from exploitation method $m$ ($E[N_{a,c,m}]$) are defined as,


\begin{equation}
E[N_{a,c,m}] = N_{a,c} U_{a,m} S_{a,c,m} 0.5 M_{a,c}
\end{equation}

where, $N_{a,c}$ are the numbers at age in category $c$ before the process is executed, $U_{a,m}$ is the exploitation rate for age $a$ from method $m$. $S_{a,c,m}$ is the selectivity and $M$ is the natural mortality. These are all relevant to the time step which the user defines.
\\\\
The observation class then acquires the variable $E[N_{a,c,m}]$ and applies ageing error if the user has specified it. Then it amalgamates the observations by method and category depending on how the user specifies the observation, before converting numbers at age to proportions and sending them to the likelihood to be evaluated.
\\\\
Likelihoods that are available for this observation class are the mulitnomial, dirichlet and the lognormal. See Section~\ref{sec:likelihood-observations} for information on the respected likelihood.
\\\\
\paragraph*{\I{Process removals by length}\label{sec:removals-by-length}}
Removals by length observations are observations of the relative number of individuals at length, partway through a process of type \texttt{mortality\_instantaneous}. This observation is exclusively associated with the process of type \texttt{mortality\_instantaneous}, and will error out if associated with any other process type.
\\\\
The observation is supplied for a given year and time-step, for some selected age classes of the population (i.e., for a range of ages multiplied by a selectivity that is associated with the process).

The expectations from this observation are generated whilst the process is being executed. The expectation of numbers at age $a$ for category $c$ from exploitation method $m$ ($E[N_{a,c,m}]$) are defined as,


\begin{equation}
E[N_{a,c,m}] = N_{a,c} U_{a,m} S_{a,c,m} 0.5 M_{a,c}
\end{equation}

where, $N_{a,c}$ are the numbers at age in category $c$ before the process is executed, $U_{a,m}$ is the exploitation rate for age $a$ from method $m$. $S_{a,c,m}$ is the selectivity and $M$ is the natural mortality. These are all relevant to the time step which the user defines.

The observation class acquires the variable $E[N_{a,c,m}]$ from the process and applies the age-length relationship specified in the model. This converts numbers at age to numbers at age and length, where \CNAME\ then converts to numbers at length. Then it amalgamates the observations by method and category depending on how the user specifies the observation, before converting numbers at age to proportions and sending them to the likelihood to be evaluated.
{\small{\begin{verbatim}
@observation observation_fishery_LF
type process_removals_by_length
...
years  1993 1994 1995 
method_of_removal FishingEast
mortality_instantaneous_process instant_mort
length_plus_group false
length_bins 0 20 40 60 80 110
delta 1e-5
table obs
1993    0.0   0.05    0.05    0.10    0.80  
1994    0.05  0.1     0.05    0.05    0.75  
1995    0.3   0.4     0.2     0.05    0.05  
end_table

table error_values
1993 31
1994 34
1995 22
end_table   
\end{verbatim}}}

Likelihoods that are available for this observation are the mulitnomial, dirichlet and the lognormal. See Section~\ref{sec:likelihood-observations} for information on the respected likelihood.

\paragraph*{\I{Proportions migrating}\label{sec:Proportions-migrating}}
This observation is of the proportion migrating from one area to another. This observation is exclusively associated with the process type \texttt{transition\_category}, and will error out when trying to associate with any other process type. This observation is used to inform migration rates in migration processes. This observation class is used in the Hoki stock assessment see~\cite{francis_03} for more information on how these observations are collected and the situation you would use it. This observation calculates an expectation $E_a$ of proportions for each age class $a$ that have migrated, by evaluating the following,

\begin{equation}
E_a = \frac{N_a - N_a'}{N_a}
\end{equation}
where, $N_a$ are the numbers of individuals in age $a$ before the migration process occurs and $N_a'$ is the number of individuals after the migration process occurs.
\\
The likelihoods that are allowed for this observation are the lognormal, multinomial and dirichlet.
\\\\
An extract of the Hoki stock assessment is as follows,
{\small{\begin{verbatim}
		@observation pspawn_1993
		type process_proportions_migrating
		years 1993
		time_step step4
		process Wspmg ## migration process that the observation is associated with
		age_plus true
		min_age 4
		max_age 9
		likelihood lognormal
		categories male.west+female.west ## Categories to evaluate the prportion for
		ageing_error Normal_offset ## label for an @ageing_error block
		table obs
		#age    4    5    6    7    8    9
		1993 0.64 0.58 0.65 0.66 0.71 0.60
		end_table
		
		table error_values
		## if lognormal these are c.v.'s
		1993 0.25
		end_table
		\end{verbatim}}}
			
\end{itemize}

\subsection{\I{Likelihoods}\label{sec:likelihood-observations}\index{Likelihoods}}

\subsubsection{Likelihoods for proportions-at-age observations}
\CNAME\ implements three likelihoods for proportions-at-age observations, the multinomial likelihood, dirichlet, and the lognormal likelihood. 

\subsubsection*{The multinomial likelihood\index{Multinomial likelihood}}
For the observed proportions at age $O_i$ for age classes $i$, with sample size $N$, and the expected proportions at the same age classes $E_i$, the negative log-likelihood is defined as; 

\begin{equation}
-\log \left(L \right) =  -\log \left(N! \right) + \sum\limits_i \log \left( \left(NO_i \right)! \right) - NO_i \log \left(Z \left(E_i,\delta \right) \right)
\end{equation}

where $\sum\limits_i O_i = 1$ and $\sum\limits_i E_i = 1$. $Z \left(\theta,\delta \right)$ is a robustifying function to prevent division by zero errors, with parameter $\delta>0$. $Z \left(\theta,\delta \right)$ is defined as,

\begin{equation}
Z \left(\theta,\delta \right) = \begin{cases}
\theta, & \text{where $\theta \ge r$} \\
\delta/\left( 2-\theta/\delta \right), & \text{otherwise} \\  
\end{cases}
\end{equation}

The default value of $\delta$ is $1 \times 10^{-11}$.
\subsubsection*{The dirichlet likelihood\index{Dirichlet likelihood}}

For the observed proportions at age $O_i$ for age classes $i$, with sample size $N$, and the expected proportions at the same age classes $E_i$, the negative log-likelihood is defined as; 

\begin{equation}
-\log \left(L \right) = -\log(\Gamma \sum\limits_i (\alpha_i)) + \sum\limits_i \log(\Gamma (\alpha_i)) - \sum\limits_i (\alpha_i-1) \log(Z(O_i,\delta))
\end{equation}

where $\alpha_i = Z \left(N E_i,\delta \right)$, $\sum\limits_i O_i = 1$, and $\sum\limits_i E_i = 1$. $Z \left(\theta,\delta \right)$ is a robustifying function to prevent division by zero errors, with parameter $\delta>0$. $Z \left(\theta,\delta \right)$ is defined as,

\begin{equation}
Z \left(\theta,\delta \right) = \begin{cases}
\theta, & \text{where $\theta \ge r$} \\
\delta/\left( 2-\theta/\delta \right), & \text{otherwise} \\  
\end{cases}
\end{equation}

The default value of $\delta$ is $1 \times 10^{-11}$.

\subsubsection*{The lognormal likelihood\index{Lognormal likelihood}}

For the observed proportions at age $O_i$ for age classes $i$, with c.v. $c_i$, and the expected proportions at the same age classes $E_i$, the negative log-likelihood is defined as; 

\begin{equation}
- \log \left(L \right) = \sum\limits_i \left( \log \left( \sigma _i \right) + 0.5\left( \frac{\log \left(O_i / Z \left(E_i,\delta \right) \right)}{\sigma_i} + 0.5 \sigma_i \right)^2 \right)
\end{equation}

where 

\begin{equation}
\sigma_i  = \sqrt{\log \left(1+c_i^2 \right)}
\end{equation}

and the $c_i$'s are the c.v.s for each age class $i$, and $Z \left(\theta,\delta \right)$ is a robustifying function to prevent division by zero errors, with parameter $\delta>0$. $Z \left(\theta,\delta \right)$ is defined as,

\begin{equation}
Z \left(\theta,\delta \right) = \begin{cases}
\theta, & \text{where $\theta \ge r$} \\
\delta/\left( 2-\theta/\delta \right), & \text{otherwise} \\  
\end{cases}
\end{equation}

The default value of $\delta$ is $1 \times 10^{-11}$.

\subsubsection{Likelihoods for abundance and biomass observations}\label{Obs:biomass}
Abundance and biomass observations are expected as an annual time series in \CNAME, where they select the same categories over that time series. The parameters and inputs needed to use this observation class are: a observation $O_i$, c.v. $c_i$, catchability coefficient $q$, where $i$ indexed the year. \CNAME\ calculates an expectation $E_i$ and scales it by $q$ before comparing it to $O_i$. This means that the value chosen for $q$ will determine whether the observation is relative ($q\neq 1$) or absolute $q = 1$. Before we describe each of the likelihoods we will discuss the methods available to handle $q's$:

\begin{enumerate}
	\item The $q's$ can be treated as ‘nuisance’ parameters. For each set of values of the free parameters, the model uses the values of the $q's$which minimise the objective function. These optimal $q's$ are calculated algebraically (see Section~\ref{subsec:nuisance}). If one of the $q's$ falls outside the bounds specified by the user, it is set equal to the closest bound. This approach reduces the size of the parameter vector and hence should improve the performance of the estimation method. However, it is not correct when calculating a sample from the posterior in a Bayesian analysis (except asymptotically, see \cite{Walters_ludwig_94}) and we offer the following alternative;
	
	\item The $q's$ can be treated as ordinary free parameters.
\end{enumerate}	
	
For both options, it is necessary to evaluate the contribution of $O_i$ to the negative loglikelihood for a given value of $q$. Each observation $O_i$ varies about $qE_i$ — express the variability of $O_i$ in terms of its c.v. $c_i$ (or in one case, its standard deviation si). Here are the likelihoods, which are expressed on the objective-function scale of -log(L):


\subsubsection*{The lognormal likelihood\index{Lognormal likelihood}\index{Lognormal likelihood}}

The negative log likelihood for a the lognormal is as follows,

\begin{equation}
- \log \left(L \right) = \sum\limits_i \left( \log \left( \sigma _i \right) + 0.5\left( \frac{\log \left(O_i / q Z \left(E_i,\delta \right) \right)}{\sigma_i} + 0.5 \sigma_i \right)^2 \right)
\end{equation}

where 

\begin{equation}
\sigma_i  = \sqrt{\log \left(1+c_i^2 \right)}
\end{equation}

and $Z \left(\theta,\delta \right)$ is a robustifying function to prevent division by zero errors, with parameter $\delta>0$. $Z \left(\theta,\delta \right)$ is defined as,

This reflects the distributional assumptions that  $O_i$ has the lognormal distribution, that the mean of $O_i$ is $qE_i$  and the c.v. of $O_i$ is $c_i$.

\begin{equation}
Z \left(\theta,\delta \right) = \begin{cases}
\theta, & \text{where $\theta \ge r$} \\
\delta/\left( 2-\theta/\delta \right), & \text{otherwise} \\  
\end{cases}
\end{equation}

The default value of $\delta$ is $1 \times 10^{-11}$.

\subsubsection*{The normal likelihood\index{Normal likelihood}\index{Normal likelihood}}

For observations $O_i$, c.v. $c_i$, and expected values $qE_i$, the negative log-likelihood is defined as;

\begin{equation}
- \log \left(L \right) = \sum\limits_i \left( \log \left( c_i E_i \right) +0.5 \left( \frac{O_i-E_i}{Z\left(c_i E_i,\delta \right)}\right)^2\right)
\end{equation}

and $Z \left(\theta,\delta \right)$ is a robustifying function to prevent division by zero errors, with parameter $\delta>0$. $Z \left(\theta,\delta \right)$ is defined as,

\begin{equation}
Z \left(\theta,\delta \right) = \begin{cases}
\theta, & \text{where $\theta \ge r$} \\
\delta/\left( 2-\theta/\delta \right), & \text{otherwise} \\  
\end{cases}
\end{equation}

The default value of $\delta$ is $1 \times 10^{-11}$.

This reflects the distributional assumptions that  $O_i$ has the normal distribution, that the mean of $O_i$ is $qE_i$  and the c.v. of $O_i$ is $c_i$.

\subsubsection{Likelihoods for tag recapture by age and length observations}
\paragraph*{The binomial likelihood\index{Binomial likelihood ! tag-recapture-by-length}}
Designed for situations where the size frequencies or age frequencies of the recaptured tagged fish and of the scanned fish are known. Available in both age or size based models.
\\\\
Here we define the likelihood as a binomial, but based on sizes, rather than ages,
\begin{equation}
\begin{split}
-\log \left(L \right)'= -\sum\limits_i & \left[ \right. \log \left(n_i! \right) - \log \left(\left(n_i - m_i \right)! \right) - \log \left(\left(m_i \right)! \right) + m_i \log \left(Z\left(\frac{M_i}{N_i},\delta \right) \right) \\
&+  \left(n_i - m_i \right)\log \left(Z\left(1 - \frac{M_i}{N_i},\delta\right) \right) \left. \right]
\end{split}
\end{equation}
where 
\\
$n_i$ = number of fish at size or age $i$ that were scanned
\\
$m_i$ = number of fish at size or age $i$ that were recaptured
\\
$N_i$ = number of fish at size or age $i$ in the available population (tagged and untagged)
\\
$M_i$ = number of fish at size or age $i$ in the available population that have the tag after a detection probability $p_d$ has been applied, $M_i = M_i'p_d$, where $M_i'$ is the expected available population that have the tag.
\\\\
where $Z(x,\delta)$ is a robustifying function with parameter $r > 0$ (to prevent division by zero errors), defined as


\[ Z(x,\delta) =
\begin{cases}
x       & \text{where } x \geq \delta\\
\frac{\delta}{(2 - x / \delta)}  & \text{otherwise}\\
\end{cases}
\]

Finally if a dispersion parameter ($\tau$) is described in the observation then the final negative log likelihood $-log(L)$ contribution is,

$$-log(L) = -log(L)' / \tau$$


\subsubsection{Likelihoods for proportions-by-category observations}
\CNAME\ implements two likelihoods for proportions-by-category observations, the binomial likelihood, and the normal approximation to the binomial (binomial-approx). 

\subsubsection*{The binomial likelihood\index{Binomial likelihood ! proportions-by-category}}

For observed proportions $O_i$ for age class $i$, where $E_i$ are the expected proportions for age class $i$, and $N_i$ is the effective sample size for age class $i$, then the negative log-likelihood is defined as;  

\begin{equation}
\begin{split}
-\log \left(L \right)= -\sum\limits_i & \left[ \right. \log \left(N_i! \right) - \log \left(\left(N_i \left(1 - O_i \right) \right)! \right) - \log \left(\left(N_i O_i \right)! \right) + N_i O_i \log \left(Z\left(E_i,\delta \right) \right) \\
&+ N_i \left(1 - O_i \right)\log \left(Z\left(1 - E_i,\delta\right) \right) \left. \right]
\end{split}
\end{equation}


where $Z \left(\theta,\delta \right)$ is a robustifying function to prevent division by zero errors, with parameter $\delta>0$. $Z \left(\theta,\delta \right)$ is defined as,

\begin{equation}
Z \left(\theta,\delta \right) = \begin{cases}
\theta, & \text{where $\theta \ge r$} \\
\delta/\left( 2-\theta/\delta \right), & \text{otherwise} \\  
\end{cases}
\end{equation}

The default value of $\delta$ is $1 \times 10^{-11}$.

\subsubsection*{The normal approximation to the binomial likelihood\index{Binomial likelihood (normal approximation) ! proportions-by-category}}

For observed proportions $O_i$ for age class $i$, where $E_i$ are the expected proportions for age class $i$, and $N_i$ is the effective sample size for age class $i$, then the negative log-likelihood is defined as;  

\begin{equation}
-\log \left(L \right)= \sum\limits_i \log \left( \sqrt{Z\left(E_i,\delta \right)Z\left(1-E_i,\delta\right)/N_i} \right)     + \frac{1}{2} \left( \frac{O_i-E_i}{\sqrt{Z\left(E_i,\delta\right)Z\left(1-E_i,\delta \right)/N_i}} \right)^2
\end{equation}

where $Z \left(\theta,\delta \right)$ is a robustifying function to prevent division by zero errors, with parameter $\delta>0$. $Z \left(\theta,\delta \right)$ is defined as,

\begin{equation}
Z \left(\theta,\delta \right) = \begin{cases}
\theta, & \text{where $\theta \ge r$} \\
\delta/\left( 2-\theta/\delta \right), & \text{otherwise} \\  
\end{cases}
\end{equation}

The default value of $\delta$ is $1 \times 10^{-11}$.

\subsection{\I{Process error}}

Additional `process error' can be defined for each set of observations. Additional process error has the effect of increasing the observation error in the data, and hence of decreasing the relative weight given to the data in the fitting process. 

For observations where where the likelihood is parameterised by the c.v., you can specify the process error for a given set of observations as a c.v., in which case all the c.v.s $c_i$ are changed to

\begin{equation}
  c'_i  = \sqrt {c_i^2  + c_{process\_error}^2 } 
\end{equation}

Note that $c_{process\_ error} \ge 0$, and that $c_{process\_ error} = 0$ is equivalent to no process error.

Similarly, if the likelihood is parameterised by the effective sample size $N$,

\begin{equation}
 N'_i  = \frac{1}{1 / {N_i}+ 1 / N_{process\_error}}
\end{equation}

Note that this requires that $N_{process\_ error} > 0$, but we allow the special case of $N_{process\_ error}=0$, and define $N_{process\_ error}=0$ as no process error (i.e., defined to be equivalent to $N_{process\_ error}=\infty$). 

For both the c.v. and $N$ process errors, the process error has more effect on small errors than on large ones. Be clear that a large value for the $N$ process error means a small process error.

\subsection{\I{Calculating nuisance q's}}\label{subsec:nuisance}
This section describes the theory used to calculate nuisance (analytical) catchability coefficients $q’s$ (see Section~\ref{Obs:biomass}). From the user's point of view, the essence is that you can use nuisance $q’s$ in the following situations:
\begin{enumerate}
	\item With maximum likelihood.
	\item With Bayesian estimation, providing that your provide an additional prior on the q is one of the following:
		\begin{itemize} 
			\item none (default)
			\item Uniform-log
			\item Lognormal with observations distributed lognormal, robustified lognormal
		\end{itemize}
\end{enumerate}
Table~\ref{tab:nus_overview} displays the scenarios when the nuisance catchability can be used for a Bayesian analysis.

\begin{table}[h!]
	\caption{\textbf{Equations used to calculate nuisance $q$'s. (*=no analytic solution found.)}}\label{tab:nus_overview}
	\begin{tabular}{cccccc}
		Distribution of observations & Maximum Likelihood & None & Uniform-log & Normal & lognormal\\
		\hline
		Normal & \eqref{EQ:1} & \eqref{EQ:1} & \eqref{EQ:3} & \textbf{*} & \textbf{*} \\
		Lognormal & \eqref{EQ:4} & \eqref{EQ:4} & \eqref{EQ:8} & \textbf{*} & \eqref{EQ:9} \\				
	\end{tabular}
\end{table}
%% Insert table 4 here
Note that $q’s$ are calculated for robustified lognormal likelihoods as if they were ordinary lognormal likelihoods.
\\\\
The equations and their derivations follow. Let $\sigma_i = \sqrt{log(1 + c_i^2)}$ throughout, and let $n$ be the number of observations in the time series. The case of multiple time series sharing the same $q$, and the modifications required for the assumption of curvature, are addressed at the end of this subsection.
\\\\
First, consider maximum likelihood estimation. When the ($Oi$) are assumed to be normally
distributed,
\begin{equation}\label{EQ:1}
-log(L) = \sum_i log (c_iqE_i) + 0.5\sum_i \bigg(\frac{O_i - qE_i}{c_iqE_i} \bigg)^2
\end{equation}
The value of $q$ which minimises the objective function is found by solving for $q$ under the following condition, $\partial/\partial q(-log(L)) = 0$

\begin{equation}\label{EQ:2}
\frac{\partial }{\partial q}(-log(L)) = \frac{n}{q} + \frac{1}{q^2} \sum_i \frac{O_i}{c_i^2E_i} - \frac{1}{q^3} \sum_i \bigg(\frac{O_i}{c_iE_i}\bigg)^2
\end{equation}
hence
\begin{equation}\label{EQ:3}
\hat q = \frac{-S_1 + \sqrt{S_1^2 + 4nS_2}}{2n} 
\end{equation}
where $S_1 = \sum_i (O_i/c_i^2E_i)$ and $S_2 = \sum_i (O_i/c_iE_i)^2$
\\\\
When the ($O_i$) are assumed to be lognormally distributed,
\begin{equation}\label{EQ:4}
-log(L) = \sum_i log (\sigma_i) + 0.5\sum_i \bigg(\frac{log(O_i) - log(qE_i) + 0.5\sigma_i^2}{\sigma_i} \bigg)^2
\end{equation}
\begin{equation}\label{EQ:5}
\frac{\partial }{\partial q}(-log(L)) = \frac{-1}{q} \sum_i\bigg( \frac{log(O_i/E_i) - log(q) + 0.5\sigma_i^2}{\sigma_i^2}\bigg)
\end{equation}

\begin{equation}\label{EQ:6}
\hat q = exp\frac{0.5n + S_3}{S_4} 
\end{equation}

where $S_3 = \sum_i (log(O_i /E_I)/\sigma_i^2)$ and $S_4 = \sum_i(1/\sigma_i^2)$ 
\\\\
Next consider Bayesian estimation, where we must also specify a prior for $q$.
\\\\
The effects of the prior on the equations are to replace likelihood $L$ by posterior $P$ throughout, to add $-log(\pi(q))$ to the equation for $-log(P)$ and $\partial/\partial q(-log(-\pi(q)))$ to the equation for $\partial/\partial q(-log(P))$
\\\\
This last term is 0 for a uniform prior on $q$, $1/q$ for a log-uniform prior, and $\frac{1}{q}\bigg( 1.5 + \frac{log(q) - log(\mu_q)}{\sigma_q^2}$ for a lognormal prior, 
\\\\
where $\mu_q$ and $c_q$ are the mean and c.v of the prior on $q$ and $\sigma_q = \sqrt{log(1+c_q^2)}$. Clearly, if the prior is uniform, the equation for $\hat q$ is teh same as teh maximum likelihood estimation.
\\\\
When the $(O_i)$ are assumed to be normally distributed and teh prior is log-uniform equation~\eqref{EQ:3} becomes,

\begin{equation}\label{EQ:7}
\hat q = \frac{-S_1 + \sqrt{S_1^2 + 4(n + 1)S_2}}{2(n+1)} 
\end{equation}

but we cannot solve for $\hat q$ with either a normal or lognormal prior.
\\\\
When the $O_i$ are assumed to be lognormally distributed and the prior is log-uniform, equation~\eqref{EQ:6} becomes 


\begin{equation}\label{EQ:8}
\hat q = exp\frac{0.5n -1 + S_3}{S_4} 
\end{equation}

and if the prior is lognormal,

\begin{equation}\label{EQ:9}
\hat q = exp\frac{0.5n -1.5 + log(\mu_q)/\sigma_q^2 + S_3}{S_4 + 1 / \sigma_q^2} 
\end{equation}
but it is not possible to solve for $\hat q$ with a normal prior. An example of specifying the syntax and an equivalent additional prior see below
{\small{\begin{verbatim}
	@catchability chatTANq
	type nuisance
	upper_bound 0.6
	lower_bound 0.0001
	
	@additional_prior chatTANq_prior
	type lognormal
	parameter catchabilityp[chatTANq].q
	mu 0.3
	cv 0.2	
	\end{verbatim}}}


\subsection{\I{Ageing error}}

\CNAME\ can apply ageing error to expected age frequency generated by the model. The ageing error is applied as a misclassification matrix, which has the effect of 'smearing' the expected age frequencies. This is mimicking the error involved in identifying the age of individuals. For example fish species are aged by reading the ear bones (otoliths) which can be quite difficult depending on the species. These are used in calculating the fits to the observed values, and hence the contribution to the total objective function. 

Ageing error is optional, and if it is used, it may be omitted for any individual time series. Different ageing error models may be applied for different observation commands. See Section \ref{sec:ageingerrorreport} for reporting the misclassification matrix at the end of model run.

The ageing error models implemented are,
\begin{enumerate}
  \item{None}: The default model is to apply no ageing error.
  \item{Off by one}: Proportion $p_1$ of individuals of each age $a$ are misclassified as age $a-1$ and proportion $p_2$ are misclassified as age $a+1$. Individuals of age $a < k$ are not misclassified. If there is no plus group in the population model, then proportion $p_2$ of the oldest age class will 'fall off the edge' and disappear. 
  \item{Normal}: Individuals of age $a$ are classified as ages which are normally distributed with mean $a$ and constant c.v. $c$. As above, if there is no plus group in the population model, some individuals of the older age classes may disappear. If $c$ is high enough, some of the younger age classes may 'fall off the other edge'. Individuals of age $a < k$ are not misclassified.
\end{enumerate}

Note that the expected values (fits) reported by \CNAME\ for observations with ageing error will have had the ageing error applied. 



\subsection{\I{Simulating observations}\label{sec:simulation-observations}}

\CNAME\ can generate simulated observations for a given model with given parameter values using \texttt{casal2 -s 1} (To simulate one set of simulated observations). Simulated observations are randomly distributed values, generated according to the error assumptions defined for each observation, around fits calculated from one or more sets of the 'true' parameter values. Simulating from a set of parameters can be used to generate observations from an operating model or as a form of parametric bootstrap. 

The procedure \CNAME\ uses for simulating observations is to first run using the `true' parameter values and generate the expected values. Then, if a set of observations uses ageing error, ageing error is applied. Finally a random value for each observed value is generated based on (i) the expected values, (ii) the type of likelihood specified, and (iii) the variability parameters (e.g., \subcommand{error\_value} and \subcommand{process\_error}). 

Methods for generating the random error, and hence simulated values, depend on the specific likelihood type of each observation. 

\begin{enumerate}
  \item{} Normal likelihood parameterised by c.v.: Let $E_{i}$ be the fitted value for observation $i$, and $c_i$ be the corresponding c.v. (adjusted by the process error if applicable). Each simulated observation value $S_i$ is generated as an independent normal deviate with mean $E_i$ and standard deviation $E_i c_i$.
  \item{} Log-normal likelihood: Let $E_i$ be the fitted value for observation $i$ and $c_i$ be the corresponding c.v. (adjusted by the process error if applicable). Each simulated observation value $S_i$ is generated as an independent lognormal deviate with mean and standard deviation (on the natural scale, not the log-scale) of $E_i$ and $E_i c_i$ respectively. The robustification parameter $\delta$ is ignored.
  \item{} Multinomial likelihood: Let $E_i$ be the fitted value for observation $i$, for $i$ between $1$ and $n$, and let $N$ be the sample size (adjusted by process error if applicable, and then rounded up to the next whole number). The robustification parameter $\delta$ is ignored. Then, 
  \begin{enumerate}
    \item{} A sample of $N$ values from $1$ to $n$ is generated using the multinomial distribution, using sample probabilities proportional to the values of $E_i$.
    \item{} Each simulated observation value $S_i$ is calculated as the proportion of the $N$ sampled values equalling $i$
    \item{} The simulated observation values $S_i$ are then rescaled so that their sum is equal to $1$
  \end{enumerate}
\item{} Binomial and the normal approximation to the binomial likelihoods: Let $E_i$ be the fitted value for observation $i$, for $i$ between $1$ and $n$, and $N_i$ the corresponding equivalent sample size (adjusted by process error if applicable, and then rounded up to the next whole number). The robustification parameter $\delta$ is ignored. Then, 
  \begin{enumerate}
    \item{} A sample of $N_i$ independent binary variates is generated, equalling $1$ with probability $E_i$ 
    \item{}	The simulated observation value $S_i$ is calculated as the sum of these binary variates divided by $N_i$
  \end{enumerate}
\end{enumerate}

\textbf{An important note when simulating:} \CNAME\ will \textbf{not} automatically report simulated observations when users undertake a \texttt{casal2 -s 1} run, you must write an explicit report using the \texttt{simulated\_observation} report (\commandlabsubarg{report}{type}{observation}). See Section \ref{sec:report-section} for more information on how to write this report.


\subsection{\I{Pseudo-observations}}
\CNAME\ can generate expected values for observations without them contributing to the total objective function. These are called pseudo-observations, and can be used to either generate the expected values from \CNAME\ for reporting or diagnostic purposes. To define an observation as a pseudo-observation, use the command \commandlabsubarg{observation}{likelihood}{none}. Any observation type can be used as a pseudo-observation. \CNAME\ can also generate simulated observations from pseudo-observations. Note that;

\begin{itemize}
  \item Output will only be generated if a report command \commandlabsubarg{report}{type}{observation} is specified.
  \item The observed values should be supplied (even if they are `dummy' observation). These will be processed by \CNAME\ as if they were actual observation values, and must conform to the validations carried out for the other types of likelihood. 
  \item The subcommands \subcommand{likelihood}, \subcommand{obs}, \subcommand{error\_value} and \subcommand{process\_error} have no effect when generating the expected values for the pseudo-observation.   
  \item When simulating observations, \CNAME\ needs the subcommand \subcommand{simulation\_likelihood} to tell it what sort of likelihood to use. In this case, the \subcommand{obs}, \subcommand{error\_value} and \subcommand{process\_error} are used to determine the appropriate terms to use for the likelihood when simulating.
\end{itemize}

\subsection{\I{Residuals}}\label{sec:Residuals}}
\CNAME\ will only print the usual residual (i.e. observed less fitted) using the report type \command{report}.type=observation. For an observation \textit{O} and \textit{F} the corresponding fit (=\textit{qE} for relative observations), then
\begin{itemize}
	\item Residuals = \textit{O} - \textit{F}
\end{itemize}

Pearson and Normalised residuals can be generated using \CNAME\ \textbf{R} package with-in the \textbf{R} environment. For specific R functions see Section~\ref{sec:post-processing}. The definitions used in the calculations are as follows,

\begin{enumerate}
	\item \textit{Pearson residuals} attempt to express the residual relative to the variability of the observation, and are defined as (\textit{O}-\textit{F})/std.dev.(\textit{O}), where std.dev.(\textit{O}) is calculated as
	\begin{itemize}
			\item F $\times$ cv for normal, lognormal, robustified lognormal, and normal-log error distributions.
			\item s for normal-by-standard deviation error distributions.
			\item $\sqrt{\frac{Z(\textit{F},r)(1 - Z(\textit{F},r))}{N}}$ for multinomial or binomial likelihoods.
			\item $\sqrt{\frac{(\textit{F} + r)(1 - \textit{F} + r)}{N}}$ for binomial-approx likelihood likelihoods.
	\end{itemize}
	\item \textit{Normalised residuals} to express the residual on a standard normal scale, and are defined as:
	\begin{itemize}
		\item Equal to the Pearson residuals for normal error distributions.
		\item (log(\textit{O}/\textit{F})+0.5$\sigma^2$)/$\sigma$ for lognormal (including robustified lognormal) error distributions, where $\sigma= \sqrt{log(1 + cv^2)}$.
		\item  log(\textit{O}/\textit{F})/$\sigma$ for normal-log error distributions, again with $\sigma= \sqrt{log(1 + cv^2)}$.
		\item And are otherwise undefined.
	\end{itemize}	
\end{enumerate}

where $Z(\textit{F},r)$ is the robustifying term on \textit{F} (fit or expectation of the observation). This robustifying is described earlier in the likelihood section.


