\section{\I{The Projection section}\label{sec:project-section}}\index{Projections}\index{Projection section}

Projection is the process of running the model forwards into the future, using stochastic and or deterministic values for population dynamic parameters, such as recruitments and catches. In a projection run in \CNAME\, a model is initialised and run through the model years from \argument{initial} to the \argument{final}. Then, the model is re run from \argument{intial} to \argument{projection\_final\_year}, where any parameter can be fixed or drawn from a stochastic process between this time period. \CNAME\ does not have default projections. Users must specify them using the \command{project} blocks. This is important for parameters that are year specific such as year class parameters. If there is no \command{project} for these parameters, they will not exist after \argument{final\_year} processes that call them will cause nonsensical output. Any estimable parameter can be projected forward. The types of processes where parameters are drawn during the projection years are: constant, LogNormal and empirical sampling.

\subsection*{\I{Constant}}\index{Projections ! Constant}
A parameter can be fixed during all projection or can be specified for all years. This assumes 

\subsection*{\I{Empirical Sampling}}\index{Projections ! Constant}
Parameters that are of type vector or map can be re sampled with replacement between a year range for projected years
\subsection*{\I{LogNormal}}\index{Projections ! LogNormal}
The randomised parameter are lognormally distributed, with mean 1, and specified standard deviation and autocorrelation on the log-scale. YCSi=exp(Xi), where (Xi) are generated as a Gaussian AR(1) process with standard deviation $\sigma_R$ and mean-0.5 $\sigma_R$ (so that the mean of YCSi is 1), and autocorrelation $\rho$. Set $\rho=0$, the default, if you don’t want autocorrelation. If the randomised parameter are modified by an arbitrary multiplier, then the only change is that parameter will have mean $\mu$, where $\mu$ is the recruitment multiplier.

