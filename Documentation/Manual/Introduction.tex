\section{Introduction\label{sec:Introduction}}
\CNAME\ is a generalised age-structured population dynamics modelling software package that allows flexibility in specifying model structure, population dynamics, parameter estimation, and model outputs. \CNAME\ can model population dynamics for an age-structured population using a range of population dynamics observations, including mark-recapture, relative and absolute abundance time series, and age frequency data. \CNAME\ does this by implementing an age-structured population dynamics model that can have user defined categories (e.g., immature, mature, male, female, predator, prey, etc.,) to specify the population structure, and a user-defined age range. 

This manual describes how to use \CNAME, including how to run \CNAME, how to set up an \config. Further, we describe the population dynamics, observations, and estimation methods, and describe how to specify and interpret output. 

\subsection{Version\label{sec:version}}
This document (last modified \DocVer) describes \CNAME\ \VER. The \CNAME\ version number is suffixed with a date/time (\texttt{yyyy-mm-dd}), giving the revision control system UTC date for the most recent modification of the underlying software source code. User manual updates will usually be issued for each minor version or date release of \CNAME. Any questions of the use of the software can be directed to the authors at \email. \index{Version number}

\subsection{Citing \CNAME}
A suitable reference for \CNAME\ and this document is:

\ManualRef\index{Citation}\index{Citing \CNAME}

\subsection{\I{Software license}\index{GNU GPL v2 licence}}
\
This program and the accompanying materials are made available under the terms of the licence \href{http://www.gnu.org/licenses/old-licenses/gpl-2.0.en.html}{GNU GPL v2} which accompanies this software (see Section \ref{sec:gpl-2.0}).

Copyright \copyright 2015-\SourceControlYearDoc, \href{http://www.niwa.co.nz}{\Organisation}. All rights reserved.

\subsection{\I{System requirements}}
\CNAME\ is available for most IBM compatible machines running 64-bit \I{Linux} and \I{Microsoft Windows} operating systems.

Several of \CNAME s tasks are highly computer intensive and a fast processor is recommended. Depending on the model implemented, some of \CNAME s tasks can take a considerable amount of time (minutes to hours), and in extreme cases can even take several days to undertake an MCMC estimate. 

The program itself requires only a few megabytes of hard-disk space but output files can consume large amounts of disk space. Depending on number and type of user output requests, the output could range from a few hundred kilobytes to several hundred megabytes. When estimating model fits, several hundred megabytes of RAM may be required, depending on the spatial size of the model, number of categories, and complexity of processes and observations. For extremely large models, several gigabytes of RAM may occasionally be required. 

\subsection{\I{Necessary files}}

For both 64-bit Linux and Microsoft Windows, only the binary file \texttt{casal2} or \texttt{casal2.exe} is required to run \CNAME . No other software is required. We do not compile a version for 32-bit operating systems. 

\CNAME\ offers little in the way of post-processing of model output, and a package available that allows tabulation and graphing of model outputs is recommended. We suggest software such as \href{http://www.r-project.org}{\R}\ \citep{R} to assist in the post processing of \CNAME\ output. We provide the \texttt{CASAL2} \R\ package for importing the \CNAME\ output into \R\ (see Section \ref{sec:post-processing}).

\subsection{Getting help\index{Getting help}\index{User assistance}\index{Notifying errors}}

\CNAME\ is distributed as unsupported software, however we would appreciate being notified of any problems or errors in \CNAME. See Section \ref{sec:reporting-errors} for the recommended template for reporting issues. For further information on \CNAME\, please contact the development team at \email.

\subsection{Technical details\index{Technical specifications}}\label{sec:tech}

\CNAME\ was compiled on Linux using \href{http://gcc.gnu.org}{\texttt{gcc}}, the C/C++ compiler developed by the \href{http://gcc.gnu.org}{GNU Project}. The 64-bit Linux \index{Linux} version was compiled using \texttt{gcc} version 5.2.1 20151010 (\href{http://www.ubuntu.com/}{Ubuntu Linux}). The \href{http://www.microsoft.com}{Microsoft Windows}\index{Microsoft Windows} version was compiled using \href{http://www.mingw.org}{Mingw32}\index{Mingw} \href{http://gcc.gnu.org}{\texttt{gcc}} (tdm64-1) 5.1.0. The \href{http://www.microsoft.com}{Microsoft Windows} installer was built using the \href{http://www.jrsoftware.org/isdl.php}{Inno Setup 5 application}.

\CNAME\ includes six different minimisers --- Different minimisers may be better at some models that others. The first three are non-differentiation based minimisers: the first is closely based on the main algorithm of \cite{779}, and which which uses finite difference gradients\index{Finite differences minimiser}; the second is an implementation of the differential evolution solver\index{Differential evolution minimiser} \citep{1442}, and based on code by \href{mailto:<godwin@pushcorp.com>}{Lester E. Godwin} of \href{http://www.pushcorp.com}{PushCorp, Inc.}; and the third is Dlib \citep{dlib09}. The three differentiation based minimisers are: ADOLC, an auto differentiation minimiser \citep{walther1996adolc}; CPPAD is an auto differentiation minimiser similar to ADOLC \citep{wachter2006cppad}; and the third is a modified version of an older version of ADOL-C (v1.8.4) that was used in the original version of CASAL \citep{1388}.

The random number generator\index{Random number generator} used by \CNAME\ uses an implementation of the Mersenne twister random number generator \citep{796}. This, the command line functionality, matrix operations, and a number of other functions use the \href{http://www.boost.org/}{BOOST} C++ library (Version 1.58.0)\index{BOOST C++ library}.

Note that the output from \CNAME\ may differ slightly on the different platforms due to different precision arithmetic or other platform dependent implementation issues. The source code\index{\CNAME\ source code} for \CNAME\ is available in the \href{ftp://ftp.niwa.co.nz/incoming/CASAL2_auto_build/casal2.tar.gz}{windows bundle} or on the github repository at \github.

Unit tests of the underlying \CNAME\ code are carried out at build time, using the \href{http://www.boost.org/}{GOOGLE} mock and unit testing framework. The unit test framework aims to cover a significant proportion of the key functionality within the \CNAME\ code base. The unit test code for \CNAME\ is available as a part of the underlying source code.

