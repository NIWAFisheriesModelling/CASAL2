\section{Introduction}\label{sec:introduction}

This document is an introductory help guide for \CNAME, a generalised age-structured population dynamic modelling package. \CNAME\ is run through the command prompt/terminal where it reads text files (configuration files) which defines the model. \CNAME\ then prints output to a screen or a file or errors out gracefully. This short document is aimed at users who are new to \CNAME. \CNAME\ is primarily used on fish populations, but is no means specific to fish population dynamics. \CNAME's predecessor is the primary tool used in assessing New Zealand's tier one stocks, it is also the standard tool used by CCAMMLR for modelling Antarctic toothfish. 
\\\\
\CNAME\ is very generalised, highly flexible, and therefore can be a bit daunting at first sight. It has a large number of run modes, settings, and user defined population dynamics choices that can be turned on and off, depending on circumstances such as, population life history and available data. While there is no requirement for a user to see or understand the underlying code base, it has been written so that is is well tested. Great effort has been put into developing a code base that can be easily interpreted by even novice programmers.
\\\\
\CNAME\ is open source, and is covered under the GNU GPL 2.0 licence. See the terms and conditions in the \CNAME\ Technical User Manual \citep{CASAL2}, or type \texttt{casal2 -l} into the command prompt. There is also supplementary information that may be useful to access when getting familiar with \CNAME. \CNAME\ has a comprehensive user manual \citep{CASAL2} which should be consulted for detail on any model component. \CNAME\ also has a contributors guide to help users add any functionality they wish to tackle any problem, the modular structure of the code base can make adding new processes, observations and likelihoods a breeze. If you have any questions, please contact the \CNAME\ development team at \email, even though this is unsupported software we are keen to help with any advancement.
\\\\
The remaining content of this chapter describes requirements and details about how to run, cite, licensing and contact info for \CNAME. If you are new to population modelling then section~\ref{sec:data_requirements} describes the types of data that you would need to run a \CNAME\ model. This seemed like a logical start to the document as getting \CNAME\ as you want to know if you have the data before you write the model, unless you are using \CNAME\ as an operating model for simulating data. 
The remaining content of the document goes over how you run \CNAME, the syntax of the configuration files that \CNAME\ uses as inputs, and finally we run through an example.
\\\\
\subsection{Version\label{sec:version}}

\CNAME\ can differ between version, especially as issues are fixed or new features added. The \CNAME\ version number is suffixed with a date/time stamp (\texttt{yyyy-mm-dd}), giving the revision control system UTC date for the most recent modification of the underlying software source code. User manual updates will usually be issued for each minor version or date release of \CNAME.

\subsection{Citing the \CNAME\ Getting Started Guide}
A suitable reference for this document is: 

\ManualRef\index{Citation}\index{Citing \CNAME}
 
\subsection{\I{Software license}\index{GNU GPL v2 licence}}
\
This program and the accompanying materials are made available under the terms of the licence \href{http://www.gnu.org/licenses/old-licenses/gpl-2.0.en.html}{GNU GPL v2} which accompanies this software.

Copyright \copyright 2015-\SourceControlYearDoc, \href{http://www.niwa.co.nz}{\Organisation}. All rights reserved.

\subsection{\I{System requirements}}

\CNAME\ is available for most IBM compatible machines running 64-bit \I{Linux} and \I{Microsoft Windows} operating systems.

Several of \CNAME's tasks are highly computer intensive and a fast processor is recommended. Depending on the model implemented, some of \CNAME's tasks can take a considerable amount of time (minutes to hours), and in extreme cases can even take several days to undertake an MCMC estimate. 

The program itself requires only a few megabytes of hard-disk space but output files can consume large amounts of disk space. Depending on number and type of user output requests, the output could range from a few hundred kilobytes to several hundred megabytes. When estimating model fits, several hundred megabytes of RAM may be required, depending on the spatial size of the model, number of categories, and complexity of processes and observations. For extremely large models, several gigabytes of RAM may occasionally be required. 

\subsection{\I{Necessary files}}

For both 64-bit Linux and Microsoft Windows, only the executable file \texttt{casal2} or \texttt{casal2.exe} is required to run \CNAME with non-auto differentiable minimisers. If you wish to use the auto differentiable minimisers the \texttt{.dll} for windows and \texttt{.so} for Linux must be in the same folder as the executable \CNAME\ files or in your system path. No other software is required. We do not compile a version for 32-bit operating systems. 

\CNAME\ offers little in the way of post-processing of model output, and a package available that allows tabulation and graphing of model outputs is recommended. We suggest software such as \href{http://www.r-project.org}{\R}\ \citep{R} to assist in the post processing of \CNAME\ output. We provide the \texttt{CASAL2} \R\ package for importing the \CNAME\ output into \R\ (see the \CNAME User Manual for more detail).

\subsection{Getting help\index{Getting help}\index{User assistance}\index{Notifying errors}}

\CNAME\ is distributed as unsupported software, however we would appreciate being notified of any problems or errors in \CNAME. See the \CNAME\ User manual \citep{CASAL2} for the recommended template for reporting issues.

